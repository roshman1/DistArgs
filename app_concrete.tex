\section{Concrete Instantiations for the Constructions }\label{app:concrete}
We discuss several possibilities for instantiating the primitives used in our constructions. Throughout the first part of this work (Section~\ref{sec:dargs} and Section~\ref{sec:distprover}), we use several cryptographic primitives, some of them quite strong. Here we refine them and show from which concrete assumptions each of them can be constructed. \TODO{CRH is not so concrete}

\begin{theorem}\label{thm:concDargs}
    Let $\Lan$ be a graph language, such that $\Lan\in \NP$. Assuming Collision Resistant Hash Families,
    \begin{enumerate}
        \item There is a succinct \emph{interactive} distributed argument with 4 rounds of communication for $\Lan$. 
        \item In the Random Oracle model, there is a succinct non-interactive distributed argument for $\Lan$.
        \item In the Common Reference String model, assuming \emph{Knowledge-of-Exponent in Bilinear Groups}, there is a succinct non-interactive distributed argument for $\Lan$. 
    \end{enumerate}
\end{theorem}
From Theorem~\ref{thm:centralized} and ~\cite{merkle1989certified}, we get that part (1) follows from \cite{kilian1992note}, part (2) follows from \cite{micali2000computationally} and part (3) follows from~\cite{bitansky2013SNARKsLIPs}.

\begin{theorem}\label{thm:concDargs}
    Let $\Lan$ be a graph language, such that $\Lan\in \PP$. Assuming Collision Resistant Hash Families, there is a succinct distributed argument for $\Lan$, assuming either
    \begin{enumerate}
        \item The $O(1)-\LIN$ assumption on a pair of cryptographic groups with efficient bilinear map, or
        \item A combination of the sub-exponential $\DDH$ assumption and the $\QR$ assumption.
    \end{enumerate}
\end{theorem}
As shown in  \cite{cryptoeprint:2022/1320}, Flexible RAM SNARGs exist under any of these assumptions. Together with theorem~\ref{theo:P}, this implies corollary \ref{cor:RAMSNARGs}.

\begin{theorem}
    Let $\calD$ be a distributed algorithm that runs in $T = \poly(n)$ rounds and sends messages of length $\poly(n)$. Assuming Sum-Collision-Resistant Hash Families,
    \begin{enumerate}
        \item There is a succinct \emph{interactive} distributed argument with 4 rounds of communication for $\Lan$. 
        \item In the Random Oracle model, there is a succinct non-interactive distributed argument for $\Lan$.
        \item In the Common Reference String model, assuming \emph{Knowledge-of-Exponent in Bilinear Groups}, there is a succinct non-interactive distributed argument for $\Lan$. 
    \end{enumerate}
    Where in all of the cases above, the prover of the distributed argument has a distributed implementation.
\end{theorem}