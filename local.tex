\section{On Polynomial-Time Local Distributed Algorithms}
\label{sec:local}
\TODO{refer to appendix: is this the right version}
In this section we investigate the power of computationally-bounded local decision algorithms:
we define complexity classes for languages decidable by 
such algorithms, and study their relationship to the class of languages
that can be decided by local algorithms with unbounded local computational power,
and to the complexity class $\PP$.
On a high level, our main result is that combining the requirements for locality and computational efficiency
in one algorithm is more restrictive than requiring that the language
be decidable by one algorithm that is local,
and also by another algorithm that is computationally efficient.

\subsection{Definitions}
\label{sec:local_def}

Fix an input domain $\calX$, and a UID space $\calU$,
and let $\calC = \calC(\calX, \calU)$ be the set of all configurations
$(G, x)$
with inputs $x : V(G) \rightarrow \calX$ drawn from $\calX$,
and UIDs drawn from $\calU$.
We let $\calB^t$ be the set of all $t$-neighborhoods that appear in $\calC$:
	$\calB^t = \set{ N_{G,x}^t(v) : (G,x) \in \calC, v \in V(G) }$,
where $N_{G,x}^t(v)$ denotes the $t$-neighborhood of $v$,
including the UIDs and the inputs of the nodes in the $t$-neighborhood.

We model a \emph{$t$-local decision algorithm} as a mapping $\A : \calB^t \rightarrow \set{0,1}$,
which outputs a Boolean value (accept / reject).
We require that $\A$ be a computable function.
As usual, a configuration is accepted by $\A$ iff when $\A$ is executed at every node,
it outputs 1 everywhere.

\begin{definition}[The classes $\LD$, $\LDP$]
	A distributed language $\calL$ is in the class $\LD(t(\cdot))$ 
	if it can be decided in graphs of size $n$ by a $t(n)$-local decision algorithm $\A$.
	If in addition the algorithm $\A$ can be computed by a Turing machine
	that runs in time $\poly(n)$ in graphs of size $n$,
	then $\calL$ is in the class $\LDP(t(\cdot))$.
    %Let $\Lan$ be a distributed language.
    %\paragraph{$t$-Local Decidability} $\Lan$ is in the class $\LD(t)$ if there exists a distributed algorithm $A$, that runs in $t$ rounds, and decides $\Lan$.
    %\paragraph{$t$-Local Polynomial Time Decidability} $\Lan$ is in the class $\LDP(t)$ if there exists a distributed algorithm $A$ and a polynomial $p$, such that $A$ runs in $t$ rounds, and performs up to $p(n)$ steps of local computation, and decides $\Lan$.
\end{definition}

We are interested in algorithms that run in a sublinear number of rounds:
let $\LD = \bigcup_{t(\cdot) \in o(n)} \LD(t(\cdot))$,
and let $\LDP = \bigcup_{t(\cdot) \in o(n)} \LDP(t(\cdot))$.
Note that, as usual in the area of local decision, the local algorithm may not know the size $n$
of the network;
nevertheless, as external observers,
we can study the dependence of the algorithm's locality radius and its local running time on $n$.

%Since we only refer here to \emph{sub-linear} number of rounds, let $\LD$ denote $\bigcup_{t\in o(n)}\LD(t)$, and let $\LDP$ denote $\bigcup_{t\in o(n)}\LDP(t)$.


%, we attend to the matter of computationally-bounded local decision. We define complexity classes for polynomial-time sub-linear-rounds decision and show in Section~\ref{sec:uncond} that it is separated from the intersection between the class of sub-linear rounds distributed decision and centralized polynomial-time decision ($\PP$). When nodes are given the size of the graph, this separation becomes much harder to prove, as demonstrated in Appendix~\ref{app:local}\TODO{where}, proving it unconditionally would be proving that $\PP\neq\NP$. In Section~\ref{sec:owf}, we prove the  separation in the case where nodes know the size of the graph conditioned on the existence of injective one-way functions. In Appendix~\ref{app:NLD}\TODO{where}, we show that the distinction disappears when non-determinism is introduced, and non-deterministic polynomial-time local decision in fact equals in power to the intersection of non-deterministic local decision and $\NP$.

\subsection{Unconditional Separation of $\LDP$ from $\LD \cap \PP$}\label{sec:uncond}
By definition we have $\LDP \subseteq \LD$,
as every $\LDP$-algorithm is also an $\LD$-algorithm.
It is also easy to see that $\LDP \subseteq \PP$:
if every node of the network computes its decision in $\poly(n)$ time,
then a poly-time centralized Turing machine can simulate the local algorithm
at every node, and accept iff all nodes accept.
Together we have that $\LDP \subseteq \LD \cap \PP$.
Our first result shows that the containment is strict.

\paragraph{High-level overview.}
To separate $\LDP$ from $\LD \cap \PP$,
we use a variation on the language $\ITER$, which was used in \cite{balliu2018can} to separate $\Pioloc$ from $\LD$.
We call our variation $\ITERIN$.

The idea is to construct a language of paths,
where the center node is given a Turing machine $M$, two inputs $a,b \in \set{0,1}^*$,
and a bound $s$;
the goal is to decide whether $M$ halts on both $a$ and $b$ within at most $s$
computation steps, and accepts either $a$ or $b$ (or both).
The bound $s$ may be much larger than the length of the input (it is encoded in binary),
so an efficient algorithm cannot afford to run $M$ for $s$ steps and check
whether it accepts $a$ or $b$,
but a local algorithm with unbounded computation time can do so,
and therefore $\ITERIN \in \LD$.
To make the task solvable for a polynomial-time centralized Turing machine,
we add additional annotations (in the form of inputs to the nodes):
on the left side of the path, from the center outwards, we write 
the sequence of configurations that $M$ goes through in its computation on $a$,
until it halts;
on the right side of the path we do the same for $b$.
This makes it possible for a poly-time Turing machine to simply examine
the computation sequence of $M$, make sure it is legal (i.e., it matches the transition function of $M$),
and verify that at either the left or the right side of the path (or both)
we have
an accepting configuration of $M$.
Thus, $\ITERIN \in \PP$.

Finally, we prove that an algorithm that is both local and efficient cannot
decide the language $\ITERIN$: intuitively, this is because 
it can neither afford to run $M$ for $s$ steps, nor can it ``see''
the endpoints of the path to verify that at least one of them has an accepting configuration.
The formal proof shows that if there existed a $\PLD$-algorithm for $\ITERIN$
then we could use it to decide in polynomial time a language
that is not in $\PP$.


\paragraph{Detailed construction.}
Let $M$ be a Turing machine,
and let $a,b \in \set{0,1}^*$ be strings such that $M$ halts on input $a$ and on input $b$.
We define a configuration $C^{n_L, n_R}({M,a,b, s}) = (G, x)$, as follows:
\begin{itemize}
	\item $G$ is a path of the form $u_{n_L},\ldots,u_1,v,w_1,\ldots,w_{n_R}$,
		consisting of a \emph{pivot node} $v \in V(G)$,
		a left sub-path $L = u_{n_L},\ldots,u_{1}$,
		and a right sub-path $R = u_1,\ldots,u_{n_R}$.
	\item The input of the pivot node $v$ is $x(v) = (0, \langle M \rangle, a, b, s )$,
		where $\langle M \rangle$ is the encoding of the Turing machine $M$.
	\item For each node $u_i \in L$ on the left sub-path, the input of $u_i$
		is given by $u_i = (i, \langle M \rangle, M_{a,i} )$,
		where $M_{a,i}$ is the configuration of $M$ after $i$ steps
		executing with input $a$
		(recall that the configuration of a Turing machine
		consists of the contents of the tape, the location of the tape head,
		and the current state).
		To avoid the clash in terminology, we refer to configurations of Turing machines
		as \emph{TM-configurations}.
	\item Similarly, for each node $w_i \in R$ on the right sub-path,
		we have $x(w_i) = (i, \langle M \rangle, M_{b,i} )$.
\end{itemize}

We simplify the notation somewhat by writing $C^n(M,a,b,s) = C^{n,n}(M,a,b,s)$,
and $C(M,a,b,s) = C^s(M,a,b,s)$.
Given a configuration $C^{n_L, n_R}({M,a,b,s}) = (G, x)$ as defined above,
we say that a node $u \in V(G)$ is \emph{$r$-central} if the distance of $u$
from the pivot is at most $r$.

The language $\ITERIN$ consists of all configuration $C^{n_L,n_R}(M,a,b,s)$ such that
	the TM-configurations written at the end of both sub-paths are both halting,
	$s \geq \max(n_L, n_R)$,
	and $M$ accepts $a$ or $b$ (or both).

	As we explained above, it is not difficult to see that $\ITERIN$ can be decided by a local algorithm,
and is also in $\PP$:
\begin{claim}\label{claimITINalgos}
	$\ITERIN \in \LD \cap \PP$.
\end{claim}
%\begin{proof}[Proof sketch]
%The key is that $\ITERIN$ contains a "trap door" for each type of algorithm:
%the sequential poly-time algorithm checks that $s \geq \max(n_L, n_R)$,
%but does not run $M$ for $s$ steps (as $s$ may not be polynomial in the input length),
%instead verifying that the TM-configurations evolve according to the transition function of $M$,
%that the TM-configurations at the ends of the sub-paths are halting, and that at least one accepts.
%On the other hand, the $\LD$-algorithm cannot directly check that at least one of the endpoints of the sub-paths
%contains an accepting TM-configuration,
%but the pivot,
%which is not polynomially-bounded,
%can afford to run $M$ for $s$ steps and verify that it accepts one of the inputs.
%\end{proof}

%\begin{proof}[Proof of Claim~\ref{claimITINalgos}]
	%To decide membership in $\ITERIN$ using a $1$-local algorithm:
	%\begin{itemize}
		%\item Each node verifies that its index is consecutive with its neighbors' indices,
			%and that the TM-configuration in its input is indeed the successor
			%to the TM-configuration in the input of its neighbor with a preceding index
			%(unless its index is 0).
		%\item The endpoints of the path verify that they have a halting TM-configuration,
			%and that $s$ is no smaller than their index.
		%\item The pivot runs the TM $M$ on $a$ and on $b$ for $s$ steps each,
			%and verifies that at least one of $a$ and $b$ is accepted.
			%Note that since $s \geq \max(n_L, n_R)$,
			%if both endpoints accept,
			%then $M$ halts in at most $s$ steps on $a$ and on $b$,
			%as both endpoints check that they have a halting TM-configuration,
			%and because the nodes along the way
			%verify the transitions of the TM,
			%the TM-configurations at the endpoints evolve from the initial configuration
			%in at most $n_L$ or $n_R$ steps, respectively.
	%\end{itemize}
	%To decide membership in $\ITERIN$ using a poly-time centralized algorithm:
	%we verify that the input is structured correctly (that the graph is a path,
	%the nodes' indices
	%are correct, the pivot is given correctly-formatted input,
	%and $s$ is at least the length of both sub-paths),
	%and in particular, 
	%that the configurations of the TM on each sub-path evolve according to the transition
	%function of $M$.
	%Examining the endpoints of the sub-paths,
	%we make sure that at least one of them has an accepting configuration of the TM.
%\end{proof}

Next we show that $\ITERIN$ is not decidable by a polynomial-time local algorithm:

\begin{claim}\label{claimITINLDP}
	$\ITERIN \not \in \LDP$.
\end{claim}
\begin{proof}
Suppose for the sake of contradiction that there is a $\LDP$-algorithm $\A$ that decides $\ITERIN$,
and let $t > 0 $ be its locality radius.
Let $\calL \in \DTIME(2^{n}) \setminus \PP$ be some language that is Turing-decidable
in time $O(2^{n})$ but not in polynomial time,
and such that $\eps \not \in \calL$ (here and in the sequel, $\eps$ denotes the empty word).
Such a language exists by the Time Hierarchy Theorem~\cite{hartmanis1965computational}.
We claim that using the $\LDP$-algorithm $\A$ that decides $\ITERIN$, we can construct a polynomial-time Turing machine
that decides $\calL$, a contradiction.

Let $M$ be a $\DTIME(2^{n})$-time Turing machine that decides $\calL$,
and let $f \in O(2^n)$ be a function bounding the running time of $M$ on inputs of length $n$.
Given input $z \in \set{0,1}^*$,
let $C_z = C(M,\eps,z,f(|z|))$
be the configuration that encodes the runs of $M$ on $\eps$ (on the left sub-path) and on $z$
(on the right sub-path)
until $M$ halts, using sub-paths of length $f(|z|)$.
Since we assume that $\eps \not \in \calL$,
we have $C_z \in \ITERIN$ iff $z \in \calL$.

We define a poly-time Turing machine $M'$ for $\calL$ as follows:
on input $z \in \set{0,1}^*$,
$M'$ constructs the configuration $C_z' \coloneq C^{2t}(M, \eps, z, f(|z|))$,
which is essentially the central portion of $C_z$, including only $2t$ 
nodes to the left and to the right of the pivot (a total of $4t+1$ nodes).
Next, $M'$ simulates the local algorithm $\A$
on all the nodes of $C_z'$.
Finally, $M'$ accepts iff $\A$ outputs 1 at all $t$-central nodes of $C_z'$
(ignoring the outputs of the other nodes).

It is not difficult to verify that the running time of $M'$ is polynomial in $|z|$,
in the description length of $M$ (which is constant),
and in $t = o(n)$.
To show that $M'$ indeed decides $\calL$,
suppose first that $z \in \calL$.
Then $C_z \in \ITERIN$ by construction,
and therefore $\A$ must output 1 at all nodes of $C_z$.
But this means that all $t$-central nodes in $C_z'$ must also accept:
for each $t$-central node $u$ in $C_z'$,
the $t$-local view of $u$ is the same in $C_z$ and in $C_z'$,
because $C_z'$ is obtained from $C_z$ by removing only nodes at distance greater than $t$ from $u$.
Since the output of $u$ depends only on its $t$-local view,
and we know that $u$ accepts in $C_z$, it must also accept in $C_z'$
Thus, $M'$ accepts $z$.

Now suppose that $z \not \in \calL$.
In this case, $C_z \not \in \ITERIN$,
because in $C_z$ the two inputs encoded in $x$ are both rejected by $M$ (as $\eps, z \not \in \calL$).
We claim that at least one $t$-central node of $C_z$ must reject;
as above, this means that the same node also rejects in $C_z'$,
causing $M'$ to reject $z$.

Suppose for the sake of contradiction that all $t$-central nodes of $C_z$ accept.
However, since $C_z \not \in \ITERIN$, we know that some node of $C_z$ rejects;
let $u$ be such a node.
The distance of $u$ from the pivot $v$ must be greater than $t$, since we assumed that no $t$-central node rejects.
Now fix some string $a \in \calL$ (which must exist, as $\emptyset \in \PP$ and we assumed $\calL \not \in \PP$),
and let $C_{a,z} = C(M,a,z,f(\max(|a|,|z|)))$
be the configuration encoding the runs of $M$ on $a$ (on the left sub-path)
and on $z$ (on the right sub-path), using paths of length $f(\max(|a|,|z|)$,
so that $M$ halts on both.
Since $a \in \calL$,
we have $C_{a,z} \in \ITERIN$,
and thus all nodes must accept $C_{a,z}$.
This includes node $u$.
However, since $u$ is at distance greater than $t$ from the pivot,
the $t$-local view of $u$ is the same in $C_{a,z}$ and in $C_z$;
thus, $u$ also accepts in $C_z$, a contradiction.
\end{proof}










%\subsection{If $\LDn \cap \PP \not \subseteq \LDnP$, Then $\PP \neq \NP$}


\subsection{Separation of $\LDnP$ From $\LDn \cap \PP$ Assuming Injective One-Way Functions}
\label{sec:owf}
In the previous section we showed that $\LD \cap \PP \not \subseteq \LDP$,
but our proof used the fact that the nodes do not know the size of the graph,
and therefore their output when the graph is a short path is the same as their output on a long path,
provided their local neighborhood stays the same.
We now ask whether the separation continues to hold if nodes do know the size of the network:
let $\LDn, \LDnP$ be variants of $\LD, \LDP$ (resp.),
where nodes receive the size $n$ of the graph as part of their input.
Is it still true that $\LDn \cap \PP \not \subseteq \LDnP$?

Perhaps surprisingly,
even though we are considering deterministic computation models,
the answer turns out to be related to whether or not $\PP = \NP$:
we prove that
$\LDn \cap \PP \not \subseteq \LDnP$ implies $\PP \neq \NP$,
and conversely, under the plausible assumption that injective one-way functions exist,%
\footnote{This is stronger than assuming that $\PP \neq \NP$,
because if $\PP = \NP$ then every function is easy to invert.}
we can still show that 
$\LDn \cap \PP \not \subseteq \LDnP$.

A \emph{one-way function family} is a family $\set{ f_n }_{n \in \nat}$,
where $f_n : \set{0,1}^n \rightarrow \set{0,1}^{m(n)}$ for some $m(n) \geq n$,
such that given an image $y \in \set{0,1}^{m(n)}$,
it is difficult to find a pre-image $x$ such that $f_n(x) = y$ (we refer to~\cite{OdedBook} for the formal definition,
as it is not needed here).
It is known that every one-way function has a \emph{hard-core predicate}~\cite{GL89},
a Boolean predicate that can be computed in poly-time from $x \in \set{0,1}^n$,
but is hard to compute given only $f_n(x)$:
\begin{definition}[Hard-core predicate]\label{def:HCP}
	A family of predicates %\\to handle the margins
 $ \set{ b_n : \set{0,1}^n \rightarrow \set{0,1}}_{n \in \nat}$
computable in poly-time
 is called a \emph{hard-core} of a
	family of functions $\set{ f_n : \set{0,1}^n \rightarrow \set{0,1}^{m(n)}}$ (where $m(n) \geq n$)
	if for every probabilistic, polynomial-time (PPT) algorithm $\Adv$, there is a negligible function $\eps(\cdot)$ such that
	for all sufficiently large $n$,
	\begin{equation*}
		\prb{
			\Adv( f(z) ) = b(z)
		}
		{
			z \leftarrow U_n
		}
		<
		\frac{1}{2} + \eps(n).
	\end{equation*}
\end{definition}
where $U_n$ denotes the uniform distribution on $\{0,1\}^n$.

\begin{proof}[Proof of Theorem~\ref{thm:local}, part (3)]
	Fix a family $\calF = \set{ f_n }_{n \in \nat}$ of injective one-way functions,
	where $f_n : \set{0,1}^n \rightarrow \set{0,1}^{m(n)}$ for $m(n) \geq n$,
	and let $\set{ b_n }_{n \in \nat}$ be a family of hard-core predicates for $\calF$.
	Consider a distributed language $\calL$, which includes all configurations $C_z = (G, x_z)$ for $z \in \set{0,1}^{\ast}$,
	where $G$ is a path $v_0,\ldots,v_{n-1}$ of length $n = |z|$,
	and the input assignment $x_z$ is given by
	$x_z(v_0) = (0, b_n(z))$,
	$x_z(v_{n-1}) = (n-1, z)$,
	and $x_z(v_i) = (i, f_n(z))$ for every $0 < i < n-1$.
	We claim that $\calL \in \LDn \cap \PP$, but $\calL \not \in \LDnP$.

	To decide membership in $\calL$ using a local algorithm with unbounded computation time,
	the first node on the path can simply invert $f_n$ to compute $z$ (recall that $f_n$
	is injective),
	and then use $z$ to compute $b_n(z)$ and compare it against
	its input. In addition, the other path nodes
	need to verify that their input is locally consistent with $\calL$
	(e.g., they are indexed properly).

	To decide membership in $\calL$ using a polynomial-time centralized algorithm,
	we can simply read $z$ off of the last node on the path, compute both $f_n(z)$ and $b_n(z)$,
	and verify that the input is consistent with $f_n(z)$ and $b_n(z)$.

	%We first show that $\calL \in \LDnP \cap \PP$.
	%To decide $\calL$ using a local algorithm, it suffices for each node to verify:
	%\begin{itemize}
		%\item If $x(v) = (0, y)$, then $v$ has degree 1,
			%its neighbor $u$ has $x(u) = (1, y')$,
			%and there exists some $z \in \set{0,1}^n$ such that
			%$y = (b_n(z), f_n(z))$ and $y' = (f_n(z), \bot)$
			%(we check by brute-force search over all possible $z$).
		%\item If $x(v) = (i, y)$ where $0 < i < n - 2$,
			%then the degree of $v$ is 2,
			%and its neighbors $u, u'$ have $x(u) = (i - 1, y)$ and $x(u') = (i+1, y)$
			%(or vice-versa).
		%\item If $x(v) = (n - 2, y)$, then $v$'s degree is 2, and one of its neighbors $u$
			%has $x(u) = (n-3, y')$
			%where $y' = (z, \bot), y = (f_n(z), \bot)$ for some $z \in \set{0,1}^n$.
		%\item If $x(v) = (n-1, y)$, then $v$'s degree is 1.
	%\end{itemize}
	%To decide $\calL$ in centralized poly-time,
	%we first verify that the configuration has the correct structure (it is a path and the input has the correct form).
	%If so, let $(z, \bot)$ be the payload of the node with index $n - 1$, 
	%and let $(a, w)$ be the payload of the node with index 0.
	%We use $z$ to compute $b_n(z)$,
	%and verify that $a = b_n(z), w = f_n(z)$, and that all the nodes in-between have payload $(f_n(z), \bot)$.

	Now suppose for the sake of contradiction that $\calL \in \LDnP$,
	and let $A$ be a $t$-local efficient algorithm for $\calL$, for some $t = o(n)$.
	Then for every sufficiently large $n$, 
	we can break the hard-core predicate $b_n$ using the following adversary $\calB$:
	given input $w = f_n(z)$ for some $z \in \set{0,1}^n$,
	the adversary constructs the first $2t$ nodes of the configuration $C' = (G, x')$,
	where $G$ is a path of length $n$,
	and $x'$ is identical to $x_z$,
	except that the inpu of the first node is $(0, 0)$ (since the adversary does not know $b_n(z)$).
	Note that the adversary does not need to know $z$ for this, because $z$ is only given to the last node
	on the path, and $t < n$; the adversary only needs to know $f_n(z)$, which it is given.
	The adversary simulates the first $2t$ nodes in $C'$,
	and
	if the first $t$ nodes in the first version accept, it outputs ``0'';
	otherwise it outputs ``1''.

	We claim that our adversary correctly computes $b_n(z)$ for all $z \in \set{0,1}^n$.
	Given $w \in \set{0,1}^{m(n)}$, there is a unique $z \in \set{0,1}^n$ such that $w = f_n(z)$, because $f$ is injective.
	If $b_n(z) = 0$,
	then the $t$-neighborhood
	of each of the first $t$ nodes in the configuration $C'$ constructed by the adversary is identical
	to their view in the ``true'' configuration $C_z = (G, x_z)$.
	Since $C_z \in \calL$, all nodes must accept, and in particular the first $t$ nodes do;
	therefore the first $t$ nodes also accept in $C'$.
	Now suppose that $b_n(z) = 1$.
	Then the configuration $C'$,
	of which the adversary constructed the first $t$ nodes,
	is not in $\calL$; some node must reject in $C'$.
	Furthermore, one of the first $t$ nodes must reject in $C'$:
	suppose they do not, and let $v_j$ be some node that rejects, with $j > t$.
	In the ``true'' configuration $C_z = (G, x_z)$, the $t$-neighborhood of node $v_j$
	is the same as in $(G_n, x_z')$, because the only difference between the two configurations is the input
	of the first node, which is at distance greater than $t$ from $v_j$.
	But this means that $v_j$ also rejects in $(G_n, x_z) \in \calL$, contradicting the correctness of the local algorithm.
	We conclude that at least one of the first $t$ nodes must reject, and therefore our adversary outputs ``1''.

	We have now shown that $\calB(f_n(z)) = b_n(z)$ for all sufficiently large $n$ and $z \in \set{0,1}^n$.
	This implies that $b_n$ is not a hard-core predicate, as it contradicts Definition~\ref{def:HCP}.
\end{proof}
