\section{Certifying Executions of Computationally-Efficient Distributed Algorithms}
\label{sec:distprover}
In this section we construct a succinct distributed argument
for certifying the execution of polynomial-time distributed algorithm,
where the prover is itself distributed;
essentially, the distributed algorithm certifies its own execution,
using an additional $O(\diam(G))$ rounds.


Fix a distributed algorithm, represented by a deterministic Turing machine $D$ that
executes at every node.
%takes as input a node's input and neighborhood,
%writes outgoing messages on an output tape, reads incoming messages from an input tape, and eventually
%produces some output.
The distributed language we would like to certify is the language
$\calL_D$
consisting of all configurations $(G, x, y)$,
where $G$ is the network graph, $x : V(G) \rightarrow \calX$ is an input assignment to the nodes,
and $y : V(G) \rightarrow \calY$ is the output stored at the nodes,
such that when $D$ is executed at every node of $G$ with input assignment $x$,
each node $v \in V(G)$ produces the output $y(v)$.
We construct a distributed prover for the statement ``$(G,x,y) \in \calL_D$''.

To simplify the presentation, we assume here that there is no input $x$,
and that $D$ is a \emph{decision} algorithm,
so that the output we want to certify is $y(v) = 1$ at all nodes.
(See Appendix~\ref{app:distprover} for the general case.)

\paragraph{Overview of the construction.}
Certifying the execution of the distributed algorithm $D$
essentially amounts to certifying a collection of ``local'' statements,
each asserting that at a specific node $v \in V(G)$,
the algorithm $D$ indeed produces the output $y(v) = 1$.
The prover at node $v$ can record the local computation at node $v$
as $D$ executes,
and use a SNARG or a SNARK
to certify that it is correct: for example, it can certify that incoming messages
are handled correctly (according to $D$),
outgoing messages are produced correctly, and eventually, the output of $v$ is indeed $y(v) = 1$.
The main obstacle is verifying consistency across the nodes:
how can we be sure that incoming messages recorded at node $v$ were indeed sent by $v$'s neighbors,
and that $v$'s outgoing messages are indeed received by $v$'s neighbors?

A na\"ive solution would be for node $v$ to record, for each neighbor $u \in N(v)$,
a hash $H_{(v,u)}$ of all the messages that $v$ sends and receives on the edge $\set{v,u}$;
on the other side of the edge, node $u$ would do the same, producing a hash $H_{(u,v)}$.
At verification time, nodes $u$ and $v$ could compare their hashes, and reject if $H_{(v,u)} \neq H_{(u,v)}$.
Unfortunately, this solution would require too much space,
as node $v$ can have up to $n - 1$ neighbors;
we cannot afford to store a separate hash for each edge.

Our solution is instead to compute a hash $h(m)$
for every message $m$ sent or received by node $v$,
but store only a \emph{sum} of the hashes:
we separate outgoing messages from incoming messages,
and store two sums,
$s_{\var{out}}(v) = \sum_{\text{outgoing $m$}} h(m)$
and
$s_{\var{in}}(v) = \sum_{\text{incoming $m$}} h(m)$.
To certify consistency across all the nodes,
we compute a spanning tree of the network,
and store at every tree node $u$ the partial sums
in its subtree,
\begin{equation*}
	S_{\var{out}}(u) = \sum_{v \in \text{$u$'s subtree}} s_{\var{out}}(v),
	\qquad
	S_{\var{in}}(u) = \sum_{v \in \text{$u$'s subtree}} s_{\var{in}}(v).
\end{equation*}
In particular, at the root $r$ of the tree, we store the full sums:
%\begin{equation*}
	$S_{\var{out}}(r) = \sum_{v \in V(G)} s_{\var{out}}(v)$
	and
	%\qquad
	$S_{\var{in}}(r) = \sum_{v \in V(G)} s_{\var{in}}(v)$.
%\end{equation*}
The root then compares the two sums, and verifies that they are equal,
which means that every message sent was indeed received, and vice-versa.


Since we compare \emph{sums} of hashed values rather than
directly comparing hashed values to one another,
our construction requires the following property,
which we call \emph{sum-collision-resistance};
it is a plausible strengthening of standard collision-resistance (see discussion in Appendix~\ref{app:crypto:hashes}).
%\TODO{add + subsection of appendix}

\begin{definition} [Sum-Collision-Resistant Hash (SCRH)]
    A hash family $(\Gen, \Hash)$ is considered \emph{sum-collision-resistant} if for any probabilistic poly-time adversary $\mathcal{A}$, there exists a negligible function $\epsilon(\cdot)$, such that for every $\secpar\in \mathbb{N}$,
    \begin{gather*}
        \prob{}
        {
        \begin{array}{cc}
        M \neq M' \\
        \sum_{ m \in M }\Hash(\hk, m) = \sum_{ m' \in M'}\Hash(\hk, m')
        \end{array}
        \middle\vert
        \begin{array}{ll}
             hk \leftarrow Gen(1^n, 1^\secpar) \\
             (M, M') \leftarrow\calA(\hk, 1^\secpar, n)
        \end{array}
        } \leq \epsilon(\secpar)
    \end{gather*}
\end{definition}

%\begin{theorem}[Informal]
    %Let $\calD$ be a distributed algorithm that runs in $T = \poly(n)$ rounds
    %and sends messages of length $\poly(n)$.
    %Assuming SNARKs for $\NP$ exist,
    %and assuming a SCRH exist,
    %there is a distributed argument of length $\polylog(n)$
    %certifying $D$'s execution, where the prover is a distributed algorithm
    %running in $O(T + \diam(G))$ rounds and sending messages of $\polylog(n)$ bits.
%\end{theorem}

\paragraph{Detailed description of the construction.}
In the sequel, fix an SCRH, $(\SCRHGen, \SCRHHash)$.

We represent a message by $\var{msg} = (r, \set{u, v}, m)$,
where $r \in \nat$ is the round number in which the message was sent,
$\set{u,v} \in E$ is the edge on which the message was sent,
and $m \in \set{0,1}^*$ is the contents of the message.
It is important that this representation of a message does not indicate whether the message was sent by $u$ and received by $v$ or vice-versa,
as our construction relies on hashing messages and verifying that every (hashed) incoming message has a corresponding (hashed) outgoing message.

The consistency of the local computation at a specific node is captured
by a
language $\calD$,
which consists of all triplets $(\hk, I(v), W(v))$ such that:
\begin{itemize}
	\item $\hk$ is a hash key obtained by calling $\SCRHGen$,
	\item $I(v) = (v, N(v), s_{\var{in}}(v), s_{\var{out}}(v))$,
		where
		$v \in \calU$ is the UID of a node,
		$N(v) \in \calU^{\ast}$ is the neighborhood of the node,
		and $s_{\var{in}}(v), s_{\var{out}}(v)$ are hash sums;
	\item $W(v) = (\var{msgout}(v), \var{msgin}(v))$ consists of two sets of messages;
	\item $(\hk, I(v), W(v)) \in \calD$ iff when the distributed algorithm $D$ is executed 
at a node with UID $v$ and neighbors $N(v)$,
and the incoming messages at node $v$ are $\var{msgin}(v)$,
then node $v$ sends the messages $\var{msgout}(v)$
and accepts (i.e., outputs 1),
and furthermore,
\begin{equation}
	s_{\var{in}} = \sum_{\var{msg} \in \var{msgin}} \SCRHHash(\hk, \var{msg}),
	\qquad
	s_{\var{out}} = \sum_{\var{msg} \in \var{msgout}} \SCRHHash(\hk, \var{msg}).
\end{equation}
\end{itemize}
We think of $W(v) = (\var{msgout}(v), \var{msgin}(v))$ as a \emph{witness}
to the correct computation at node $v$.
%We refer to $s_{\var{in}}$ as the \emph{hash-sum of incoming messages}, and to $s_{\var{out}}$ as the \emph{hash-sum of outgoing messages}
%at $v$.

Since the algorithm $D$ is itself a polynomial-time Turing machine,
and the SCRH can be computed in polynomial time,
there is a polynomial-time Turing machine $M$ that decides the language $\calD$.
Fix a SNARK $(\SNARKGen, \SNARKPro, \SNARKVer, \SNARKExt)$
for the language
of pairs $(\hk, I)$ satisfying
$\exists W = (\var{msgout}, \var{msgin}) : \text{$M$ accepts $(\hk, I, W)$}$.

The distributed prover at each node $v$ computes the following certificate $\pi(v)$:
\begin{itemize}
	\item The hash-sums $s_{\var{out}}(v), s_{\var{in}}(v)$.
	\item A SNARK proof $\pisnark(v)$, certifying that there exists a witness $W(v) = (\var{msgout}(v), \var{msgin}(v))$
		such that $(\hk, I(v), W(v)) \in \calD$.
\end{itemize}
In addition, the distributed prover computes a spanning tree of the network in $O(\diam(G))$
rounds, and stores at each node $v$ the parent $p(v)$ of $v$ (or $\bot$, if $v$ is the root),
and a spanning-tree certificate~\cite{korman2005proof} consisting
of the UID of the root and the distance of $v$ from the root.
Finally, by convergecast up the tree,
the distributed prover computes and stores at $v$
the partial sums 
\begin{equation*}
	S_{\var{out}}(v) = s_{\var{out}}(v) + \sum_{u \in \mathrm{children}(v)} S_{\var{out}}(u),
	\qquad
	S_{\var{in}}(v) \leftarrow s_{\var{in}}(v) + \sum_{u \in \mathrm{children}(v)} S_{\var{in}}(u).
\end{equation*}


