\section{Introduction}
\label{sec:intro}

%\begin{subfigures}\label{fig:global}

%\begin{nicefig}[h]{An Example Figure }{fig:example1}
%This is a nice figure.
%\begin{algorithmic}[1]
    %\State Do something \Comment{Doing something}
    %\If {$x > 5$}
        %\State  $x$ is big, Yay!
    %\Else
        %\State $x$ still has room to improve.
    %\EndIf
%\end{algorithmic}
%
%\end{nicefig}
%\begin{nicefig}[h]{Another Example Figure }{fig:example2}
%This is another nice figure
%\end{nicefig}
%
%\end{subfigures}
%

%This is a reference to \cref{fig:example1}, and one to \cref{fig:example2} and one to \cref{fig:example1,fig:example2} together. The top-level fig is \cref{fig:global}.

In distributed graph algorithms, the typical complexity measures that one tries to minimize are related to communication and synchronization: we aim to construct algorithms that use few synchronized communication rounds, and send a small number of messages, each as short as possible. There is a rich body of literature on the power of such algorithms: from the $\mathsf{LOCAL}$ model,
where communication rounds are the key resource, to $\mathsf{CONGEST}$, where bandwidth is the main bottleneck,
and many other combinations and variations.
Distributed algorithms are typically allowed to have unbounded \emph{local} computational power,
with each network node able to compute any Turing-computable function for free (e.g., ~\cite{fraigniaud2013can,fraigniaud2013towards}), or sometimes even any function at all (e.g.,~\cite{NS95}).
This puts the theory of local decision on completely different footing from classical centralized notions of efficiency, such as $\PP$ and $\NP$, and makes them impossible to compare.

In this paper, we study \emph{computationally-bounded} distributed local decision, and ask what can be achieved by computationally-efficient local algorithms and provers.
We show that computational restrictions can be {helpful} for prover-assisted distributed certification, but on the other hand,
when there is no prover,
computational restrictions
do limit the power of local algorithms beyond what one might expect.

\paragraph{Computationally-sound distributed proofs.}
In distributed certification (also known as \emph{proof labeling schemes}~\cite{korman2005proof} or \emph{locally checkable proofs}~\cite{LCP}), our goal is to certify some property of the network: for example, one might wish to certify that the output of a distributed algorithm is correct, or that the network graph has some desirable property.
To facilitate this goal, we enlist the help of a \emph{prover}, which provides each node with a short certificate;
the nodes exchange their certificates with one another (or more generally, carry out some efficient verification procedure)
and decide whether to accept or reject.
The proof is considered to be \emph{accepted} if all nodes accept.
While many useful properties can be certified using short certificates, some problems are known to require very long certificates and a lot of communication between the nodes, up to $\Omega(n^2)$ bits~\cite{LCP}.
%Moreover, there is no general way to obtain short certificates proving the correctness of an efficient distributed algorithm,

The area of distributed certification has so far stood apart from the rich theory of  centralized decision problems (e.g., $\PP$ vs.\ $\NP$) and delegated computation. In the centralized setting, under cryptographic assumptions, a computationally-bounded prover can present a weak verifier with a short proof that convinces the verifier of a statement of the form ``$x \in \calL$'', where $\calL$ is any language in $\PP$~\cite{choudhuri2021snargs, cryptoeprint:2022/1320}, or, under stronger assumptions, in $\NP$~\cite{micali2000computationally, bitansky2013recursive, groth2016size}. This is called a \emph{computationally-sound proof}, or a \emph{succinct argument} for $\calL$. 
The key is that rather than requiring perfect soundness against any prover, we require only that the verifier not be fooled by any computationally-bounded prover, except with negligible probability. In return, we get much shorter proofs. If the argument is non-interactive, it is called a \emph{succinct non-interactive argument} (SNARG). If it also has the property that whenever the prover convinces the verifier, one can extract an $\NP$-witness from the prover, then the argument is a \emph{succinct non-interactive argument of knowledge} (SNARK).

We ask whether the same can be done in the distributed setting,	and show that the answer is \emph{yes}, under standard cryptographic assumptions in the case of language in $\PP$, or under somewhat strong assumptions for languages in $\NP$. We define \emph{succinct distributed arguments}, which are computationally-sound (non-interactive) proofs in the distributed network setting, and show:
 \begin{theorem}\label{thm:centralized}
     Let $\Lan$ be a language on graphs.
     \begin{enumerate}
         \item If $\Lan\in\PP$, assuming SNARGs for $\PP$ and collision-resistant hash function exist,%
		\footnote{
            We introduce these primitives in Section~\ref{sec:related} and~\ref{sec:prelim}, and discuss concrete hardness assumptions under which they are known to exist  in Appendix~\ref{app:crypto} and~\ref{app:concrete}.
         	}
		 there is a succinct distributed argument for $\Lan$ with certificates of length $\polylog(n)$.
         \item If $\Lan\in \NP$, assuming SNARKs for $\NP$ and %Vector Commitments exist,
         collision-resistant hash functions exist,
	 there is a distributed argument for $\Lan$ using certificates of length $\polylog(n)$.    
     \end{enumerate}
 \end{theorem}

 \paragraph{Certifying executions of efficient distributed algorithms.}
	One of the main motivations for studying distributed certification is fault-tolerance and self-stabilization:
	to cope with a dynamic and fault-prone environment, it is useful to be able to identify when 
	the network is  in an illegal state, so that we can undertake actions to correct the problem.
	Proof labeling schemes were originally defined at least in part with this motivation in mind~\cite{korman2005proof}.
	One property that is very interesting to certify is whether output that was
	previously produced by a distributed algorithm $\calD$
	is still up-to-date: if executed in the current network, would $\calD$ still produce the same output?
	It was shown in~\cite{korman2005proof} that if $\calD$ runs in $r$ rounds, and every node sends at most $m$ messages
	of $b$ bits per round,
	then the execution of $\calD$ can be certified using certificates of length $O(r m b)$ bits per node%
	\footnote{In~\cite{korman2005proof} the scheme given is for certifying any property $\calP$ that can be \emph{verified}
	by a distributed algorithm that accepts or rejects at each node. The property ``the algorithm produces the given output'' can certainly be verified by running the algorithm itself and examining its output.}
	by storing the entire history of messages sent at each node.
	Unfortunately, these certificates can be very long when the algorithm uses many rounds or messages.

	We show that using a computationally-sound proof,
	a distributed algorithm that runs in polynomial number of rounds, message size
	and local computation time can \emph{certify its own execution}, using certificates of polylogarithmic length
	at each node, and incurring an additive overhead to the running time that 
	is linear in the diameter of the graph.
	

\begin{theorem}
    Let $\calD$ be a distributed algorithm that runs in $T = \poly(n)$ rounds
    and sends messages of length $\poly(n)$.
    Assuming SNARKs for $\NP$ %exist,
    and %assuming
    a certain type of collision-resistant hash functions exist,%
    \footnote{We require a new property called \emph{sum collision resistance}, see Section~\ref{sec:distprover}.}
    there is a distributed argument of length $\polylog(n)$
    certifying $\calD$'s execution, where the prover is an efficient distributed algorithm
    running in $O(T + \diam(G))$ rounds and sending messages of $\polylog(n)$ bits.
    \label{thm:distprover_informal}
\end{theorem}

\paragraph{Computationally-bounded local decision.}
The power of local decision algorithms has been extensively studied, under many variations (e.g.,~\cite{NS95,fraigniaud2013can,fraigniaud2013towards}, and many others),
perhaps the most famous of which is the $\LD$ model.
In all cases (to our knowledge), the algorithm is allowed unbounded local computational power, and as a result,
deterministic local decision is incomparable with the usual notion of computational efficiency, the class $\PP$
of polynomial-time algorithms.
To bridge this gap, we define the class $\PLD(t)$ of local distributed algorithms that run in 
$t(n)$ synchronous rounds, and require local computation time $\poly(n)$ at every node.
(The size of the network is not necessarily known to the nodes;
we consider both options.)

What is the power of algorithms in $\PLD(t)$?
Clearly, such algorithms cannot decide languages that are not in $\PP$, nor can they decide languages 
that are not in $\LD(t)$ (i.e., decidable in $t(n)$ rounds with no computational restrictions).
But can they decide every language that is both in $\LD(t)$ and in $\PP$?
It turns out that the answer is ``probably not'', 
but whether or not we can prove it unconditionally depends on whether the nodes know the size of the network,
and thus know \emph{how long} they are allowed to run.
Let $\LDn, \LDnP$ be variants of $\LD$ and $\PLD$ (resp.) where nodes know the size of the network.
Then we can show:

\begin{theorem}\label{theo:local}
	%For every locality radius $t = o(n)$,
	We have:
	\begin{enumerate}
		\item $\PLD(o(n)) \subsetneq \LD(o(n)) \cap \PP$;%
			\footnote{
A preliminary version of part (1) of Theorem~\ref{thm:local} 
appeared in the brief announcement~\cite{BA}.}
		\item If $\LDnP(o(n)) \neq \LDn(o(n)) \cap \PP$, then $\PP \neq \NP$; and
		\item Assuming injective one-way functions exist,%
			\footnote{A \emph{one-way function}
			is a function that is easy to compute, but hard to invert.
			See Section~\ref{sec:local} for the details.}
			$\LDnP(o(n)) \subsetneq \LDn(o(n)) \cap \PP$.
	\end{enumerate}
	\label{thm:local}
\end{theorem}

When we introduce nondeterminism,
the distinction disappears:
\begin{theorem}\label{theo:NLD}
	Let $\NLD(t)$, $\NPLD(t)$ be the classes of languages decidable by 
	nondeterministic $t(n)$-round algorithms with unbounded or, resp., polynomially-bounded
	local computation time.
	Then $\NPLD(t) = \NLD(t) \cap \NP$.
	\label{thm:nlocal}
\end{theorem}



\paragraph{Organization.}
The remainder of the paper is organized as follows.
In Sections~\ref{sec:related} and~\ref{sec:prelim}
we review the relevant background, discuss some of the cryptographic primitives we use,
and give formal definitions for some of them;
some definitions are deferred to Appendix~\ref{app:crypto}. % \TODO{make this this is the right one}.
In Section~\ref{sec:dargs} we define succinct distributed arguments, and 
show how to construct them for $\NP$-languages and for languages in $\PP$
(Theorem~\ref{thm:centralized}).
In Section~\ref{sec:distprover} we construct the distributed prover of Theorem~\ref{thm:distprover_informal}.
Finally, in Section~\ref{sec:local},
we discuss the power of computationally-efficient distributed algorithms,
and prove Theorem~\ref{thm:local}.
Many of the proofs, as well as pseudocode for the constructions in Sections~\ref{sec:dargs}
and~\ref{sec:distprover}, are deferred to the appendix.


