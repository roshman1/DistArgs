\section{Background and Related Work}
\label{sec:related}

\paragraph{Distributed certification.}
Although its roots trace back to work in self-stabilization,
the field of distributed certification was formally initiated in~\cite{korman2005proof},
which introduced \emph{proof labeling schemes}, and showed several constructions and impossibility results,
among them the scheme for certifying spanning trees which is used in the current paper (and is
a central building block for many certification schemes).
Many variants of the basic model have been studied, featuring different communication constraints
for the verifiers (e.g.,\cite{ostrovsky2017space,patt2017proof,FFHPP21}),
allowing randomization or interaction with the prover (e.g.,~\cite{baruch2015randomized,KOS18,NPY20}),
and studying other settings;
we refer to the excellent survey~\cite{CertSurvey} for a comprehensive overview.
To our knowledge, in all prior work, the prover and the verifier have unbounded local computational power.
%
In~\cite{EGK22}, the authors consider {locally-restricted proof labeling schemes},
where the prover itself is a (computationally-unbounded) local algorithm;
however, the proof is required to be sound against any prover, not just a local one.

\paragraph{Local distributed decision.}
Local algorithms have received an enormous amount of attention from the community,
and local decision in particular.
Over the past decade there has been a significant effort towards building
a complexity theory for the area: for example, in~\cite{fraigniaud2013towards},
the authors study the classes $\mathsf{LD}, \mathsf{BPLD}$ and $\mathsf{NLD}$ of languages decidable
by deterministic, randomized, or nondeterministic local algorithms,%
\footnote{In~\cite{fraigniaud2013towards} and follow-up, the output
of the algorithm may depend on the nodes' identifiers.
%and there is particular emphasis on the role of identifiers
%and their effect on expressive power.
%In this work
Here we do not restrict the way that identifiers may be used;
for this reason we use the notation $\LD, \NLD$
instead of $\mathsf{LD}, \mathsf{NLD}$.}
% to denote the languages decidable by local deterministic and nondeterministic algorithms, respectively.}
relate them to one another, and prove (among other results) that combining randomization
and nondeterminism allows a constant-round local algorithm to decide any language.
We refer to the survey~\cite{LDSurvey} for an overview of the area of local decision.
Again, to our knowledge, in prior work the local computation power of the nodes is
always unbounded.
