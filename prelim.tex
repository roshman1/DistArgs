\section{Preliminaries}
\label{sec:prelim}


\paragraph{Local distributed algorithms.}
We study the \emph{local decision} model of~\cite{fraigniaud2013towards},%
\footnote{Except that unlike~\cite{fraigniaud2013towards,balliu2018can},
we do not restrict the use of UIDs, as explained above.}
where we have an unknown communication network $G$ and an input assignment $x : V(G) \rightarrow \set{0,1}^*$.
The pair $(G, x)$ is called a \emph{configuration},
and we use $n$ to denote the size of the graph ($n = |V(G)|$).
A \emph{distributed language} $\calL$ is a set of configurations.
The \emph{locality radius} of a distributed algorithm is $t : \nat \rightarrow \nat$
if all nodes halt within $t(n)$ rounds in networks of size $n$.
We let $N_G(v)$ denote the neighborhood of node $v$ in $G$, omitting the subscript $G$
when the graph is clear from the context.


We assume that the nodes have unique identifiers, drawn from some large domain $\calU$,
and we typically assume that a UID from $\calU$ can be represented using $O(\log n)$ bits.%
\footnote{This assumption is not essential, as UIDs from a larger domain can be hashed down to $\set{1,\ldots,n}$
in our constructions.}%
\footnote{In Section~\ref{sec:local}, when we consider networks of unknown size, we do not make
any assumptions on the encoding of the UIDs, and in fact our results continue to hold even
in anonymous networks.}
We often conflate nodes with their UIDs.
We assume that we have some linear ordering $\calU$ of the UID space, that is, for any two UIDs $u \neq v$ from $\calU$,
either $u < v$ or $v < u$.

%When we need to encode a graph $G$, we represent it as an adjacency list, $L(G) = (N(v_1),\ldots,N(v_n))$,
%where $N(v_i)$ is the neighborhood of node $v_i$.
%The nodes appear in $L(G)$ in order of their UIDs, $v_1 < \ldots < v_n$.
%
The local computation of each network node $v \in V(G)$ is represented by a Turing machine which takes as input the UID $v$,
the neighborhood $N(v)$
of $v$ in $G$ and its input $x(v)$, and in each round, reads the messages received by $v$ from a dedicated input tape,
and writes the messages sent by $v$ on a dedicated output tape.
Eventually, the machine enters a halting state, which is either accepting or rejecting.
A configuration $(G, x)$ is \emph{accepted} iff all nodes accept, and otherwise the configuration is \emph{rejected};
a distributed algorithm $D$ \emph{decides} the distributed language $\calL(D)$
of configurations $(G,x)$ that are accepted when $D$ is executed in $(G,x)$.

\paragraph{On security parameters, succinctness and efficient provers.}
Throughout \Cref{sec:dargs,sec:distprover}, we use cryptographic primitives
that are sound against adversaries whose running time is bounded, typically polynomially, as a function
of a security parameter, $\lambda \in \nat$.
The \emph{succinctness} of these primitives---that is, the encoding length of whatever object they produce (e.g., the length of a hash value, or a proof)---is $\poly(\lambda, \log n)$.
To get proofs of length $\polylog(n)$, the security parameter we use is $\lambda = \log^c(n)$
for some $c > 1$;
we are interested in adversaries whose running time is polynomial in $n$,
which means they are sub-exponential in $\lambda$. % (e.g., if we set $\lambda = \log^2 n$, then $n^d = 2^{d \sqrt{\lambda}}$).
To allow for such provers, our hardness assumptions require security not against a polynomial-time adversary
but against a sub-exponential one.
It is relatively common to assume sub-exponential hardness;
for example, the learning-with-errors (LWE) problem is believed to be sub-exponentially hard.
%\TODO{citations}.

In the sequel, whenever we say ``efficient adversary/prover'',
we mean sub-exponential in $\lambda$ and polynomial in $n$.

\paragraph{Common reference string (CRS) model}
The cryptographic primitives that we use are proved sound in the \emph{common reference string} (CRS) model,
where we assume that a trusted setup phase has occurred, during which all parties get access to
a \emph{common reference string} drawn from a known distribution.
For example, the CRS can be used to select a hash function.
%which is essentially shared randomness between the prover and the verifiers (which the prover is not able to influence): for example, the prover and the verifiers may need to agree on a hash function, which is chosen uniformly at random from a large family of functions. CRS is a common assumption, underlying most of the work on delegated computation.
%We believe that CRS is an appropriate assumption for our setting, %where the prover is not truly ``malicious'' but rather ``misguided'' or ``misinformed'', reflecting bugs in the code or changes in the network.
In the distributed prover of Section~\ref{sec:distprover},
the CRS can be generated by having every node $v$ propose a random string $r_v$,
and summing the strings up a spanning tree to produce $\crs = \oplus_{v \in V} r_v$,
which is then disseminated to all the nodes. As long as a single node generates its random string honestly, the resulting $\crs$
will be uniformly random.

\paragraph{Hash functions.}
A hash family is accessed using two procedures, $\Gen$ and $\Hash$:
$\Gen(1^\secpar, \ell)\rightarrow \hk$ is a setup procedure that takes the security parameter $\lambda$
(in unary) and the length $\ell$ of the values to be hashed, and returns a \emph{hash key} $\hk$;
$\Hash(\hk, x)$ takes a hash key and a value $x$, and returns a hashed value.

\paragraph{Vector commitment schemes}
A vector commitment consists of the following algorithms.

\vspace{-1ex}
\paragraph{$\Gen(1^\secpar, q)\to \crs$:} a randomized algorithm that takes as input the security parameter $\secpar$ and the length $q$ of the committed vector, and outputs a common reference string $\crs$.
%\rc{this is slightly different than in \cite{catalano2013vector} who consider public parameters (that define $\mathcal{M}$) -- is there a reason to change?}
\vspace{-1ex}
\paragraph{$\Com(\crs, m_1,\ldots , m_q)\to (c,\aux)$:} a deterministic algorithm that takes $q$ messages $m_1,\ldots, m_q\in \mathcal{M}$ and a common reference string $\crs$, and outputs a commitment string $c$ together with auxiliary information $\aux$
(which may just be the values $m_1,\ldots,m_1$ themselves, but can also be more general).
\vspace{-4ex}
\paragraph{$\Open(\crs, m, i, \aux)\to \Lambda_i$:} a deterministic algorithm that takes the \crs, a message $m$, an index $i$, and auxiliary information \aux,
and produces a proof $\Lambda_i$ that $m$ is the \ith committed message.
\vspace{-1ex}
\paragraph{$\Verify(\crs, C, m, i, \Lambda)\to b$:} the verification algorithm takes the \crs, a message $m$, an index $i$, and a proof $\Lambda$, and outputs an acceptance bit.
%

\begin{definition} [Vector Commitments]\label{def:VC}
A VC $(\Gen, \Com, \Open, \Verify)$ is required to satisfy:
\vspace{-1ex}
\paragraph{Completeness}

For every messages sequence, $m_1,\ldots m_q$,
\begin{gather*}
    \prob{}
    {
    \begin{array}{ll}
    \forall i\in [q]: 
         \Verify(\crs, C, m_i, i, \Lambda_i) = 1
    \end{array}
    \middle\vert
    \begin{array}{ll}
         \crs \leftarrow \Gen(1^\secpar, q)\\
         C, aux\leftarrow\Com(\crs, m_1,\ldots,m_q) \\
         \forall i\in [q]: \Lambda_i \leftarrow \Open(\crs, m_i, i, \aux))
    \end{array}
    } = 1
\end{gather*}
\vspace{-3ex}
\paragraph{Position-Binding} For every $i\in [q]$, for any efficient adversary $\Adv$, there exists a negligible function $\epsilon(\cdot)$ such that for every $\secpar$, % the following probability (which is taken over all honestly generated parameters) is at most negligible in $k$:
\begin{gather*}
    \prob{}
    {
    \begin{array}{ll}
    \Verify(\crs, C, m, i, \Lambda_i) = 1 \ \wedge \\
    \Verify(\crs, C, m', i, \Lambda_i') = 1
    \end{array}
    \middle\vert
    \begin{array}{cc}
         &  \crs \leftarrow \Gen(1^\secpar, q)\\
         & (C, m, m', i, \Lambda, \Lambda_i')\leftarrow\Adv(\crs)
    \end{array}
    } \leq \epsilon(\secpar)
\end{gather*}

\vspace{-1ex}
\paragraph{Succinctness}
%\footnote{
%In \cite{catalano2013vector}, the succinctness property is even stronger and promises that the length of $c$ and $\Lambda_i$ are independent of $q$. We do not require it.
%}
The length of the commitment $c$ output by $\Com$ and the length of the opening $\Lambda_i$ output by $\Open$ are both bounded by $\poly(\lambda, \log q)$.
\end{definition}


%Our constructions in Sections~\ref{sec:dargsForNP} and Appendix~\ref{app:dargsForP} use a \emph{vector commitment scheme} (VC), which allows a prover to produce a short commitment $c$ to a vector $(m_1,\ldots,m_q)$, such that for every $1 \leq i \leq q$, the prover can later provide a short \emph{local opening} $\Lambda_i$ that would convince the verifier that the $i$-th value it committed to is $m_i$. A vector commitment is required to be \emph{complete} and \emph{position-binding}. Completeness asserts that verifying an honestly-generated commitment and opening always succeeds. The position-binding property requires that an efficient adversary cannot open the same location of a vector commitment to two different values, such that the verifier is convinced by both, except for a negligible probability \footnote{A function $\eps(n)$ is \emph{negligible} if it is asymptotically smaller than $1/n^c$ for every constant $c$.}. We require that the length of the commitment $c$, and the length of the opening $\Lambda_i$, are both of length $\poly(\lambda, \log q)$.
%which produces a short commitment for a vector $(m_1,\ldots,m_q)$, such that for any $1 \leq i \leq q$, the prover can convince the verifier that the  $i$-th value it committed to is $m_i$, but it cannot convince the verifier of any other value $m_i' \neq m_i$.
%The VC is accessed using the following functions:
%\begin{itemize}
%	\item $\Gen(1^\secpar, q)\to \crs$: takes the security parameter $\lambda$
%		and a vector length $q$, and returns a $\crs$.
%	\item $\Com(\crs, m_1,\ldots , m_q)\to (c,\aux)$: takes the $\crs$ and $q$ values $m_q,\ldots,m_q$,
%		and returns a commitment $c$ and auxiliary information $\aux$.
%	\item $\Open(\crs, m, i, \aux)\to \Lambda_i$: takes the $\crs$, a value $m$, a coordinate $i$
%		and auxiliary information $\aux$,
%		and produces a proof $\Lambda_i$ that $m$ is the $i$-th committed value.
%		We refer to $\Lambda_i$ as the \emph{opening} for coordinate $i$.
%	\item $\Verify(\crs, c, m, i, \Lambda)\to b$: takes the $\crs$,
%		the commitment $c$, a value $m$, a coordinate $i$ and an opening 
%		obtained from $\Open(\crs, m, i, \aux)$,
%		and returns 0 or 1.
%\end{itemize}
%A vector commitment is required to be \emph{complete} and \emph{position-binding}.
%Completeness asserts that verifying an honestly-generated commitment always succeeds:
%\begin{gather*}
    %\prob{}
    %{
    %\begin{array}{ll}
    %\forall i\in [q]: 
         %\Verify(\crs, c, m_i, i, \Lambda_i) = 1
    %\end{array}
    %\middle\vert
    %\begin{array}{ll}
         %\crs \leftarrow \Gen(1^\secpar, q)\\
         %c, aux\leftarrow\Com(\crs, m_1,\ldots,m_q) \\
         %\forall i\in [q]: \Lambda_i \leftarrow \Open(\crs, m_i, i, \aux))
    %\end{array}
    %} = 1
    %.
%\end{gather*}
%The position-binding property requires that an efficient adversary cannot lie about the values it committed to: for every $i\in [q]$, for any efficient adversary $\Adv$, there exists a negligible function%
%\footnote{A function $\eps(n)$ is \emph{negligible} if it is asymptotically smaller than $1/n^c$
%for every constant $c$.}
%$\epsilon(\cdot)$ such that for every $\secpar$,
%\begin{gather*}
%    \prob{}
%    {
%    \begin{array}{ll}
%    \Verify(\crs, C, m, i, \Lambda_i) = 1 \ \wedge \\
%    \Verify(\crs, C, m', i, \Lambda_i') = 1
%    \end{array}
%    \middle\vert
%    \begin{array}{cc}
%         &  \crs \leftarrow \Gen(1^\secpar, q)\\
%         & (C, m, m', i, \Lambda, \Lambda_i')\leftarrow\Adv(\crs)
%    \end{array}
%    } \leq \epsilon(\secpar)
%    .
%\end{gather*}
%We require that the length of the commitment $c$ output by $\Com$,
%and the length of the opening $\Lambda_i$ output by $\Open$,
%are both of length $\poly(\lambda, \log q)$.

%where the prover sends a short commitment $c$ to a vector $(m_1,\ldots,m_1)$,
%such that for any coordinate $i$, the prover can produce an \emph{opening} $\Lambda_i$
%that convinces the verifier that the value of the vector in coordinate $i$ is $m_i$.
%The commitment is \emph{binding}: after committing to $(m_1,\ldots,m_1)$,
%the prover cannot convince the verifier that the $i$-th coordinate is $m_i'$ for any $m_i' \neq m_i$
%(except with negligible probability).

\paragraph{Succinct non-interactive arguments of knowledge (SNARKs).}
A SNARK consists of the procedures:
\vspace{-1ex}
\paragraph{$\Gen(1^\secpar, \ell)\to \crs$:} a setup procedure that takes a security parameter $\lambda$
		and an instance length $\ell$, and generates a $\crs$.
\vspace{-1ex}
\paragraph{$\Pro(\crs, x, w)\to \pi$:} a proof generation algorithm that takes the $\crs$, an instance $x$ of length $\ell$,
		and a witness $w$ of length $\poly(\ell)$,
		and produces a proof $\pi$.
\vspace{-1ex}
\paragraph{$\Ver(\crs, x, \pi)\to \set{0,1}$:} a verification algorithm that takes the $\crs$, an instance $x$ of length $\ell$,
		and a proof $\pi$, and returns 1 or 0 (accept or reject).
\begin{definition} [Succinct Non-Interactive Argument for $\NP$]\label{def:snarg}
Let $\Lan$ be an $\NP$ language, with a verifying machine $M$
such that $x \in \Lan$ iff $\exists w: M(x,w)=1$, and let $\secpar$ be a security parameter. $(\Gen, \Ver, \Pro)$ is a \emph{Succinct Non-Interactive Argument for $\Lan$}
if it satisfies the following properties.
\vspace{-1ex}
\paragraph{Completeness.} For every $x$ and $w$ such that $M(x,w)=1$,
    \begin{gather*}
        \prob{}
        { 
        \begin{array}{ll}
        \Ver(\crs, x, \pi) = 1
        \end{array}
        \middle\vert
        \begin{array}{ll}
        \var{\crs} \leftarrow \Gen(1^\secpar, \ell) \\
        \pi \leftarrow \Pro(\var{\crs}, x, w) \\
        \end{array}
        } = 1
	.
    \end{gather*}
\vspace{-1ex}
\paragraph{Soundness.} For any efficient prover $\Pro^*$, there exists a negligible function $\epsilon(\cdot)$, such that \begin{gather*}
        \prob{}
        {
        \begin{array}{ll}
        \Ver(\var{\crs}, x, \pi^*) = 1
        \end{array}
        \middle\vert
        \begin{array}{ll}
        \var{\crs} \leftarrow \Gen(1^\secpar, \ell) \\
        (x, \pi^*) \leftarrow \Pro^*(\var{\crs}) \\
        \end{array}
        } \leq \epsilon(\secpar)
	.
    \end{gather*}
\vspace{-1ex}
    \paragraph{Efficiency and succinctness.} $\Ver$ runs in time $\poly(\lambda, |\pi|) = \poly(\lambda, \log n)$
    and $\Pro$ runs in time $\poly(\lambda, n)$.
	The length of the proof $\pi$ produced by $\Pro$ is $\poly(\lambda, \log \ell)$.
%and the $\Ver$ procedure should run in time $\poly(\lambda, |\pi|) = \poly(\lambda, \log \ell)$.
%\paragraph{Verifier Efficiency.} 
%\paragraph{Prover Efficiency.} 

\end{definition}
Note that the prover $\Pro^*$ \emph{chooses} the statement $x$ that
it would like to prove, and it does so after seeing the $\crs$.
This is called \emph{adaptive soundness}, and it is stronger than asserting that there does not exist any $x \not \in \calL$
that the prover can cause the verifier to accept with non-negligible probability
(if there existed such an $x$, we could ``hard-wire''
it into the prover in the adaptive definition).

