\documentclass[acm]{acmart}


\usepackage{listings}
\usepackage[colorinlistoftodos]{todonotes}

\usepackage{appendix}
\usepackage{graphicx}
\usepackage[countmax]{subfloat}
\usepackage{caption}
\usepackage[framemethod=tikz]{mdframed}

\setlength{\abovecaptionskip}{0pt} % We use mdframed frametitles
%\usepackage[unicode,pdfstartview=FitH]{hyperref}

\usepackage{algpseudocode}
\usepackage{algorithm}
\usepackage{xspace}


\floatname{algorithm}{Protocol}
% Define an \INPUT command for the algorithmicx environment
\algnewcommand\algorithmicinput{\textbf{INPUT:}}
\algnewcommand\INPUT{\item[\algorithmicinput]}
\newsubfloat{algorithm}

%%%% Fix bad autorefs
% Correctly autoref appendix
\newcommand*{\Appendixautorefname}{Appendix}
% Correctly autoref protocols
\newcommand{\algorithmautorefname}{Protocol}

% For nice autorefs
\usepackage{cleveref}

\crefname{algorithm}{protocol}{protocols}
\crefname{hybrid}{hybrid}{hybrids}
\crefname{step}{step}{steps}

%%
%% \BibTeX command to typeset BibTeX logo in the docs
\AtBeginDocument{%
  \providecommand\BibTeX{{%
    Bib\TeX}}}

%% Rights management information.  This information is sent to you
%% when you complete the rights form.  These commands have SAMPLE
%% values in them; it is your responsibility as an author to replace
%% the commands and values with those provided to you when you
%% complete the rights form.
%\setcopyright{acmcopyright}
%\copyrightyear{2023}
%\acmYear{2023}
%\acmDOI{XXXXXXX.XXXXXXX}

%% These commands are for a PROCEEDINGS abstract or paper.
%\acmConference[PODC 2023]{ACM Symposium on Principles of Distributed Computing}{June 19--23,
  %2023}{Orlando, FL}
%%
%%  Uncomment \acmBooktitle if the title of the proceedings is different
%%  from ``Proceedings of ...''!
%%
%%\acmBooktitle{Woodstock '18: ACM Symposium on Neural Gaze Detection,
%%  June 03--05, 2018, Woodstock, NY}
%\acmPrice{15.00}
%\acmISBN{978-1-4503-XXXX-X/18/06}



%%%%%%%%%%%% Eden's commands %%%%%%%%5
%\newcommand{\Enote}[1]{\textcolor{violet}{Eden: #1}}
%\newcommand{\prob}[2]{\ensuremath{\underset{#1}{\text{Pr}}\left [#2\right ]}}
%\newcommand{\Prob}[1]{\ensuremath{\emph{Pr}\left [#1\right ]}}
%\newcommand{\poly}{\operatorname{poly}}
%\newcommand{\polylog}{\operatorname{polylog}}
%\newcommand{\negl}{\mathsf{negl}}
%\newcommand{\Lan}{\ensuremath{\mathcal{L}}}
%%\newcommand{\NP}{\mathsf{NP}}
%%\newcommand{\PP}{{\mathsf{P}}}
%\newcommand{\Adv}{\ensuremath{\mathcal{A}}}


%<draft>%
% Tal's macros
\newcommand{\addcite}{{\bf[Cite]}}
%\newcommand{\tc}[1]{{\bf Tal's Comment: \emph{#1}}}
\newcommand{\tc}[1]{ \todo[inline,color=green!40,size=\small]{Tal: #1}}
\newcommand{\rc}[1]{ \todo[inline,color=blue!30,size=\small]{Ran: #1}}
\newcommand{\rcm}[1]{ \todo[color=blue!30,size=\small]{Ran: #1}}

\newcommand{\ignore}[1]{}
%<end>%

%\newcommand{\eps}{\ensuremath{\varepsilon}\xspace}
\newcommand{\eps}{\epsilon}
\newcommand{\set}[1]{\left \{{#1} \right \}}
\newcommand{\abs}[1]{\left \|{#1} \right \|}
\newcommand{\N}{\ensuremath{\mathbb{N}}\xspace}
\newcommand{\Z}{\ensuremath{\mathbb{Z}}\xspace}
\newcommand{\defeq}{\ensuremath{\stackrel{=}{def}}}
%\providecommand{\poly}{\mbox{poly}}
\newcommand{\remove}[1]{}
\newcommand{\B}{\set{0,1}}
\newcommand{\ar}{\rightarrow}
\newcommand{\lar}{\leftarrow}
\newcommand{\Ar}{\Rightarrow}
\newcommand{\Lar}{\Leftarrow}
\newcommand{\Lrar}{\Leftrightarrow}

\newcommand{\secpar}{\lambda}
\newcommand{\crs}{\ensuremath{\mathsf{crs}}\xspace}
\newcommand{\rt}{\ensuremath{\mathsf{rt}}\xspace}
\newcommand{\td}{\ensuremath{\mathsf{td}}\xspace}
\newcommand{\aux}{\ensuremath{\mathsf{aux}}\xspace}
\newcommand{\comm}{\ensuremath{\mathsf{com}}\xspace}
\newcommand{\crsvc}{\crs_\textsc{vc}}
\newcommand{\crssnark}{\crs_\textsc{snark}}
\newcommand{\pisnark}{\pi_\textsc{snark}}

\newcommand{\adv}{\ensuremath{\mathcal{A}}\xspace}
\providecommand{\half}{\frac{1}{2}}
\providecommand{\third}{\frac{1}{3}}

\newcommand{\ith}{$i$\textsuperscript{th}\xspace}
\newcommand{\jth}{$j$\textsuperscript{th}\xspace}
\newcommand{\kth}{$k$\textsuperscript{th}\xspace}
\newcommand{\nth}{$n$\textsuperscript{th}\xspace}


% Setup frames
\mdfsetup
{
	roundcorner=10pt,
	leftmargin=2em,
	frametitleaboveskip=0pt,
}


\newenvironment{nicefig}[3][htp!]{ %
	\begin{figure}[#1]
        \captionsetup{skip=0pt}
		\mdfsetup{frametitle={%
				\tikz[baseline=(current bounding box.east),outer sep=0pt]
				\node[anchor=east,rectangle,line width=1pt,draw,rounded corners=10pt,fill=white]
				{\strut \parbox{\widthof{Figure: \thefigure}+\widthof{#2}+1em}{\captionof{figure}{#2\label{#3}}}};},%
			innertopmargin=5pt,%
			linewidth=1pt,topline=true,
			frametitleaboveskip=\dimexpr-\ht\strutbox\relax,
		}
		\begin{mdframed}[]\relax
		}%
		{%
		\end{mdframed}
	\end{figure}
}

\newenvironment{mydesc}{
    \begin{list}{}{
        \setlength{\itemsep}{0pt}
        \renewcommand{\makelabel}[1]{\bfseries{##1}}
}}{\end{list}}

\newenvironment{myenum}[1][]{
    \begin{enumerate}[#1]
    \setlength{\topsep}{0pt}
    \setlength{\itemsep}{0pt}
}{\end{enumerate}}

\newenvironment{caseenum}{
    \begin{enumerate}[label=Case \arabic*:,ref=\arabic*]
    \setlength{\topsep}{0pt}
    \setlength{\itemsep}{0pt}
}{\end{enumerate}}

\newenvironment{caseenumi}{
    \begin{enumerate}[label=Case \arabic{enumi}.\arabic*:,ref=\arabic{enumi}.\arabic*]
    \setlength{\topsep}{0pt}
    \setlength{\itemsep}{0pt}
}{\end{enumerate}}

\newenvironment{caseenumii}{
    \begin{enumerate}[label=Case \arabic{enumi}.\arabic{enumii}.\arabic*:,ref=\arabic{enumi}.\arabic{enumii}.\arabic*]
    \setlength{\topsep}{0pt}
    \setlength{\itemsep}{0pt}
}{\end{enumerate}}

\newenvironment{caseenumiii}{
    \begin{enumerate}[label=Case \arabic{enumi}.\arabic{enumii}.\arabic{enumiii}.\arabic*:,ref=\arabic{enumi}.\arabic{enumii}.\arabic{enumiii}.\arabic*]
    \setlength{\topsep}{0pt}
    \setlength{\itemsep}{0pt}
}{\end{enumerate}}



% ALGORITHMS %%%%%%%%%%%%%%%%%%%%%%%%%%%%%%%%%%%%%%%%%%%%%%%%%%%%%%%%
\renewcommand{\algorithmiccomment}[1]{ // {#1}}
\floatname{algorithm}{Protocol}

\newif\ifLNCS

% Change below to true if draft version should be LNCS
%<draft>%
\LNCSfalse
\ifLNCS\relax\else
\newtheorem{theorem}{Theorem}[section]
\numberwithin{equation}{section}
\numberwithin{figure}{section}

\newtheorem{lemma}[theorem]{Lemma}

\newtheorem{claim}[theorem]{Claim}

\theoremstyle{definition}
\newtheorem{definition}[theorem]{Definition}
\newtheorem{remark}[theorem]{Remark}
\fi








% MACROS RELATED TO LOCAL ALGORITHMS %%%%%%%%%%%%%%%%%%%%%%%%%%%%%%%%%%%%%%%%%%%
\newcommand{\var}[1]{\mathit{ #1 } }
\newcommand{\Rnote}[1]{ { \color{purple} Rotem: #1 } }
\newcommand{\Enote}[1]{ { \color{violet} Eden: #1 } }
\newcommand{\TODO}[1]{ { \color{red} TODO: #1 } }


\DeclareMathOperator{\dist}{dist}
\DeclareMathOperator{\diam}{diam}
\newcommand{\acc}{\var{accept}}
\newcommand{\rej}{\var{reject}}
\newcommand{\calL}{\mathcal{L}}
\newcommand{\ID}{\var{ID}}
\newcommand{\calF}{\mathcal{F}}
\newcommand{\calB}{\mathcal{B}}
\newcommand{\calH}{\mathcal{H}}
\newcommand{\calC}{\mathcal{C}}
\newcommand{\calG}{\mathcal{G}}
\newcommand{\calX}{\mathcal{X}}
\newcommand{\calY}{\mathcal{Y}}
\newcommand{\calW}{\mathcal{W}}
\newcommand{\nat}{\mathbb{N}}
\newcommand{\integers}{\mathbb{Z}}
\newcommand{\n}{\mathsf{n}}



\newcommand{\R}{{\mathsf{R}}}
\newcommand{\RE}{{\mathsf{RE}}}
\newcommand{\coRE}{{\mathsf{coRE}}}
\newcommand{\LD}{{\mathsf{LOCAL}}}
\newcommand{\NLD}{{\mathsf{NLOCAL}}}
\newcommand{\NP}{{\mathsf{NP}}}
\newcommand{\coNP}{{\mathsf{coNP}}}
\newcommand{\E}{{\mathsf{E}}}
\newcommand{\Exp}{{\mathsf{Exp}}}
\newcommand{\DTIME}{{\mathsf{DTIME}}}
\newcommand{\PH}{{\mathsf{PH}}}
\newcommand{\PP}{{\mathsf{P}}} %\P was taken
\newcommand{\LDP}{{\mathsf{PolyLOCAL}}}
%\newcommand{\NLDP}{{\mathsf{NLD}^P}}
\newcommand{\ALL}{{\mathsf{ALL}}}
\newcommand{\UP}{\mathsf{UP}}
\newcommand{\coUP}{\mathsf{coUP}}
\newcommand{\DDH}{\mathrm{DDH}}




\newcommand{\Piloc}[1]{{\Pi_{#1}^{\var{local}}}}
\newcommand{\Siloc}[1]{{\Sigma_{#1}^{\var{local}}}}
\newcommand{\Pip}[1]{{\Pi_{#1}^{P}}}
\newcommand{\Sip}[1]{{\Sigma_{#1}^{P}}}
\newcommand{\Piploc}[1]{{\Pi_{#1}^{P\text{-}\var{local}}}}
\newcommand{\Siploc}[1]{{\Sigma_{#1}^{P\text{-}\var{local}}}}


\newcommand{\Pioloc}{\Piloc{1}}
\newcommand{\Pioploc}{\Piploc{1}}
\newcommand{\Piop}{\Pip{1}}
\newcommand{\Pitloc}{\Piloc{2}}
\newcommand{\Pitploc}{\Piploc{2}}
\newcommand{\Pitp}{\Pip{2}}
\newcommand{\Pikloc}{\Piloc{k}}
\newcommand{\Pikploc}{\Piploc{k}}
\newcommand{\Pikp}{\Pip{k}}
\newcommand{\Sioloc}{\Siloc{1}}
\newcommand{\Sioploc}{\Siploc{1}}
\newcommand{\Siop}{\Sip{1}}
\newcommand{\Sitloc}{\Siloc{2}}
\newcommand{\Sitploc}{\Siploc{2}}
\newcommand{\Sitp}{\Sip{2}}
\newcommand{\Sikloc}{\Siloc{k}}
\newcommand{\Sikploc}{\Siploc{k}}
\newcommand{\Sikp}{\Sip{k}}

\newcommand{\LDnP}{\PLD^{[\n]}}
\newcommand{\LDn}{\LD^{[\n]}}
\newcommand{\PLD}{\mathsf{PolyLOCAL}}
\newcommand{\NPLD}{\mathsf{NPolyLOCAL}}

%languages
\newcommand{\HALT}{{\mathrm{HALT}}}
\newcommand{\HALTC}{{\overline{\mathrm{HALT}}}}
\newcommand{\HALTe}{{\mathrm{HALT}_\epsilon}}
\newcommand{\HALTeC}{{\overline{\mathrm{HALT}_\epsilon}}}
\newcommand{\HALTd}{{\HALT^D}}
\newcommand{\HALTCd}{{\HALTC^D}}

\newcommand{\ITER}{{\mathrm{ITER}}}
\newcommand{\ITERIN}{{\mathrm{ITER\text{-}BOUND}}}
\newcommand{\PITER}{{\mathrm{POLY\text{-}ITER}_c}}
\newcommand{\ALTS}{{\mathrm{ALTS}}}
\newcommand{\EXTS}{{\mathrm{EXTS}}}
\newcommand{\clq}{{\mathrm{CLIQUE}}}
\newcommand{\cld}{{\mathrm{CLIQUE}^D}}
\newcommand{\nocld}{{\mathrm{no\text{-}CLIQUE}^{D}}}
\newcommand{\extclqd}{{\mathrm{EXACT\text{-}CLIQUE}^D}}
\newcommand{\extclq}{{\mathrm{EXACT\text{-}CLIQUE}}}
\newcommand{\Keyex}{\mathrm{KEYEX}}
\newcommand{\Path}{\text{\textsc{Path}}}
\newcommand{\instgen}{\mathcal{IG}}

%other shortenings

\newcommand{\LOCAL}{{\mathcal{LOCAL}}}

\newcommand{\LDcond}[0]{\begin{gather*}
	(G,x)\in \mathcal{L} \Rightarrow \forall id\in \IDG \forall u\in V(G): \mathcal{A}_{G,x}(u)=accept \notag\\
	(G,x)\notin \mathcal{L} \Rightarrow \forall id \in \IDG \exists u\in V(G): \mathcal{A}_{G,x}(u)=reject
\end{gather*}}
\newcommand{\pitercond}{$$\PITER = \left\{ (G,x)\in \ITER | M_x^{time}(a)\leq |a|^c, M_x^{time}(b)\leq |b|^c  \right\}$$}
\newcommand{\ALTSdef}{$$\ALTS = \left\{(G,x):\ \forall u\in V(G), x(u)\in \{\top,\bot\}\ and\  \left|\left\{u\in V(G): x(u)=1\right\}\right|\geq 2\right\}$$}
\newcommand{\EXTSdef}{$$\EXTS = \left\{(G,x):\ \forall u\in V(G), x(u)\in \{\top,\bot\}\ and\  \left|\left\{u\in V(G): x(u)=1\right\}\right|= 2\right\}$$}


\newcommand{\cldcond}{$$\cld =  \left\{ (G,x):\ \exists k\in \mathbb{N} \forall u\in V(G), x(u)=k \ and\ G \ has\ a\ k\mathsf{-clique} \right \}$$}
\newcommand{\nocldcond}{$$\nocld =  \left\{ (G,x):\  \exists k\in \mathbb{N} \forall u\in V(G), x(u)=k \ and\ G \ does\ not\ have\ a\ k\mathsf{-clique} \right \}$$}
\newcommand{\extclqdcond}{$$\extclqd =  \left\{ (G,x):\  \exists k\in \mathbb{N} \forall u\in V(G), x(u)=k \ and\ the\ largest\ clique\ in\ G \ is\ of\ size\  k  \right \}$$}
\newcommand{\IDG}{{\mathsf{ID}(G)}}
\newcommand{\A}{{\mathcal{A}}} %for an algorithm
\newcommand{\polyhie}{\textit{polynomial hierarchy}}
\newcommand{\pls}{\textit{proof labeling schemes}}
\newcommand{\lcl}{\textit{locally checkable proofs}}
\newcommand{\idobl}{\textit{identity oblivious}}
\newcommand{\CG}{\mathcal{C}(G)}
\newcommand{\balliuResult}{{\LD\subsetneq\Pioloc\subsetneq\NLD=\Sitloc\subsetneq\Pitloc=\ALL}}

\newcommand{\prob}[2]{\ensuremath{\underset{#1}{\text{Pr}}\left [#2\right ]}}
\newcommand{\Prob}[1]{\ensuremath{\emph{Pr}\left [#1\right ]}}
\newcommand{\poly}{\operatorname{poly}}
\newcommand{\polylog}{\operatorname{polylog}}
\newcommand{\negl}{\operatorname{negl}}
\newcommand{\Lan}{\ensuremath{\mathcal{L}}}
\newcommand{\Adv}{\ensuremath{\mathcal{A}}}
\newcommand{\Pro}{\mathcal{P}}
\newcommand{\Ver}{\mathcal{V}}
\newcommand{\VC}{\mathsf{VC}}
\newcommand{\Gen}{\mathsf{Gen}}
\newcommand{\Com}{\mathsf{Com}}
\newcommand{\Open}{\mathsf{Open}}
\newcommand{\Verify}{\mathsf{Ver}}
\newcommand{\VCGen}{\VC.\Gen}
\newcommand{\VCCom}{\VC.\Com}
\newcommand{\VCOpen}{\VC.\Open}
\newcommand{\VCVer}{\VC.\Verify}
\newcommand{\GC}{\mathsf{GC}}
\newcommand{\GCGen}{\GC.\Gen}
\newcommand{\GCCom}{\GC.\Com}
\newcommand{\GCOpen}{\GC.\Open}
\newcommand{\GCVer}{\GC.\Ver}
\newcommand{\GCPro}{\GC.\Pro}
\newcommand{\SNARK}{\mathsf{SNARK}}
\newcommand{\SNARKGen}{\SNARK.\Gen}
\newcommand{\SNARKVer}{\SNARK.\Ver}
\newcommand{\SNARKPro}{\SNARK.\Pro}
\newcommand{\Ext}{\mathcal{E}}
\newcommand{\SNARKExt}{\SNARK.\Ext}
\newcommand{\calR}{\mathcal{R}}
\newcommand{\VerNext}{\Ver\mathsf{Next}}
\newcommand{\calA}{\mathcal{A}}
\newcommand{\SNARG}{\mathsf{SNARG}}
\newcommand{\SNARGGen}{\SNARG.\Gen}
\newcommand{\SNARGHash}{\SNARG.\Hash}
\newcommand{\SNARGVer}{\SNARG.\Ver}
\newcommand{\SNARGPro}{\SNARG.\Pro}
\newcommand{\SNARGExt}{\SNARG.\Ext}
\newcommand{\LVDSNARG}{\mathsf{LVD}-\SNARG}
\newcommand{\DistConstruct}{\mathsf{DistConstruct}}
\newcommand{\DistHash}{\mathsf{DistHash}}
\newcommand{\DistOpen}{\mathsf{DistOpen}}
\newcommand{\DistVer}{\mathsf{DistVer}}
\newcommand{\DMT}{\mathsf{DMT}}
\newcommand{\Digest}{\mathsf{Digest}}
\newcommand{\SNARGDigest}{\SNARG.\Digest}
\newcommand{\HT}{\mathsf{HT}}
\newcommand{\MT}{\mathsf{MT}}
\newcommand{\HTGen}{\HT.\Gen}
\newcommand{\HTHash}{\HT.\Hash}
\newcommand{\HTOpen}{\HT.\Open}
\newcommand{\HTVer}{\HT.\Verify}

\newcommand{\crssnarg}{\crs_{\SNARG}}
\newcommand{\pisnarg}{\pi_{\SNARG}}

\newcommand{\Lancom}{\ensuremath{\Lan^{\comm}}}

\newcommand{\Hash}{\mathsf{Hash}}

\newcommand{\calU}{\mathcal{U}}
\newcommand{\calP}{\mathcal{P}}

\newcommand{\hash}{\mathsf{Hash}}


\newcommand{\Enc}{\mathsf{Enc}}
\newcommand{\Dec}{\mathsf{Dec}}

\newcommand{\prb}[2]{
	\Pr
	\left[
        \begin{array}{ll}
		#1
        \end{array}
        \middle\vert
        \begin{array}{ll}
		#2
        \end{array}
\right]
}


\algnewcommand\algorithmicforeach{\textbf{for each}}
\algdef{S}[FOR]{ForEach}[1]{\algorithmicforeach\ #1\ \algorithmicdo}
\newcommand{\SCRH}{\mathsf{SCRH}}
\newcommand{\SCRHGen}{\SCRH.\Gen}
\newcommand{\SCRHHash}{\SCRH.\Hash}
\newcommand{\crsh}{\crs_\textsc{h}}
\newcommand{\calD}{\mathcal{D}}
\newcommand{\hk}{\mathsf{hk}}
\newcommand{\val}{\mathsf{val}}
\newcommand{\zo}{\set{0,1}}

\newcommand{\SNARKBreak}{\var{SNARKBreak}}
\newcommand{\VCBreak}{\var{VCBreak}}
\newcommand{\VCCompBreak}{\var{VCCompBreak}}
\newcommand{\ICRBreak}{\var{VCICRBreak}}
\newcommand{\SNARGCompBreak}{\var{SNARGCompBreak}}
\newcommand{\SNARGComBreak}{\SNARGCompBreak}
\newcommand{\SNARGSoundBreak}{\var{SNARGSoundBreak}}

\newcommand{\LIN}{\mathsf{LIN}}
\newcommand{\QR}{\mathsf{QR}}







%%
%% end of the preamble, start of the body of the document source.
\begin{document}

%%
%% The "title" command has an optional parameter,
%% allowing the author to define a "short title" to be used in page headers.
\title{Distributed, Locally-Verifiable SNARGs}
%\title{The Name of the Title Is Hope}

%%
%% The "author" command and its associated commands are used to define
%% the authors and their affiliations.
%% Of note is the shared affiliation of the first two authors, and the
%% "authornote" and "authornotemark" commands
%% used to denote shared contribution to the research.

%%
%% By default, the full list of authors will be used in the page
%% headers. Often, this list is too long, and will overlap
%% other information printed in the page headers. This command allows
%% the author to define a more concise list
%% of authors' names for this purpose.
%\renewcommand{\shortauthors}{Trovato et al.}


%%
%% The abstract is a short summary of the work to be presented in the
%% article.
\begin{abstract}
In local distributed algorithms, nodes are traditionally allowed to have unbounded computational power.
This makes the model incomparable with centralized notions of efficient 
%computing 
computations such as $\PP$ and $\NP$.
In this paper, we study computationally-bounded distributed local decision and ask what can be achieved by computationally-efficient local algorithms and provers.

The contributions of this work are twofold.
First, we study distributed certification, where we
%are interested in
%certifying
wish to certify
that a distributed network satisfies
some
%desired
property, or that a distributed algorithm has produced correct output.
To that end, a \emph{prover} assigns to each node of the network a certificate, and the nodes then interact amongst themselves to verify the proof.
%In this work
%First,
We introduce the notion of \emph{computationally-sound distributed certification}, where instead of requiring perfect soundness against any prover, we require only that a \emph{computationally-efficient} prover must not be able to convince the network of a false statement, except with negligible probability.
%Using tools from cryptography, we show that under computational assumptions,
We show that under certain cryptographic assumptions, any property in $\NP$ can be certified using a polylogarithmic number of bits by a global prover that knows the entire network,
and any computationally-efficient distributed algorithm can be certified by an efficient distributed prover that produces certificates of polylogarithmic length in the algorithm's local computation time, round complexity, and message size.
%Furthermore, we show that the execution of any computationally efficient distributed algorithm can be certified by an efficient \emph{distributed prover} that produces certificates of polylogarithmic length in the algorithm's local computation time, round complexity, and message size.

%Next, we study the effect of restricting local distributed algorithms to be computationally efficient.
Next, we study the effect of restricting the nodes themselves to be computationally efficient.
We introduce the classes $\PLD$ and $\NPLD$ of polynomial-time local decision and nondeterministic polynomial-time local decision, respectively, and compare them to the centralized complexity classes $\PP$ and $\NP$, and to the distributed classes $\LD$ and $\NLD$, which correspond to local deterministic and nondeterministic decision, respectively.
We show that when the size of the network is not known to the nodes, $\PLD \subsetneq \LD \cap \PP$; that is, there exists a problem that can be decided by an inefficient local algorithm and also by a poly-time centralized algorithm, but not by a poly-time local algorithm.
When the size of the network is known, an unconditional separation of $\PLD$ from $\LD \cap \PP$ would imply that $\PP \neq \NP$;
%but under a computational assumption called Decisional Diffie-Hellman (DDH), we are still able to show that $\PLD \subsetneq \LD \cap \PP$.
however, we are still able to show that $\PLD \subsetneq \LD \cap \PP$ assuming that injective one-way functions exist.
When nondeterminism is introduced, the distinction vanishes, and $\NPLD = \NLD \cap \NP$.
\end{abstract}



%%
%% The code below is generated by the tool at http://dl.acm.org/ccs.cfm.
%% Please copy and paste the code instead of the example below.
%%
%\section{Introduction}
\label{sec:intro}

%\begin{subfigures}\label{fig:global}

%\begin{nicefig}[h]{An Example Figure }{fig:example1}
%This is a nice figure.
%\begin{algorithmic}[1]
    %\State Do something \Comment{Doing something}
    %\If {$x > 5$}
        %\State  $x$ is big, Yay!
    %\Else
        %\State $x$ still has room to improve.
    %\EndIf
%\end{algorithmic}
%
%\end{nicefig}
%\begin{nicefig}[h]{Another Example Figure }{fig:example2}
%This is another nice figure
%\end{nicefig}
%
%\end{subfigures}
%

%This is a reference to \cref{fig:example1}, and one to \cref{fig:example2} and one to \cref{fig:example1,fig:example2} together. The top-level fig is \cref{fig:global}.

In distributed graph algorithms, the typical complexity measures that one tries to minimize are related to communication and synchronization: we aim to construct algorithms that use few synchronized communication rounds, and send a small number of messages, each as short as possible. There is a rich body of literature on the power of such algorithms: from the $\mathsf{LOCAL}$ model,
where communication rounds are the key resource, to $\mathsf{CONGEST}$, where bandwidth is the main bottleneck,
and many other combinations and variations.
Distributed algorithms are typically allowed to have unbounded \emph{local} computational power,
with each network node able to compute any Turing-computable function for free (e.g., ~\cite{fraigniaud2013can,fraigniaud2013towards}), or sometimes even any function at all (e.g.,~\cite{NS95}).
This puts the theory of local decision on completely different footing from classical centralized notions of efficiency, such as $\PP$ and $\NP$, and makes them impossible to compare.

In this paper, we study \emph{computationally-bounded} distributed local decision, and ask what can be achieved by computationally-efficient local algorithms and provers.
We show that computational restrictions can be {helpful} for prover-assisted distributed certification, but on the other hand,
when there is no prover,
computational restrictions
do limit the power of local algorithms beyond what one might expect.

\paragraph{Computationally-sound distributed proofs.}
In distributed certification (also known as \emph{proof labeling schemes}~\cite{korman2005proof} or \emph{locally checkable proofs}~\cite{LCP}), our goal is to certify some property of the network: for example, one might wish to certify that the output of a distributed algorithm is correct, or that the network graph has some desirable property.
To facilitate this goal, we enlist the help of a \emph{prover}, which provides each node with a short certificate;
the nodes exchange their certificates with one another (or more generally, carry out some efficient verification procedure)
and decide whether to accept or reject.
The proof is considered to be \emph{accepted} if all nodes accept.
While many useful properties can be certified using short certificates, some problems are known to require very long certificates and a lot of communication between the nodes, up to $\Omega(n^2)$ bits~\cite{LCP}.
%Moreover, there is no general way to obtain short certificates proving the correctness of an efficient distributed algorithm,

The area of distributed certification has so far stood apart from the rich theory of  centralized decision problems (e.g., $\PP$ vs.\ $\NP$) and delegated computation. In the centralized setting, under cryptographic assumptions, a computationally-bounded prover can present a weak verifier with a short proof that convinces the verifier of a statement of the form ``$x \in \calL$'', where $\calL$ is any language in $\PP$~\cite{choudhuri2021snargs, cryptoeprint:2022/1320}, or, under stronger assumptions, in $\NP$~\cite{micali2000computationally, bitansky2013recursive, groth2016size}. This is called a \emph{computationally-sound proof}, or a \emph{succinct argument} for $\calL$. 
The key is that rather than requiring perfect soundness against any prover, we require only that the verifier not be fooled by any computationally-bounded prover, except with negligible probability. In return, we get much shorter proofs. If the argument is non-interactive, it is called a \emph{succinct non-interactive argument} (SNARG), and if it also has the property that whenever the prover convinces the verifier, one can extract an $\NP$-witness from the prover, then the argument is a \emph{succinct non-interactive argument of knowledge} (SNARK).

We ask whether the same can be done in the distributed setting,	and show that the answer is \emph{yes}, under standard cryptographic assumptions in the case of language in $\PP$, or under somewhat strong assumptions for languages in $\NP$. We define \emph{succinct distributed arguments}, which are computationally-sound (non-interactive) proofs in the distributed network setting, and show:
 \begin{theorem}\label{thm:centralized}
     Let $\Lan$ be a language on graphs.
     \begin{enumerate}
         \item If $\Lan\in\PP$, assuming SNARGs for $\PP$ and collision-resistant hash function exist,%
		\footnote{
            We introduce these primitives in Section~\ref{sec:related} and~\ref{sec:prelim}, and discuss concrete hardness assumptions under which they are known to exist  in Appendix~\ref{app:crypto} and~\ref{app:concrete}.
         	}
		 there is a succinct distributed argument for $\Lan$ with certificates of length $\polylog(n)$.
         \item If $\Lan\in \NP$, assuming SNARKs for $\NP$ and %Vector Commitments exist,
         collision-resistant hash functions exist,
	 there is a distributed argument for $\Lan$ using certificates of length $\polylog(n)$.    
     \end{enumerate}
 \end{theorem}

 \paragraph{Certifying executions of efficient distributed algorithms.}
	One of the main motivations for studying distributed certification is fault-tolerance and self-stabilization:
	to cope with a dynamic and fault-prone environment, it is useful to be able to identify when 
	the network is  in an illegal state, so that we can undertake actions to correct the problem.
	Proof labeling schemes were originally defined at least in part with this motivation in mind~\cite{korman2005proof}.
	One property that is very interesting to certify is whether output that was
	previously produced by a distributed algorithm $\calD$
	is still up-to-date: if executed in the current network, would $\calD$ still produce the same output?
	It was shown in~\cite{korman2005proof} that if $\calD$ runs in $r$ rounds, and every node sends at most $m$ messages
	of $b$ bits per round,
	then the execution of $\calD$ can be certified using certificates of length $O(r m b)$ bits per node%
	\footnote{In~\cite{korman2005proof} the scheme given is for certifying any property $\calP$ that can be \emph{verified}
	by a distributed algorithm that accepts or rejects at each node. The property ``the algorithm produces the given output'' can certainly be verified by running the algorithm itself and examining its output.}
	by storing the entire history of messages sent at each node.
	Unfortunately, these certificates can be very long when the algorithm uses many rounds or messages.

	With this motivation in mind, we show that a distributed algorithm that runs in polynomial number of rounds, message size
	and local computation time can \emph{certify its own execution}, using certificates of polylogarithmic length
	at each node, and incurring an additive overhead to the running time that 
	is linear in the diameter of the graph.
	

\begin{theorem}
    Let $\calD$ be a distributed algorithm that runs in $T = \poly(n)$ rounds
    and sends messages of length $\poly(n)$.
    Assuming SNARKs for $\NP$ %exist,
    and %assuming
    a certain type of collision-resistant hash functions exist,%
    \footnote{We require a new property called \emph{sum collision resistance}, see Section~\ref{sec:distprover}.}
    there is a distributed argument of length $\polylog(n)$
    certifying $\calD$'s execution, where the prover is an efficient distributed algorithm
    running in $O(T + \diam(G))$ rounds and sending messages of $\polylog(n)$ bits.
    \label{thm:distprover_informal}
\end{theorem}

\paragraph{Computationally-bounded local decision.}
The power of local decision algorithms has been extensively studied, under many variations (e.g.,~\cite{NS95,fraigniaud2013can,fraigniaud2013towards}, and many others),
perhaps the most famous of which is the $\LD$ model.
In all cases (to our knowledge), the algorithm is allowed unbounded local computational power, and as a result,
deterministic local decision is incomparable with the usual notion of computational efficiency, the class $\PP$
of polynomial-time algorithms.
To bridge this gap, we define the class $\PLD(t)$ of local distributed algorithms that run in 
$t(n)$ synchronous rounds, and require local computation time $\poly(n)$ at every node.
(The size of the network is not necessarily known to the nodes;
we consider both options.)

What is the power of algorithms in $\PLD(t)$?
Clearly, such algorithms cannot decide languages that are not in $\PP$, nor can they decide languages 
that are not in $\LD(t)$ (i.e., decidable in $t(n)$ rounds with no computational restrictions).
But can they decide every language in $\LD(t) \cap \PP$?
It turns out that the answer is ``probably not'', 
but whether or not we can prove it unconditionally depends on whether the nodes know the size of the network,
and thus know \emph{how long} they are allowed to run.
Let $\LDn, \LDnP$ be variants of $\LD$ and $\PLD$ (resp.) where nodes know the size of the network.
Then we can show:

\begin{theorem}\label{theo:local}
	%For every locality radius $t = o(n)$,
	We have:
	\begin{enumerate}
		\item $\PLD(o(n)) \subsetneq \LD(o(n)) \cap \PP$;%
			\footnote{
A preliminary version of part (1) of Theorem~\ref{thm:local} 
appeared in the brief announcement~\cite{BA}.}
		\item If $\LDnP(o(n)) \neq \LDn(o(n)) \cap \PP$, then $\PP \neq \NP$; and
		\item Assuming injective one-way functions exist,%
			\footnote{A \emph{one-way function}
			is a function that is easy to compute, but hard to invert.
			See Section~\ref{sec:local} for the details.}
			$\LDnP(o(n)) \subsetneq \LDn(o(n)) \cap \PP$.
	\end{enumerate}
	\label{thm:local}
\end{theorem}

When we introduce nondeterminism,
the distinction disappears:
\begin{theorem}\label{theo:NLD}
	Let $\NLD(t)$, $\NPLD(t)$ be the classes of languages decidable by 
	nondeterministic $t(n)$-round algorithms with unbounded or, resp., polynomially-bounded
	local computation time.
	Then $\NPLD(t) = \NLD(t) \cap \NP$.
	\label{thm:nlocal}
\end{theorem}



\paragraph{Organization.}
The remainder of the paper is organized as follows.
In Sections~\ref{sec:related} and~\ref{sec:prelim}
we review the relevant background, discuss some of the cryptographic primitives we use,
and give formal definitions for some of them;
some definitions are deferred to Appendix~\ref{app:crypto}. % \TODO{make this this is the right one}.
In Section~\ref{sec:dargs} we define succinct distributed arguments, and 
show how to construct them for $\NP$-languages and for languages in $\PP$
(Theorem~\ref{thm:centralized}).
In Section~\ref{sec:distprover} we construct the distributed prover of Theorem~\ref{thm:distprover_informal}.
Finally, in Section~\ref{sec:local},
we discuss the power of computationally-efficient distributed algorithms,
and prove Theorem~\ref{thm:local}.
Many of the proofs, as well as pseudocode for the constructions in Sections~\ref{sec:dargs}
and~\ref{sec:distprover}, are deferred to the appendix.



%\section{Background and Related Work}
\label{sec:related}

\paragraph{Distributed certification.}
Although its roots trace back to work in self-stabilization,
the field of distributed certification was formally initiated in~\cite{korman2005proof},
which introduced \emph{proof labeling schemes}, and showed several constructions and impossibility results,
among them the scheme for certifying spanning trees which is used in the current paper (and is
a central building block for many certification schemes).
Many variants of the basic model have been studied, featuring different communication constraints
for the verifiers (e.g.,\cite{ostrovsky2017space,patt2017proof,FFHPP21}),
allowing randomization or interaction with the prover (e.g.,~\cite{baruch2015randomized,KOS18,NPY20}),
and studying other settings;
we refer to the excellent survey~\cite{CertSurvey} for a comprehensive overview.
To our knowledge, in all prior work, the prover and the verifier have unbounded local computational power.
%
In~\cite{EGK22}, the authors consider {locally-restricted proof labeling schemes},
where the prover itself is a (computationally-unbounded) local algorithm;
however, the proof is required to be sound against any prover, not just a local one.

\paragraph{Local distributed decision.}
Local algorithms have received an enormous amount of attention from the community,
and local decision in particular.
Over the past decade there has been a significant effort towards building
a complexity theory for the area: for example, in~\cite{fraigniaud2013towards},
the authors study the classes $\mathsf{LD}, \mathsf{BPLD}$ and $\mathsf{NLD}$ of languages decidable
by deterministic, randomized, or nondeterministic local algorithms,%
\footnote{In~\cite{fraigniaud2013towards} and follow-up, the output
of the algorithm may depend on the nodes' identifiers.
%and there is particular emphasis on the role of identifiers
%and their effect on expressive power.
%In this work
Here we do not restrict the way that identifiers may be used;
for this reason we use the notation $\LD, \NLD$
instead of $\mathsf{LD}, \mathsf{NLD}$.}
% to denote the languages decidable by local deterministic and nondeterministic algorithms, respectively.}
relate them to one another, and prove (among other results) that combining randomization
and nondeterminism allows a constant-round local algorithm to decide any language.
We refer to the survey~\cite{LDSurvey} for an overview of the area of local decision.
Again, to our knowledge, in prior work the local computation power of the nodes is
always unbounded.

%\paragraph{Computationally sound proofs.}
The idea of a proof system that is only sound against adversaries with bounded computational power
was introduced by Micali~\cite{micali2000computationally},
who gave an implementation 
based on an earlier interactive protocol by Kilian~\cite{kilian1992note}.
%Micali defined the \emph{random oracle model} (ROM),
%an idealized model for hash functions, and showed that Kilian's protocol 
%can be made non-interactive in the random oracle model
%using the Fiat-Shamir paradigm~\cite{fiat1986prove}.
Since Micali's work
there has been extensive effort to obtain succinct, non-interactive arguments (SNARGs) in more realistic models, such as the \emph{Common Reference String} (CRS) model (see Section~\ref{sec:prelim} and Appendix~\ref{app:crypto}).
Very recently SNARGs for all languages in $\PP$ were constructed from standard cryptographic assumptions~\cite{kalai2019delegate, jawale2021snargs, choudhuri2021snargs, cryptoeprint:2022/1320}. For languages in $\NP$, \cite{gentry2011separating} presented a substantial barrier to constructing SNARGs from standard hardness assumptions, and indeed, all known constructions of SNARGs use \emph{knowledge assumptions}, which are considered nonstandard.
%
Knowledge assumptions capture the intuition that an algorithm whose output implicitly relates to some
hard-to-compute value must \emph{obtain} that value during its computation.%
\footnote{For example, 
the knowledge-of-exponent assumption~\cite{KOE}
essentially asserts that given a cyclic group $G$ of prime order,
a generator $g$ of $G$, and an element $h = g^a$ for some $a \in [|G|]$,
if we want to compute a pair $(c,y)$ such that $c^a = y$,
we must \emph{compute the exponent $a$}, which is believed to be hard (this is the \emph{discrete log assumption}).}
Under such assumptions, a SNARG candidate becomes even stronger---it becomes
a \emph{SNARG of knowledge}, a SNARK:
under the knowledge assumption, whenever the prover manages to convince the verifier to accept,
we can extract from the prover a \emph{witness}.%
\footnote{E.g., in the knowledge-of-exponent example, we can extract the exponent $a$.}
The ability to extract a witness is useful for composing SNARKs with other primitives,
as we do in Sections~\ref{sec:dargsForNP},~\ref{sec:distprover}.
Despite the barrier of~\cite{gentry2011separating} on constructing
SNARKs from standard cryptographic assumptions, they are 
nevertheless used on some blockchains, including Ethereum~\cite{ZKSnarks} and others~\cite{filecoin,Zcash,Starkware}.
%the SNARG candidate becomes stronger (it becomes a SNARG of knowledge --- a SNARK); instead of only being sound, we can promise (under the knowledge assumption), that any prover that manages to convince the verifier, knows a witness. This is useful for composing such arguments with other primitives, and in particular, this is useful for our construction (in \Cref{sec:dargsForNP,sec:distprover}). We discuss in \cref{??} \TODO{where} the different assumptions under which SNARKs can be constructed; they are much stronger than standard cryptographic assumptions, but nevertheless, SNARKs are used on some blockchains, including Ethereum~\cite{ZKSnarks}. 

%SNARGs for deterministic computation (\cite{kalai2019delegate}, \cite{jawale2021snargs}, \cite{choudhuri2021snargs}) and batch arguments for $\NP$ (\cite{choudhuri2021non}, \cite{hulett2022snargs},  \cite{devadas2022rate}, \cite{cryptoeprint:2022/1320}), incrementally verifiable computation (\cite{paneth2022incrementally}) and the relationship between these primitives.


%(e.g.,~\cite{aiello2000fast, dwork2004succinct, di2008succinct, groth2010short, bitansky2012extractable, bitansky2013recursive}).

%\paragraph{SNARGs vs.\ SNARKs.}
%A SNARG is a short non-interactive proof that convinces the verifier of a statement of the form ``$x \in \calL$'' for some language $\calL$. However, for $\NP$-languages, it is sometimes useful to require more: when the prover convinces the verifier of a statement of the form ``$\exists w R(x,w)$'' (e.g., ``there exists a satisfying assignment for the formula $x$''), we would like to be sure that the prover \emph{knows} a witness $w$ such that $R(x,w)$ holds, and further, we would like to be able to extract $w$ from the prover's code or the proof. If a SNARG has this additional \emph{proof of knowledge} property, it is called a \emph{SNARG of knowledge} (SNARK). SNARKs are especially useful for composition, and two of our constructions (in Sections~\ref{sec:dargsForP} and~\ref{sec:distprover}) use them for this purpose. We discuss in Appendix \TODO{where} the different assumptions under which SNARKs can be constructed; they are much stronger than standard cryptographic assumptions, but nevertheless, SNARKs are used on some blockchains, including Ethereum~\cite{ZKSnarks}.

%its soundness only holds for adversaries of bounded computational power (also known as an argument) was introduced by Micali in \cite{micali2000computationally}, and was based on an earlier interactive protocol of Kilian (\cite{kilian1992note}) that in turn is based on Merkle Trees (\cite{merkle1989certified}) and Probabilistically Checkable Proofs (PCPs) (\cite{babai1991checking, arora1998probabilistic, arora1998proof, feige1991approximating} \TODO{Choose citation}).  %Kilian's protocol uses Merkle Trees induced by a collision-resistant hash function, and PCPs to construct an interactive computationally sound proof, where the entire transcript of the interaction is much shorter than the length of the corresponding $\NP$-witness.
%In Micali's work, he defined the Random Oracle model and then showed that in that model, the interaction in Kilian's protocol can be removed by the Fiat-Shamir paradigm (\cite{fiat1986prove}).

%Since Micali's work, there have been some attempts (\cite{aiello2000fast, dwork2004succinct, di2008succinct, groth2010short, bitansky2012extractable}) to obtain succinct, non-interactive, arguments (SNARGs) in the common reference string (CRS) model (\cite{canetti2001universally} \TODO{verify citation}) \TODO{what crs means for the distributed prover case}, where all parties have access to a common string that was chosen according to some predefined distribution in an earlier stage. Schemes proven secure in the CRS model are secure given that the setup was performed correctly.

%Most of these works use a knowledge assumption. Knowledge assumptions capture the intuition that any algorithm whose output is related to a certain value that is hard to compute (for instance, a convincing proof, that is related to an $\NP$-witness), must obtain that value along the computation. This assumption is non-falsifiable, meaning, one cannot define a game where at the end of the game, it is easy to tell whether the assumption was broken or not. Under such assumptions, the SNARG candidate stronger (it becomes a SNARG of knowledge -- a SNARK); instead of only being sound, we can promise (under the knowledge assumption), that any prover that manages to convince the verifier, knows a witness. This is useful for composing such arguments with other primitives, and in particular, this is useful for our construction.

%In \cite{gentry2011separating}, a substential barrier to proving the soundness of a SNARG for $\NP$ under falsifiable assumptions alone. Since, many works were either focused on what can be done without these knowledge assumptions, which include SNARGs for deterministic computation (\cite{kalai2019delegate}, \cite{jawale2021snargs}, \cite{choudhuri2021snargs}) and batch arguments for $\NP$ (\cite{choudhuri2021non}, \cite{hulett2022snargs},  \cite{devadas2022rate}, \cite{cryptoeprint:2022/1320}), incrementally verifiable computation (\cite{paneth2022incrementally}) and the relationship between these primitives.
%In this work, we use SNARKs (SNARGs of Knowledge) for $\NP$ and SNARGs for $\PP$.

%\section{Preliminaries}
\label{sec:prelim}


\paragraph{Local distributed algorithms.}
We study the \emph{local decision} model of~\cite{fraigniaud2013towards},%
\footnote{Except that unlike~\cite{fraigniaud2013towards,balliu2018can},
we do not restrict the use of UIDs, as explained above.}
where we have an unknown communication network $G$ and an input assignment $x : V(G) \rightarrow \set{0,1}^*$.
The pair $(G, x)$ is called a \emph{configuration},
and we use $n$ to denote the size of the graph ($n = |V(G)|$).
A \emph{distributed language} $\calL$ is a set of configurations.
The \emph{locality radius} of a distributed algorithm is $t : \nat \rightarrow \nat$
if all nodes halt within $t(n)$ rounds in networks of size $n$.
We let $N_G(v)$ denote the neighborhood of node $v$ in $G$, omitting the subscript $G$
when the graph is clear from the context.


We assume that the nodes have unique identifiers, drawn from some large domain $\calU$,
and we typically assume that a UID from $\calU$ can be represented using $O(\log n)$ bits.%
\footnote{This assumption is not essential, as UIDs from a larger domain can be hashed down to $\set{1,\ldots,n}$
in our constructions.}%
\footnote{In Section~\ref{sec:local}, when we consider networks of unknown size, we do not make
any assumptions on the encoding of the UIDs, and in fact our results continue to hold even
in anonymous networks.}
We often conflate nodes with their UIDs.
We assume that we have some linear ordering $\calU$ of the UID space, that is, for any two UIDs $u \neq v$ from $\calU$,
either $u < v$ or $v < u$.

%When we need to encode a graph $G$, we represent it as an adjacency list, $L(G) = (N(v_1),\ldots,N(v_n))$,
%where $N(v_i)$ is the neighborhood of node $v_i$.
%The nodes appear in $L(G)$ in order of their UIDs, $v_1 < \ldots < v_n$.
%
The local computation of each network node $v \in V(G)$ is represented by a Turing machine which takes as input the UID $v$,
the neighborhood $N(v)$
of $v$ in $G$ and its input $x(v)$, and in each round, reads the messages received by $v$ from a dedicated input tape,
and writes the messages sent by $v$ on a dedicated output tape.
Eventually, the machine enters a halting state, which is either accepting or rejecting.
A configuration $(G, x)$ is \emph{accepted} iff all nodes accept, and otherwise the configuration is \emph{rejected};
a distributed algorithm $D$ \emph{decides} the distributed language $\calL(D)$
of configurations $(G,x)$ that are accepted when $D$ is executed in $(G,x)$.

\paragraph{On security parameters, succinctness and efficient provers.}
Throughout \Cref{sec:dargs,sec:distprover}, we use cryptographic primitives
that are sound against adversaries whose running time is bounded, typically polynomially, as a function
of a security parameter, $\lambda \in \nat$.
The \emph{succinctness} of these primitives---that is, the encoding length of whatever object they produce (e.g., the length of a hash value, or a proof)---is $\poly(\lambda, \log n)$.
To get proofs of length $\polylog(n)$, the security parameter we use is $\lambda = \log^c(n)$
for some $c > 1$;
we are interested in adversaries whose running time is polynomial in $n$,
which means they are sub-exponential in $\lambda$. % (e.g., if we set $\lambda = \log^2 n$, then $n^d = 2^{d \sqrt{\lambda}}$).
To allow for such provers, our hardness assumptions require security not against a polynomial-time adversary
but against a sub-exponential one.
It is relatively common to assume sub-exponential hardness;
for example, the learning-with-errors (LWE) problem is believed to be sub-exponentially hard.
%\TODO{citations}.

In the sequel, whenever we say ``efficient adversary/prover'',
we mean sub-exponential in $\lambda$ and polynomial in $n$.

\paragraph{Common reference string (CRS) model}
The cryptographic primitives that we use are proved sound in the \emph{common reference string} (CRS) model,
where we assume that a trusted setup phase has occurred, during which all parties get access to
a \emph{common reference string} drawn from a known distribution.
For example, the CRS can be used to select a random hash function
from a family of hash functions.
%which is essentially shared randomness between the prover and the verifiers (which the prover is not able to influence): for example, the prover and the verifiers may need to agree on a hash function, which is chosen uniformly at random from a large family of functions. CRS is a common assumption, underlying most of the work on delegated computation.
%We believe that CRS is an appropriate assumption for our setting, %where the prover is not truly ``malicious'' but rather ``misguided'' or ``misinformed'', reflecting bugs in the code or changes in the network.
In the distributed prover of Section~\ref{sec:distprover},
the CRS can be generated by having every node $v$ propose a random string $r_v$,
and summing the strings up a spanning tree to produce $\crs = \oplus_{v \in V} r_v$,
which is then disseminated to all the nodes. As long as a single node generates its random string honestly, the resulting $\crs$
will be uniformly random.

\paragraph{Hash functions.}
A hash family is accessed using two procedures, $\Gen$ and $\Hash$:
$\Gen(1^\secpar, \ell)\rightarrow \hk$ is a setup procedure that takes the security parameter $\lambda$
(in unary) and the length $\ell$ of the values to be hashed, and returns a \emph{hash key} $\hk$;
$\Hash(\hk, x)$ takes a hash key and a value $x$, and returns a hashed value.

\paragraph{Vector commitment schemes}
A vector commitment produces a short commitment for a vector $(m_1,\ldots,m_q)$,
such that an efficient adversary cannot later convince a verifier that it committed to a value $m_i' \neq m_i$
in any position $i$.
The commitment
consists of the following algorithms.

%\vspace{-1ex}
\begin{itemize}
    \item $\Gen(1^\secpar, q)\to \crs$: a randomized algorithm that takes as input the security parameter $\secpar$ and the length $q$ of the committed vector, and outputs a common reference string $\crs$.
    \item $\Com(\crs, m_1,\ldots , m_q)\to (c,\aux)$: a deterministic algorithm that takes $q$ a vector $m_1,\ldots, m_q$ and a common reference string $\crs$, and outputs a commitment string $c$ together with auxiliary information $\aux$ (which may just be $m_1,\ldots,m_1$ itself, but can also be more general).
    \item $\Open(\crs, m, i, \aux)\to \Lambda_i$: a deterministic algorithm that takes the \crs, a value $m$, an index $i$, and auxiliary information \aux, and produces a proof $\Lambda_i$ that $m$ is the \ith committed value.
    \item $\Verify(\crs, C, m, i, \Lambda)\to b$: a verification algorithm that takes the \crs, a value $m$, an index $i$, and a proof $\Lambda$, and outputs an acceptance bit.
\end{itemize}

\begin{definition} [Vector Commitments]\label{def:VC}
A VC $(\Gen, \Com, \Open, \Verify)$ is required to satisfy:
\vspace{-1ex}
\begin{itemize}
    \item \emph{Completeness. } For every value sequence, $m_1,\ldots m_q$,
\begin{gather*}
    \prob{}
    {
    \begin{array}{ll}
    \forall i\in [q]: 
         \Verify(\crs, C, m_i, i, \Lambda_i) = 1
    \end{array}
    \middle\vert
    \begin{array}{ll}
         \crs \leftarrow \Gen(1^\secpar, q)\\
         C, aux\leftarrow\Com(\crs, m_1,\ldots,m_q) \\
         \forall i\in [q]: \Lambda_i \leftarrow \Open(\crs, m_i, i, \aux))
    \end{array}
    } = 1
\end{gather*}
    \item \emph{Position-Binding.} For every $i\in [q]$, for any efficient adversary $\Adv$, there exists a negligible function $\epsilon(\cdot)$ such that for every $\secpar$, % the following probability (which is taken over all honestly generated parameters) is at most negligible in $k$:
\begin{gather*}
    \prob{}
    {
    \begin{array}{ll}
    \Verify(\crs, C, m, i, \Lambda_i) = 1 \ \wedge \\
    \Verify(\crs, C, m', i, \Lambda_i') = 1
    \end{array}
    \middle\vert
    \begin{array}{cc}
         &  \crs \leftarrow \Gen(1^\secpar, q)\\
         & (C, m, m', i, \Lambda, \Lambda_i')\leftarrow\Adv(\crs)
    \end{array}
    } \leq \epsilon(\secpar)
\end{gather*}
    \item \emph{Succinctness.} The length of the commitment $c$ output by $\Com$ and the length of the opening $\Lambda_i$ output by $\Open$ are both bounded by $\poly(\lambda, \log q)$.
\end{itemize}
\end{definition}


%Our constructions in Sections~\ref{sec:dargsForNP} and Appendix~\ref{app:dargsForP} use a \emph{vector commitment scheme} (VC), which allows a prover to produce a short commitment $c$ to a vector $(m_1,\ldots,m_q)$, such that for every $1 \leq i \leq q$, the prover can later provide a short \emph{local opening} $\Lambda_i$ that would convince the verifier that the $i$-th value it committed to is $m_i$. A vector commitment is required to be \emph{complete} and \emph{position-binding}. Completeness asserts that verifying an honestly-generated commitment and opening always succeeds. The position-binding property requires that an efficient adversary cannot open the same location of a vector commitment to two different values, such that the verifier is convinced by both, except for a negligible probability \footnote{A function $\eps(n)$ is \emph{negligible} if it is asymptotically smaller than $1/n^c$ for every constant $c$.}. We require that the length of the commitment $c$, and the length of the opening $\Lambda_i$, are both of length $\poly(\lambda, \log q)$.
%which produces a short commitment for a vector $(m_1,\ldots,m_q)$, such that for any $1 \leq i \leq q$, the prover can convince the verifier that the  $i$-th value it committed to is $m_i$, but it cannot convince the verifier of any other value $m_i' \neq m_i$.
%The VC is accessed using the following functions:
%\begin{itemize}
%	\item $\Gen(1^\secpar, q)\to \crs$: takes the security parameter $\lambda$
%		and a vector length $q$, and returns a $\crs$.
%	\item $\Com(\crs, m_1,\ldots , m_q)\to (c,\aux)$: takes the $\crs$ and $q$ values $m_q,\ldots,m_q$,
%		and returns a commitment $c$ and auxiliary information $\aux$.
%	\item $\Open(\crs, m, i, \aux)\to \Lambda_i$: takes the $\crs$, a value $m$, a coordinate $i$
%		and auxiliary information $\aux$,
%		and produces a proof $\Lambda_i$ that $m$ is the $i$-th committed value.
%		We refer to $\Lambda_i$ as the \emph{opening} for coordinate $i$.
%	\item $\Verify(\crs, c, m, i, \Lambda)\to b$: takes the $\crs$,
%		the commitment $c$, a value $m$, a coordinate $i$ and an opening 
%		obtained from $\Open(\crs, m, i, \aux)$,
%		and returns 0 or 1.
%\end{itemize}
%A vector commitment is required to be \emph{complete} and \emph{position-binding}.
%Completeness asserts that verifying an honestly-generated commitment always succeeds:
%\begin{gather*}
    %\prob{}
    %{
    %\begin{array}{ll}
    %\forall i\in [q]: 
         %\Verify(\crs, c, m_i, i, \Lambda_i) = 1
    %\end{array}
    %\middle\vert
    %\begin{array}{ll}
         %\crs \leftarrow \Gen(1^\secpar, q)\\
         %c, aux\leftarrow\Com(\crs, m_1,\ldots,m_q) \\
         %\forall i\in [q]: \Lambda_i \leftarrow \Open(\crs, m_i, i, \aux))
    %\end{array}
    %} = 1
    %.
%\end{gather*}
%The position-binding property requires that an efficient adversary cannot lie about the values it committed to: for every $i\in [q]$, for any efficient adversary $\Adv$, there exists a negligible function%
%\footnote{A function $\eps(n)$ is \emph{negligible} if it is asymptotically smaller than $1/n^c$
%for every constant $c$.}
%$\epsilon(\cdot)$ such that for every $\secpar$,
%\begin{gather*}
%    \prob{}
%    {
%    \begin{array}{ll}
%    \Verify(\crs, C, m, i, \Lambda_i) = 1 \ \wedge \\
%    \Verify(\crs, C, m', i, \Lambda_i') = 1
%    \end{array}
%    \middle\vert
%    \begin{array}{cc}
%         &  \crs \leftarrow \Gen(1^\secpar, q)\\
%         & (C, m, m', i, \Lambda, \Lambda_i')\leftarrow\Adv(\crs)
%    \end{array}
%    } \leq \epsilon(\secpar)
%    .
%\end{gather*}
%We require that the length of the commitment $c$ output by $\Com$,
%and the length of the opening $\Lambda_i$ output by $\Open$,
%are both of length $\poly(\lambda, \log q)$.

%where the prover sends a short commitment $c$ to a vector $(m_1,\ldots,m_1)$,
%such that for any coordinate $i$, the prover can produce an \emph{opening} $\Lambda_i$
%that convinces the verifier that the value of the vector in coordinate $i$ is $m_i$.
%The commitment is \emph{binding}: after committing to $(m_1,\ldots,m_1)$,
%the prover cannot convince the verifier that the $i$-th coordinate is $m_i'$ for any $m_i' \neq m_i$
%(except with negligible probability).

\paragraph{Succinct non-interactive arguments of knowledge (SNARKs).}
A SNARK consists of the procedures:
\begin{itemize}
    \item $\Gen(1^\secpar, \ell)\to \crs$: a setup procedure that takes a security parameter $\lambda$
		and an instance length $\ell$, and generates a $\crs$.
    \item $\Pro(\crs, x, w)\to \pi$: a proof generation algorithm that takes the $\crs$, an instance $x$ of length $\ell$,
		and a witness $w$ of length $\poly(\ell)$,
		and produces a proof $\pi$.
    \item $\Ver(\crs, x, \pi)\to \set{0,1}$: a verification algorithm that takes the $\crs$, an instance $x$ of length $\ell$,
		and a proof $\pi$, and returns 1 or 0 (accept or reject).
\end{itemize}
\vspace{-1ex}
\vspace{-1ex}
\begin{definition} [Succinct Non-Interactive Argument of Knowledge for $\NP$]\label{def:snarg}
Let $\Lan$ be an $\NP$ language, with a verifying machine $M$
such that $x \in \Lan$ iff $\exists w: M(x,w)=1$, and let $\secpar$ be a security parameter. $(\Gen, \Ver, \Pro)$ is a \emph{Succinct Non-Interactive Argument of Knowledge for $\Lan$}
if it satisfies the following properties.
\vspace{-1ex}
\begin{itemize}
    \item For every $x$ and $w$ such that $M(x,w)=1$,
    \begin{gather*}
        \prob{}
        { 
        \begin{array}{ll}
        \Ver(\crs, x, \pi) = 1
        \end{array}
        \middle\vert
        \begin{array}{ll}
        \var{\crs} \leftarrow \Gen(1^\secpar, \ell) \\
        \pi \leftarrow \Pro(\var{\crs}, x, w) \\
        \end{array}
        } = 1
	.
    \end{gather*}
    \item \emph{Argument of Knowledge.} For any efficient prover $\Pro^*$, there exists an efficient extraction algorithm, $\Ext_{\Pro^*}$,
%which produces a witness $w$ for an instance $x$,
%which intuitively produces a witness for any instance that $\Pro^*$ can convince the verifier to accept.
%, such that if $\Pro^*$ generates a statement $x$ and a proof $\pi$ that is accepted by $\Ver$ with some probability $p$,
%then $\Ext^*$ generates a witness $w$ such that $M(x,w) = 1$ with probability close to $p$.
%More formally,
and a negligible function $\epsilon(\cdot)$, such that,
%the probability that the prover ${\Pro^*}$ can convince the verifier to accept $x$ when $\Ext_{\Pro^*}$
%produces an \emph{invalid} witness for $x$ is negligible:
    \begin{gather*}
        \prob{}
        {
        \begin{array}{ll}
        \Ver(\var{\crs}, x, \pi^*) = 1 \\
        \wedge\ M(x,w)\neq 1
        \end{array}
        \middle\vert
        \begin{array}{ll}
        \var{\crs} \leftarrow \Gen(1^\secpar, \ell) \\
        (x, \pi^*) \leftarrow \Pro^*(\var{\crs}) \\
        w\leftarrow \Ext_{\Pro^*}(\var{\crs}, x)
        \end{array}
        } \leq \epsilon(\secpar)
	.
    \end{gather*}
    \item \emph{Succinctness and Efficiency.} The length of the proof $\pi$ produced by $\Pro$ is $\poly(\lambda, \log \ell)$.
    $\Ver$ runs in time $\poly(\lambda, |\pi|) = \poly(\lambda, \log n)$
    and $\Pro$ runs in time $\poly(\lambda, n)$. 
\end{itemize}
\end{definition}
Note that the prover $\Pro^*$ \emph{chooses} the statement $x$ that
it would like to prove, and it does so after seeing the $\crs$.
This is called \emph{adaptive soundness}, and it is stronger than asserting that there does not exist any $x \not \in \calL$
that the prover can cause the verifier to accept with non-negligible probability
(if there existed such an $x$, we could ``hard-wire''
it into the prover in the adaptive definition).


%\section{Succinct Distributed Arguments}\label{sec:dargs}
In this section we define \emph{succinct distributed arguments} and show how to construct them for graph languages in
$\PP$ and in $\NP$.
%In Section~\ref{sec:dargsForP} we give a construction for graph languages in $\PP$, which is similar in spirit
%but uses different cryptographic primitives and has a different soundness proof. In particular, it can be instantiated under \emph{standard} cryptographic assumptions.
For simplicity, in this section and the next,
we restrict attentiont to \emph{graph languages},
where the nodes have no input.
The definition and the constructions easily extend to the case where there are inputs
(see Appendix~\ref{app:code}).

%\subsection{Defining Succinct Distributed Arguments}
%\label{sec:def}
A succinct distributed argument consists of the following algorithms.

\paragraph{$\Gen(1^\secpar, n)\to \var{\crs}$:} a randomized algorithm that takes as input a security parameter $1^\secpar$ and a graph size $n$, and outputs a common reference string $\var{\crs}$.

\paragraph{$\Pro(\var{\crs}, G, w)\to \{\pi_v\}_{v\in V(G)}$:}
%, c, L_i, \Lambda_i)$.}
takes a $\var{\crs}$ obtained from $\Gen(1^{\secpar}, n)$,
, %a node $v$, an index $i$, an adjacency list $L$, where the \ith row is $N(v)$
a graph $G$ on $n$ nodes, and possibly a witness $w$ of length $\poly(n)$ (which may be empty, e.g., for languages in $\PP$),
and outputs a list of proofs $\{\pi(v)\}_{v\in V(G)}$,
one for each node $v \in G$.

\paragraph{$\Ver(\var{\crs}, v, N(v), \pi(v), \pi(N(v)) )\to \{0,1\}$:}
takes a common reference string $\crs$ obtained from $\Gen(1^{\secpar}, n)$,
a UID $v$, a list of neighbors $N(v)$, a proof $\pi(v)$,
and the proofs of the neighbors, $\pi(N(v)) = \set{ \pi(u) : u \in N(v) }$,%
\footnote{For simplicity, we follow the original design of proof labeling schemes~\cite{korman2005proof},
where neighbors only exchange their certificates with their immediate neighbors.
The model can be generalized to allow for more general verification procedures.}
and outputs an acceptance bit.

%\rc{this should be in a definition environment}
%\Enote{I'm not sure... }

\begin{definition}\label{defSDarg}
Let $\Lan$ and $\calR$ be an $\NP$ language and a compatible relation on graphs,
such that $G \in \calL$ iff there exists a witness $w$ such that $(G,w) \in \calR$.
A \emph{succinct distributed argument} for $\Lan$ and $\calR$, denoted $(\Gen, \Pro, \Ver)$, satisfies the following properties:
\paragraph{Completeness}  For $(G,w)\in \calR$,
\begin{gather*}
    \prb
    {
    	\forall v \in V(G):\\
	    \Ver(\crs, v, N(v), \pi(v), \pi(N(v))) = 1
	}
	{
    \crs \leftarrow \Gen(1^\secpar, n) \\
    \set{ \pi(v) }_{v\in V(G)} \leftarrow \Pro(\var{\crs}, G, w)
    }
    = 1
    .
\end{gather*}
\paragraph{Soundness}
For any efficient algorithm $\Pro^*$ %, and $G\notin \Lan$,
there exists a negligible function $\epsilon(\cdot)$ such that
\begin{gather*}
    \prob{}
    {
    \begin{array}{ll}
    G\notin \Lan \\
    \wedge\ \forall v\in V(G) :\\
    \quad \Ver(\var{\crs}, v, N(v), \pi_{v}, \pi(N(v))) = 1
    \end{array}
    \middle\vert
    \begin{array}{ll}
    \var{\crs} \leftarrow \Gen(1^\secpar, n) \\
    ( G, \{\pi_{v}\}_{v\in V(G)}) \leftarrow \Pro^*(\var{\crs}, 1^\secpar, 1^n) \\
    \end{array}
    } \leq \epsilon(\secpar)
    .
\end{gather*}
Note that, as in the definition of SNARKs, the prover gets
to choose the false statement it would like to prove after seeing the $\crs$.

\paragraph{Succinctness} The $\var{\crs}$ and the proof $\pi$ are of length at most $\poly(\secpar, \log n)$.
\paragraph{Efficiency.} $\Ver$ runs in time $\poly(\lambda, |\pi|) = \poly(\lambda, \log n)$,
and $\Pro$ runs in time $\poly(\lambda, n)$.
\end{definition}

%\subsection{Succinct Distributed Arguments for NP from SNARKS}\label{sec:dargsForNP}
We first show how to construct a distributed argument for graph languages in $\NP$ using Vector Commitments and SNARKs,
as this is simpler than the construction for $\PP$ (which does not use SNARKs).
%proving the second part of Theorem~\ref{thm:centralized}.
We give an overview of the construction and the details
can be found in Appendix~\ref{app:dargsForNP}.

%\begin{theorem}\label{theo:NP}
%Let $\Lan$ be a graph language, such that $\Lan\in \NP$. Assuming SNARKs and VC exist, there is a succinct distributed argument for $\Lan$. 
%\end{theorem}

Suppose first that the UIDs in the graph $G$ are $1,\ldots,n$.
The prover constructs the adjacency list $\var{AdjL} = (N(1),\ldots,N(n))$ for $G$,
and
provides all the nodes with the same proof, which consists of
\begin{itemize}
	\item A vector commitment $c$ to the adjacency list $\var{AdjL}$, and
        \item A SNARK proof $\pisnark$ proving that there exists an adjacency list $\var{AdjL}'$ whose vector commitment is $c$, such that the graph represented by $\var{AdjL}'$ is in $\calL$.
\end{itemize}
Additionally, to convince the nodes that $\var{AdjL}' = \var{AdjL}$,
the prover gives each node $i$ an opening to the $i$-th coordinate of the vector commitment, allowing $i$ to verify that the $i$-th coordinate of $c$ opens to its true neighborhood $N(i)$.
If all nodes succeed in their verification, and they all received the same commitment $c$, then $c$ is indeed a commitment to the true graph $G$; the nodes then verify the SNARK proof $\pisnark$, which convinces them that $G \in \calL$.

To handle general UIDs, the prover orders the nodes $V(G)$ by UID, $v_1 < \ldots < v_n$,
and informs each node $v_i$ of its index $i$.
However, the prover can now try to cheat in two ways:
it can give two nodes the same index, or it can commit to the adjacency list of a graph that is
larger than $G$ (that is, an adjacency list of length $> n$).
In both cases, some position of the vector commitment will not be opened by any node,
and the prover could get away with committing to an incorrect graph.

To forestall this we modify the construction slightly:
\begin{itemize}
	\item Instead of committing to the adjacency list $\var{AdjL} = (N(v_1),\ldots,N(v_n))$,
the prover adds the UIDs to the list, and commits to $L = ( (v_1, N(v_1)), \ldots, (v_n, N(v_n)))$.
This prevents the prover from giving the same index to two nodes, as one of the nodes will open 
its entry and see the UID of the other node there.
\item 
We strengthen the property proved by the SNARK,
and ask the prover to prove that there exists a \emph{symmetric} adjacency list $\var{AdjL}'$
whose vector commitment is $c$,
such that the graph represented by $\var{AdjL'}$ \emph{is connected} and in $\calL$.
	\end{itemize}
%
%One issue with this scheme is that the nodes do not initially have an ordering $v_1,\ldots,v_n$ of their UIDs, so each node does not know the coordinate of the vector commitment to which it is supposed to receive an opening. %We can ask the prover to provide such an ordering by giving each node an index, but then the prover may cheat by giving multiple nodes the same index, or by committing to a graph $G'$ that is larger than the real graph $G$ (that is, to an adjacency list $\var{AdjL}'$ that is longer than $\var{AdjL}$); in both cases, some coordinates of the vector commitment are never assigned to any node of $G$, and will never be opened.
%We resolve these issues by modifying the approach above slightly:
%\begin{itemize}
    %\item Instead of committing to the adjacency list $\var{AdjL} = (N(v_1),\ldots,N(v_n))$, the prover commits to a list $L = ( (v_1, N(v_1) ), \ldots, (v_n, N(v_n)) )$ that also includes the UIDs of the nodes, so that when a node opens a given coordinate, it can verify that its own UID appears there.
    %\item To prevent the prover from committing to a graph that is larger than $G$, we ask the prover to prove a stronger property in the SNARK proof: it proves that there exists an adjacency list $\var{AdjL}'$ whose vector commitment is $c$, such that $\var{AdjL'}$ \emph{is symmetric}, and the graph represented by $\var{AdjL}'$ \emph{is connected} and satisfies $\calL$.
%\end{itemize}
If the prover now tries to commit to a list $\var{AdjL'}$ that is longer than the size of the real graph, then since the graph $G'$, represented by $\var{AdjL}'$, is connected, the cut between the ``fake nodes'' $V(G') \setminus V(G)$ and the ``real nodes'' $V(G)$ includes some edge, $\set{u,v}$, where $u \in V(G)$ and $v \not \in V(G)$. Since $G'$ is symmetric, $v \in N_{G'}(u)$. Thus, when node $u$ opens its coordinate in the vector commitment, it will see that its purported neighborhood there includes the ``fake node'' $v$, and it will reject.

	We remark that in the construction as presented above, the prover sends the same SNARK proof to every node,
	and all nodes verify it. This is not needed;
	for example, 
	using an additional $O(\log n)$ bits,
	we can ask the prover to provide a spanning tree of the network~\cite{korman2005proof},
	and have only the root receive the SNARK proof and verify it.

We briefly sketch the soundness proof for this construction (see Appendix~\ref{app:dargsForNP} for the full proof).
Suppose we have an efficient prover $\Pro^*$ that generates ``false statements''
$G \not \in \calL$,
together with certificates $\set{\pi(v)}_{v \in V(G)}$
that are accepted with non-negligible probability.
The certificates include a commitment $c$ to an adjacency list, and a SNARK proof $\pisnark$.
By the argument of knowledge property of the SNARK, we can extract a \emph{witness} from $\Ext_{\Pro^*}$ in the form of an adjacency list $\var{AdjL}'$,
which is supposed to match the commitment $c$ and represent a symmetric and connected graph $G' \in \calL$.
Now there are two cases:
if $\var{AdjL}'$ is \emph{not} a proper witness---if it does not match the commitment $c$,
or it does not represent a graph $G'$ that is symmetric, connected, and in $\calL$---%
then we have broken the \emph{argument of knowledge} property of the SNARK,
by extracting an improper witness for a statement that is accepted with non-negligible probability.
On the other hand, if $\var{AdjL}'$ is a proper witness,
then since $G \not \in \calL$, we know that $\var{AdjL} \neq \var{AdjL}'$ (where $\var{AdjL}$
is the adjacency list of $G$).
We show that this means we have broken the position-binding property of the vector commitment,
by proving that every coordinate of $\var{AdjL}'$ is opened by some node of $G$,
which verifies that its UID and its neighborhood are correctly represented.
The prover is thus able to fool at least one node $v$ into accepting a commitment to
a value that differs from $(v, N(v))$,
violating the position-binding property of the vector commitment.

%\subsection{Succinct Distributed Arguments for $\PP$ from RAM SNARGs}\label{sec:dargsForP}
We give a very high-level sketch of our construction for graph languages in $\PP$,
which does not rely on knowledge assumptions; the details appear in Appendix~\ref{app:dargsForP}.

The construction uses a primitive called a \emph{flexible RAM SNARG for $\PP$}~\cite{KP16,cryptoeprint:2022/1320},
whose precise definition we defer to Appendix~\ref{app:ramsnargs}.
In general, a %RAM
SNARG is used to prove statements of the form ``$M(x) = b$'',
where $M$ is a deterministic polynomial-time Turing machine, $x$ is an input,
and $b \in \set{0,1}$ indicates whether $M$ accepts or rejects the input.%
\footnote{Since $\PP$ is closed under complement, we can prove both membership in $\calL(M)$
and non-membership in $\calL(M)$.}
The key property of the RAM SNARG of~\cite{KP16,cryptoeprint:2022/1320} is that 
the prover and the verifier
are actually not given the instance $x$,
but only a \emph{hash} of $x$, called a \emph{digest},
which is much shorter than the input $x$ itself.
Roughly speaking, the prover then proves the statement ``the input whose digest is $d$
is accepted/rejected by $M$''.

Of course, this is not well-defined: since the digest $x$ is much shorter than $x$,
there may be \emph{many} inputs that have the same digest as $x$---%
some of them may be accepted by $M$, and some may be rejected.
What does it mean to prove that ``the input whose digest is $d$ is accepted/rejected by $M$'',
when this input is not unique?
This is resolved in~\cite{KP16} by re-defining
the soundness of the SNARG:
we now require only that 
an
efficient adversary should not be able to find a digest $d$ and two proofs $\pi_0, \pi_1$,
such that $\pi_0$ convinces the verifier that $M$ rejects,
and $\pi_1$ convinces the verifier that $M$ accepts (both with respect to ``some input'' whose digest is $d$).
This soundness definition suffices for our purposes here.

The digest we use is a vector commitment $c$ to the network graph, which the prover
computes and gives to all the nodes, as in the previous section.
In addition, the prover computes a RAM SNARG proof for the statement ``the graph whose vector commitment
is $c$ is accepted by $M$'', where $M$ is a deterministic Turing machine that decides
the graph language $\calL = \calL(M)$ that we would like to certify.
As before, each node opens its entry in the vector commitment and verifies that its neighborhood
is correctly represented, and then the nodes verify the SNARG proof.

The soundess proof for the new construction is quite different from the previous one:
before, we relied on the proof-of-knowledge property, which allowed us to \emph{extract}
from a cheating prover $\Pro^*$
a concrete graph $G' \neq G$ that has the same vector commitment as $G$,
and argue that $\Pro^*$ breaks the position-binding property of the vector commitment.
A SNARG does not have the proof-of-knowledge property,
so even if the prover $\Pro$* has successfully convinced all nodes to accept a graph $G \not \in \calL$,
this does not mean we can use $\Pro^*$ to find a graph $G' \neq G$ that has the same vector commitment as $G$.
To get around this issue,
we require an additional property from the vector commitment,
which essentially asserts that for any given vector $m = (m_1,\ldots,m_q)$,
there is only one commitment $c$ that opens to $m_i$ at every position $i$:
\begin{definition}[Inverse Collision-Resistance]
A VC $(\Gen, \Com, \Open, \Verify)$ is \emph{Inverse Collision-Resistant} if for any efficient algorithm $\calA$, there
is a negligible $\epsilon(\cdot)$ such that for all $\lambda\in \mathbb{N}$,
\begin{gather*}
    \prob{}
    {
    \begin{array}{ll}
         \forall i: \Verify(\crs, C^*, m, i) = 1 \\
         \wedge C^* \neq C
    \end{array}
    \middle\vert
    \begin{array}{ll}
         \crs \leftarrow \Gen(1^\lambda, q) \\
         C^*, \{(m_i, \lambda_i)\}_{i\in [q]} = \calA(\crs) \\
         C \leftarrow \Com(\crs, m_1,\ldots,m_q)
    \end{array}
    } \leq \epsilon(\lambda).
\end{gather*}
\end{definition}
In Appendix~\ref{app:crypto} we show that a succinct inverse collision-resistant VC can be implemented from 
a collision-resistant hash function using a Merkle tree~\cite{merkle1989certified}.

To conclude our sketch of the soundness proof, suppose a cheating prover $\Pro^*$ is able
to
find a graph $G \not \in \calL(M)$ that is accepted by all the nodes with non-negligible probability.
Since $G \not \in \calL(M)$, we know that $M(G) = 0$,
and we can compute the vector commitment $c$ of $G$,
and an honest SNARG proof $\pi_0$ for the statement ``the graph whose vector commitment is $c$
is rejected by $M$''.
However, since $\Pro^*$ was able to convince all nodes to accept,
it has found a proof $\pi_1$ that convinces them that ``the graph whose vector commitment is $c'$
is accepted by $M$'', where $c'$ is \emph{also} a vector commitment to $G$
(otherwise, some node would open its entry in $c'$, see that its neighborhood is not represented correctly, and reject).
By the inverse collision-resistance property of the commitment,
there can only be one commitment that opens correctly at all nodes, and therefore $c = c'$.
But this violates the soundness of the RAM SNARG, as we have now found a digest ($c$)
and two proofs $\pi_0, \pi_1$,
both of which convince the verifiers,
but they prove opposite statements.

%In this section, we construct a succinct distributed argument for $\PP$. The construction itself is very similar to the one for $\NP$, but they differ in the cryptographic primitives they use. In this construction, we use Flexible RAM SNARGs.%, which are known for $\PP$ from standard cryptographic assumptions (see Appendix \TODO{where} for details). 

%\paragraph{RAM Delegation} For a polynomial-time Turing machine $M$, a \emph{RAM Delegation Scheme} for $M$ allows a verifier that does not have full access to an input $x$ (for instance, because $x$ is too long) and holds an alleged output $y$, to verify that $M$ was executed correctly on $x$, and in particular, the output of that execution is indeed $y$.
%In general, a RAM Machine is a deterministic Turing machine that has random access to memory that is much longer (mostly, exponentially longer) than its local state, and a RAM SNARG is a SNARG that proves that a RAM machine indeed outputs a certain output, without having the verifier simulate the entire execution (that requires access to a long memory). A RAM SNARG is associated with a \emph{digest} algorithm, that processes the long input into a much shorter string that the verifier can read.
%The soundness requirement of a RAM SNARG is that no efficient adversary would be able to produce a digest $d$ and two different proofs, $\pi_0$ and $\pi_1$, such that the verifier is convinced by $\pi_0$ that $M$ outputs $0$ and by $\pi_1$ that $M$ outputs $0$.
%The soundness of the Flexible RAM SNARG is defined as follows.\footnote{
%In \cite{cryptoeprint:2022/1320}, it is shown that this soundness notion can be replaced by a different one, called \emph{Partial Input Soundness}. We do not require it.
%}
%A RAM SNARG $(\Gen, \Ver, \Pro)$ for a machine $M$ that has a binary input is sound if for any efficient adversarial prover $\Pro^*$ and a polynomial $T = T(\lambda)$, there exists a negligible function $\epsilon(\cdot)$, such that
%\begin{gather*}
%    \prob{}
%    {
%    \begin{array}{ll}
%    \Ver(\var{\crs}, d, 0, \pi_0) = 1 \\
%    \wedge \Ver(\crs, d, 1, \pi_1) = 1
%    \end{array}
%    \middle\vert
%    \begin{array}{ll}
%    \var{\crs} \leftarrow \Gen(1^\secpar, T) \\
%    (d, x, \pi_0, \pi_1) \leftarrow \Pro^*(\var{\crs}) \\
%    \end{array}
%    } \leq \epsilon(\secpar)
%\end{gather*}
%That is, a RAM SNARG is sound if for any efficient adversary, it cannot produce a proof that $M$ outputs
%In \cite{cryptoeprint:2022/1320}, this definition is extended to \emph{Flexible SNARGs for RAM}, which is a RAM SNARG where the digest can be implemented by any hash family, and the SNARG is sound if that hash family has local openings.

%We use Flexible RAM SNARGs to construct a succinct distributed argument for $\PP$. As mentioned, such RAM SNARGs are defined w.r.t some hash family with local opening. For our use, that hash family will be a succinct vector commitment, which already satisfies all of the hash family with local openings requirements. In addition, we require that the vector commitment has %a property we call \emph{inverse collision-resistance}. A VC is Inverse Collision-Resistant (ICR)\TODO{?} any efficient adversary cannot find a false commitment to a vector, cannot open two different commitments to the same vector in \emph{every} index, such that the verifier is convinced by all openings.
%the following property.\footnote{
%A succinct, inverse collision-resistant VC can be instantiated by a Merkle Tree \cite{merkle1989certified}. See Appendix~\ref{} \TODO{where} for more details.
%}
%\begin{definition}[Inverse Collision-Resistance]
%A VC $(\Gen, \Com, \Open, \Ver)$ is \emph{Inverse Collision-Resistant} if for any efficient adversary $\calA$, there exists a negligible function $\epsilon(\cdot)$, such that for every $\lambda\in \mathbb{N}$,
%\begin{gather*}
    %\prob{}
    %{
    %\begin{array}{ll}
         %\forall i: \Ver(crs, C^*, m, i) = 1 \\
         %\wedge C^* \neq C
    %\end{array}
    %\middle\vert
    %\begin{array}{ll}
         %crs \leftarrow Gen(1^\lambda, q) \\
         %C^*, \{(m_i, \lambda_i)\}_{i\in [q]} = \calA(crs) \\
         %C \leftarrow \Com(crs, m_1,\ldots,m_q)
    %\end{array}
    %} \leq \epsilon(\lambda)
%\end{gather*}
%\end{definition}

%Let $\Lan$ be a language on graphs that is decidable in polynomial time, given the entire graph as input, and let $M_\Lan$ be the Turing machine that decides it:
%$G\in \Lan \Leftrightarrow M_\Lan(L(G)) = 1$. 
%The idea of our construction is similar to the one in the last section: the prover commits to the graph with the VC, and sends the commitment $c$ with a RAM SNARG proof $\pi$. The main difference is that in the construction for $\NP$, the prover proved that \emph{there exists} a graph $\Tilde{G}$ such that $\Tilde{G}$ is in the language and $c$ is a valid commitment to it, while 
%here, the prover proves that when $M$ accepts $G$. This alone would not be sound; since the VC is succinct, that is, it compresses the input, it might be very likely that there is a graph $\Tilde{G}$ such that $c$ is a valid commitment to it and $\Tilde{G}\in \Lan$. In the construction for $\PP$, the argument of knowledge property of the SNARK guaranteed that if a prover $\Pro^*$ produces an accepted commitment $c$, then it must \emph{know} (that is, we can extract from it) a witness, which includes $G$, and since it is not likely that the prover would know a pre-image of $c$ other then $G$, our argument was sound.
%The RAM SNARG does not have the argument of knowledge property, but instead, the Inverse Collision-Resistance property of the VC guarentees that if $c$ opens to $G$ everywhere, then it is very unlikely that committing to $G$ honestly produces a value $c'\neq c$. If $c$ is indeed the honest commitment to $G$, and $M$ rejects $G$, by the completeness of the SNARG, we can construct a proof $\pi_0$ that is convincing that $M(G)=0$. Since $\Pro^*$ produces $\pi_1$ that convinces the verifier that $M(G)=1$, this breaks the SNARG.

%\section{Certifying Executions of Computationally-Efficient Distributed Algorithms}
\label{sec:distprover}
In this section we construct a succinct distributed argument
for certifying the execution of polynomial-time distributed algorithm,
where the prover is itself distributed;
essentially, the distributed algorithm certifies its own execution,
using an additional $O(\diam(G))$ rounds.


Fix a distributed algorithm, represented by a deterministic Turing machine $D$ that
executes at every node.
%takes as input a node's input and neighborhood,
%writes outgoing messages on an output tape, reads incoming messages from an input tape, and eventually
%produces some output.
The distributed language we would like to certify is the language
$\calL_D$
consisting of all configurations $(G, x, y)$,
where $G$ is the network graph, $x : V(G) \rightarrow \calX$ is an input assignment to the nodes,
and $y : V(G) \rightarrow \calY$ is the output stored at the nodes,
such that when $D$ is executed at every node of $G$ with input assignment $x$,
each node $v \in V(G)$ produces the output $y(v)$.
We construct a distributed prover for the statement ``$(G,x,y) \in \calL_D$''.

To simplify the presentation, we assume here that there is no input $x$,
and that $D$ is a \emph{decision} algorithm,
so that the output we want to certify is $y(v) = 1$ at all nodes.
(See Appendix~\ref{app:distprover} for the general case.)

\paragraph{Overview of the construction.}
Certifying the execution of the distributed algorithm $D$
essentially amounts to certifying a collection of ``local'' statements,
each asserting that at a specific node $v \in V(G)$,
the algorithm $D$ indeed produces the output $y(v) = 1$.
The prover at node $v$ can record the local computation at node $v$
as $D$ executes,
and use a SNARG or a SNARK
to certify that it is correct: for example, it can certify that incoming messages
are handled correctly (according to $D$),
outgoing messages are produced correctly, and eventually, the output of $v$ is indeed $y(v) = 1$.
The main obstacle is verifying consistency across the nodes:
how can we be sure that incoming messages recorded at node $v$ were indeed sent by $v$'s neighbors,
and that $v$'s outgoing messages are indeed received by $v$'s neighbors?

A na\"ive solution would be for node $v$ to record, for each neighbor $u \in N(v)$,
a hash $H_{(v,u)}$ of all the messages that $v$ sends and receives on the edge $\set{v,u}$;
on the other side of the edge, node $u$ would do the same, producing a hash $H_{(u,v)}$.
At verification time, nodes $u$ and $v$ could compare their hashes, and reject if $H_{(v,u)} \neq H_{(u,v)}$.
Unfortunately, this solution would require too much space,
as node $v$ can have up to $n - 1$ neighbors;
we cannot afford to store a separate hash for each edge.

Our solution is instead to compute a hash $h(m)$
for every message $m$ sent or received by node $v$,
but store only a \emph{sum} of the hashes:
we separate outgoing messages from incoming messages,
and store two sums,
$s_{\var{out}}(v) = \sum_{\text{outgoing $m$}} h(m)$
and
$s_{\var{in}}(v) = \sum_{\text{incoming $m$}} h(m)$.
To certify consistency across all the nodes,
we compute a spanning tree of the network,
and store at every tree node $u$ the partial sums
in its subtree,
\begin{equation*}
	S_{\var{out}}(u) = \sum_{v \in \text{$u$'s subtree}} s_{\var{out}}(v),
	\qquad
	S_{\var{in}}(u) = \sum_{v \in \text{$u$'s subtree}} s_{\var{in}}(v).
\end{equation*}
In particular, at the root $r$ of the tree, we store the full sums:
%\begin{equation*}
	$S_{\var{out}}(r) = \sum_{v \in V(G)} s_{\var{out}}(v)$
	and
	%\qquad
	$S_{\var{in}}(r) = \sum_{v \in V(G)} s_{\var{in}}(v)$.
%\end{equation*}
The root then compares the two sums, and verifies that they are equal,
which means that every message sent was indeed received, and vice-versa.


Since we compare \emph{sums} of hashed values rather than
directly comparing hashed values to one another,
our construction requires the following property,
which we call \emph{sum-collision-resistance};
it is a plausible strengthening of standard collision-resistance (see discussion in Appendix~\ref{app:crypto:hashes}).
%\TODO{add + subsection of appendix}

\begin{definition} [Sum-Collision-Resistant Hash (SCRH)]
    A hash family $(\Gen, \Hash)$ is considered \emph{sum-collision-resistant} if for any probabilistic poly-time adversary $\mathcal{A}$, there exists a negligible function $\epsilon(\cdot)$, such that for every $\secpar\in \mathbb{N}$,
    \begin{gather*}
        \prob{}
        {
        \begin{array}{cc}
        M \neq M' \\
        \sum_{ m \in M }\Hash(\hk, m) = \sum_{ m' \in M'}\Hash(\hk, m')
        \end{array}
        \middle\vert
        \begin{array}{ll}
             hk \leftarrow Gen(1^n, 1^\secpar) \\
             (M, M') \leftarrow\calA(\hk, 1^\secpar, n)
        \end{array}
        } \leq \epsilon(\secpar)
    \end{gather*}
\end{definition}

%\begin{theorem}[Informal]
    %Let $\calD$ be a distributed algorithm that runs in $T = \poly(n)$ rounds
    %and sends messages of length $\poly(n)$.
    %Assuming SNARKs for $\NP$ exist,
    %and assuming a SCRH exist,
    %there is a distributed argument of length $\polylog(n)$
    %certifying $D$'s execution, where the prover is a distributed algorithm
    %running in $O(T + \diam(G))$ rounds and sending messages of $\polylog(n)$ bits.
%\end{theorem}

\paragraph{Detailed description of the construction.}
In the sequel, fix an SCRH, $(\SCRHGen, \SCRHHash)$.

We represent a message by $\var{msg} = (r, \set{u, v}, m)$,
where $r \in \nat$ is the round number in which the message was sent,
$\set{u,v} \in E$ is the edge on which the message was sent,
and $m \in \set{0,1}^*$ is the contents of the message.
It is important that this representation of a message does not indicate whether the message was sent by $u$ and received by $v$ or vice-versa,
as our construction relies on hashing messages and verifying that every (hashed) incoming message has a corresponding (hashed) outgoing message.

The consistency of the local computation at a specific node is captured
by a
language $\calD$,
which consists of all triplets $(\hk, I(v), W(v))$ such that:
\begin{itemize}
	\item $\hk$ is a hash key obtained by calling $\SCRHGen$,
	\item $I(v) = (v, N(v), s_{\var{in}}(v), s_{\var{out}}(v))$,
		where
		$v \in \calU$ is the UID of a node,
		$N(v) \in \calU^{\ast}$ is the neighborhood of the node,
		and $s_{\var{in}}(v), s_{\var{out}}(v)$ are hash sums;
	\item $W(v) = (\var{msgout}(v), \var{msgin}(v))$ consists of two sets of messages;
	\item $(\hk, I(v), W(v)) \in \calD$ iff when the distributed algorithm $D$ is executed 
at a node with UID $v$ and neighbors $N(v)$,
and the incoming messages at node $v$ are $\var{msgin}(v)$,
then node $v$ sends the messages $\var{msgout}(v)$
and accepts (i.e., outputs 1),
and furthermore,
\begin{equation}
	s_{\var{in}} = \sum_{\var{msg} \in \var{msgin}} \SCRHHash(\hk, \var{msg}),
	\qquad
	s_{\var{out}} = \sum_{\var{msg} \in \var{msgout}} \SCRHHash(\hk, \var{msg}).
\end{equation}
\end{itemize}
We think of $W(v) = (\var{msgout}(v), \var{msgin}(v))$ as a \emph{witness}
to the correct computation at node $v$.
%We refer to $s_{\var{in}}$ as the \emph{hash-sum of incoming messages}, and to $s_{\var{out}}$ as the \emph{hash-sum of outgoing messages}
%at $v$.

Since the algorithm $D$ is itself a polynomial-time Turing machine,
and the SCRH can be computed in polynomial time,
there is a polynomial-time Turing machine $M$ that decides the language $\calD$.
Fix a SNARK $(\SNARKGen, \SNARKPro, \SNARKVer, \SNARKExt)$
for the language
of pairs $(\hk, I)$ satisfying
$\exists W = (\var{msgout}, \var{msgin}) : \text{$M$ accepts $(\hk, I, W)$}$.

The distributed prover at each node $v$ computes the following certificate $\pi(v)$:
\begin{itemize}
	\item The hash-sums $s_{\var{out}}(v), s_{\var{in}}(v)$.
	\item A SNARK proof $\pisnark(v)$, certifying that there exists a witness $W(v) = (\var{msgout}(v), \var{msgin}(v))$
		such that $(\hk, I(v), W(v)) \in \calD$.
\end{itemize}
In addition, the distributed prover computes a spanning tree of the network in $O(\diam(G))$
rounds, and stores at each node $v$ the parent $p(v)$ of $v$ (or $\bot$, if $v$ is the root),
and a spanning-tree certificate~\cite{korman2005proof} consisting
of the UID of the root and the distance of $v$ from the root.
Finally, by convergecast up the tree,
the distributed prover computes and stores at $v$
the partial sums 
\begin{equation*}
	S_{\var{out}}(v) = s_{\var{out}}(v) + \sum_{u \in \mathrm{children}(v)} S_{\var{out}}(u),
	\qquad
	S_{\var{in}}(v) \leftarrow s_{\var{in}}(v) + \sum_{u \in \mathrm{children}(v)} S_{\var{in}}(u).
\end{equation*}

The nodes then verify the spanning tree
and the SNARK proof,
and
make sure the partial sums agree with their children's partial sums.
The root $r$ verifies that $S_{\var{out}}(r) = S_{\var{in}}(r)$.

\section{Distributed Merkle Trees}
\label{sec:distmerkle}
In this section we define our notion of a distributed Merkle tree and show how to construct it from collision-resistant hash functions.

Distributed Merkle trees (DMTs) are regular Merkle trees, that hold information of a distributed network, in a way that on one hand allows nodes to compute the Merkle root jointly and communication-wise efficiently, such that in the end of the computation each node also has a succinct opening from the Merkle root to the information it held originally. In the next section, we'll use DMTs as the $\MT$ family for the distributed SNARG construction.

A $t(G,x)$-efficient distributed Merkle tree, is associated with a recursively constructable hash family with local openings collision-resistant hash family $\MT = (\MT.\Gen, \MT.\Hash, \MT.\Open, \MT.\Verify)$, and consists of the following algorithms.

\begin{itemize}
    \item $\Gen(1^\secpar, N)\to \hk$: a randomized algorithm that takes as input the security parameter $\secpar$ and the total size of the graph configuration, and outputs a hash key $\hk = \MT.\Gen(1^\secpar, N)$
    \item $\DistConstruct(\hk, G, \{\{x_{v,u}\}_{u\in N(v)}\}_{v\in V(G)})\rightarrow \val, \{\{(\rt_{v,u}, h_{v,u}, I_{v,u}, \rho_{v,u})\}_{u\in N(v)}\}_{v\in V(G)}$ A distributed algorithm, that when running on a network $G$, with all nodes receiving the same hash key $\hk$, and each node $v$ receives for each neighbor $u\in N(v)$ an input $x_{v,u}$, outputs $\val$ at all nodes, and at each node $v$ an $\MT$ root, height, index and an opening path $\rho_u$ for every neighbor of $v$, $u\in N(v)$.
    \item $\DistVer(\hk, G, \val, \{(\rt_v, I_v, \rho_v)\}_{v\in V(G)}) \rightarrow \{b_v\}_{v\in V(G)}$ : a distributed algorithm, that when running on a network $G$, with all nodes receiving the same hash key $\hk$, and each node $v$ receives an $\MT$ root $\rt_v$, index $I_v$, and opening path $I_v$, outputs an acceptence bit at each node.
    \item $\Verify(\hk, \val, I, b, \rho) \in \{0,1\}$: a verification algorithm that takes a hash key $\hk$, a value $\val$, an index $I$, a bit $b$ and an opening $\rho$, and outputs an acceptance bit $a = \MT.\Verify(\hk, \val, (v,u), b, \rho)$.
    %\item $\Verify(\hk, \val, (v,u), b, \rho) \in \{0,1\}$: a verification algorithm that takes a hash key $\hk$, a value $\val$, a node $u$, a bit $b$ and an opening $\rho$, and outputs an acceptance bit $a = \MT.\Verify(\hk, \val, (v,u), b, \rho)$, where $(v,u)$ is referred to as an index.
\end{itemize}

\begin{definition}[Properties of $\DMT$]
    For a graph network $G$, we define the following centralized algorithms:
    \begin{itemize}
        \item $\DistHash_G(\hk, \{\{x_{v,u}\}_{u\in N(v)}\}_{v\in V(G)}) \rightarrow \val$: the algorithm that on input $(\hk, \\ \{\{x_{v,u}\}_{u\in N(v)}\}_{v\in V(G)})$, simulates $\DistConstruct$ on $(\hk, G \{\{x_{v,u}\}_{u\in N(v)}\}_{v\in V(G)})$, parses the output into $\val, \{\{\rho_{v,u}\}_{u\in N(v)}\}_{v\in V(G)}$ and outputs only $\val$.
        \item $\DistOpen_G(\hk, \{\{x_{v,u}\}_{u\in N(v)}\}_{v\in V(G)}) \rightarrow x_{v,u}, \rho_{v,u}$: the algorithm that on input $(\hk, \\  \{\{x_{v,u}\}_{u\in N(v)}\}_{v\in V(G)})$, simulates $\DistConstruct$ on $(\hk, G \{\{x_{v,u}\}_{u\in N(v)}\}_{v\in V(G)})$, parses the output into $\val, \{\{\rho_{v,u}\}_{u\in N(v)}\}_{v\in V(G)}$ and outputs only $x_{v,u}, \rho_{v,u}$.
    \end{itemize}
    $(\Gen, \DistConstruct, \DistVer, \Verify)$ is a $\DMT$ if for every network $G$, $(\Gen, \DistHash_G, \DistOpen_G, \Verify)$ is a recursively constractable hash family with local openings, and $\DistConstruct$ runs in \TODO{Communication efficiency?}
    \TODO{A soundness condition on distributed verification}
\end{definition}

\subsection{Construction from $\MT$s}

Let $\MT = (\MT.\Gen, \MT.\Hash, \MT.\Open, \MT.\Verify)$ be a recursively constructable hash family with local openings. The algorithms $\DMT.\Gen, \DMT.\Verify$ are simply $\MT.\Gen, \MT.\Verify$, respectively. We now describe the algorithm $\DMT.\DistConstruct$.

On input $(\hk, G \{x((u,v), x(v,u))\}_{(v,u)\in E(G)})$, the distributed algorithm $\DistConstruct$ executes the following stages:


\subsubsection{The algorithm $\DistConstruct$}

\paragraph{Stage 1: Inner Merkle tree. } Each node $v$ computes $\rt_v = \MT.\Hash(\hk, \{x_{u,v}\}_{u\in N(v)})$.

\paragraph{Stage 2: spanning tree. } The network jointly computes a spanning tree of itself, $\var{ST}(G)$, and certifies it. Meaning, at the end of the computation, each node $v$ holds $(\var(root), p(v), d(v), C(v))$, where $\var{root}\in V(G)$ is the UID of the root of the spanning tree, $p(v)$ is $v$'s parent in the spanning tree ($p(v)\in N(v)$), or $\bot$ if $v=\var{root}$, $d(V)$ is the distance between the root and $v$ in the spanning tree ($0 \leq d(v) \leq n$), and $C(v)\subseteq N(v)$ is the set of $v$'s children in the spanning tree (empty for spanning tree leaves). This is done as in \cite{korman2005proof}. \TODO{Check details: is this robust for an anonymous network where $n$ is not known? Does leaves now they're leaves? }

\paragraph{Stage 3: converge-cast of Merkle forests. } In this stage, each node computes a Merkle forest while maintaining the forest structure. The root of the spanning tree computes the hash value.
\begin{enumerate}
    \item Set $S_v = (\rt_v, 0) \cup \bigcup_{u\in C(v)}F_u$.
    \item Set $F_v = S_v$, $h = 0$. While $h \leq \max_{(\rt, h)\in S_v} h$:
    \begin{enumerate}
        \item If $|\{(\rt_c, h_c)\in F_v ~|~ h_c = h\}| \geq 2$:
        \begin{enumerate}
            \item Take two such tuples: $(rt_0, h) \neq (rt_1, h) \in F_v$,
            \item Set $\rt = \calH(\hk, \rt_0 || \rt_1)$,
            \item Set in $S_v$, $\rt_0.p = \rt$, $\rt_0.s = 0$
            \item Set in $S_v$, $\rt_1.p = \rt$, $\rt_1.s = 1$
            \item Set in $S_v$, $\rt.l = \rt_0$,
            \item Set in $S_v$, $\rt.r = \rt_1$,
            \item Update $F_v = (F_v \backslash \{(rt_1, h), (rt_2, h)\}) \cup (rt, h+1)$.
        \end{enumerate}
        \item Else, 
        \begin{enumerate}
            \item If $v = \var{root}$ and $|\{(\rt_c, h_c)\in F_v ~|~ h_c = h\}| = 1$,
            \item Else, increment $h = h+1$
        \end{enumerate}
    \end{enumerate}
\end{enumerate}
After this, $F_{\var{root}}$ contains only one tuple: $(\rt, h)$. Set $\val = \rt$.

\paragraph{Stage 4: broadcast of Merkle openings}
Starting from $\var{root}$, each node $v$, upon receiving input $(\val, \alpha_v)$, from its spanning tree parent $p(v)$, where $\alpha_v = \{(\rt_i, h_i, I_i, \rho_i)| (\rt_i, h_i) \in F_v\}$, does the following:
\begin{enumerate}
    \item For every spanning tree child of $v$, $c\in C(v)$:
    \begin{enumerate}
        \item Set $\alpha_c = \phi$, $I^F_c = \phi$
        \item For every Merkle tuple $(\rt_a, h_a)\in F_c$:
        \begin{enumerate}
            \item Get $(\rt_a, h_a)$ in $S_v$, \ //Where the side and parent is documented
            \item Set $\rho_a = I_a = \epsilon$ \ //The empty string
            \item Set $\rt = \rt_a$. While $\rt.p \notin F_v$, do:
            \begin{enumerate}
                \item If $\rt.s = 0$: update $\rho_a = \rho_a || (\rt.p).l$
                \item If $\rt.s = 1$: update $\rho_a = \rho_a || (\rt.p).r$
                \item Update $I_a = \rt.s || I_a$
                \item Update $\rt = \rt.p$
            \end{enumerate}
            \item Get $\rho, I$ such that $(\rt.p, h, I, \rho)\in \alpha_v$ for some $h$.
            \item Set $\rho_a = \rho_a || \rho$
            \item Set $I_a = I || I_a$
            \item Update $\alpha_c = \alpha_c \cup \{(\rt_a, h_a, I_a, \rho_a)\}$
        \end{enumerate}
        \item Send $\alpha_c$ to $c$
    \end{enumerate}
\end{enumerate}

\subsubsection{The algorithm $\DistVer$}

\paragraph{}
\TODO{Formalize and extend}
Each node $v$ has $\hk, \val, \rt_v, p(v), C(v), F^I_V, I_v, \rho_v$. Upon receiving from each child $c\in C(v)$, $F^I_c$, it runs stage 3 of the algorithm $\DistConstruct$, obtains the Merkle forest $F'_v$, and verifies that $F'_v = F_v$. Then, since (unlike in $\DistConstruct$) it already has $I$ for every $rt\in F^I_v$, it also verifies that the indices in $F^I_c$ for every $c\in C(v)$ are correct.
%\section{On Polynomial-Time Local Distributed Algorithms}
\label{sec:local}
\TODO{refer to appendix: is this the right version??}
In this section we investigate the power of computationally-bounded local decision algorithms:
we define complexity classes for languages decidable by 
such algorithms, and study their relationship to the class of languages
that can be decided by local algorithms with unbounded local computational power,
and to the complexity class $\PP$.
On a high level, our main result is that combining the requirements for locality and computational efficiency
in one algorithm is more restrictive than requiring that the language
be decidable by one algorithm that is local,
and also by another algorithm that is computationally efficient.

\subsection{Definitions}
\label{sec:local_def}

Fix an input domain $\calX$, and a UID space $\calU$,
and let $\calC = \calC(\calX, \calU)$ be the set of all configurations
$(G, x)$
with inputs $x : V(G) \rightarrow \calX$ drawn from $\calX$,
and UIDs drawn from $\calU$.
We let $\calB^t$ be the set of all $t$-neighborhoods that appear in $\calC$:
	$\calB^t = \set{ N_{G,x}^t(v) : (G,x) \in \calC, v \in V(G) }$,
where $N_{G,x}^t(v)$ denotes the $t$-neighborhood of $v$,
including the UIDs and the inputs of the nodes in the $t$-neighborhood.

We model a \emph{$t$-local decision algorithm} as a mapping $\A : \calB^t \rightarrow \set{0,1}$,
which outputs a Boolean value (accept / reject).
We require that $\A$ be a computable function.
As usual, a configuration is accepted by $\A$ iff when $\A$ is executed at every node,
it outputs 1 everywhere.

\begin{definition}[The classes $\LD$, $\LDP$]
	A distributed language $\calL$ is in the class $\LD(t(\cdot))$ 
	if it can be decided in graphs of size $n$ by a $t(n)$-local decision algorithm $\A$.
	If in addition the algorithm $\A$ can be computed by a Turing machine
	that runs in time $\poly(n)$ in graphs of size $n$,
	then $\calL$ is in the class $\LDP(t(\cdot))$.
    %Let $\Lan$ be a distributed language.
    %\paragraph{$t$-Local Decidability} $\Lan$ is in the class $\LD(t)$ if there exists a distributed algorithm $A$, that runs in $t$ rounds, and decides $\Lan$.
    %\paragraph{$t$-Local Polynomial Time Decidability} $\Lan$ is in the class $\LDP(t)$ if there exists a distributed algorithm $A$ and a polynomial $p$, such that $A$ runs in $t$ rounds, and performs up to $p(n)$ steps of local computation, and decides $\Lan$.
\end{definition}

We are interested in algorithms that run in a sublinear number of rounds:
let $\LD = \bigcup_{t(\cdot) \in o(n)} \LD(t(\cdot))$,
and let $\LDP = \bigcup_{t(\cdot) \in o(n)} \LDP(t(\cdot))$.
Note that, as usual in the area of local decision, the local algorithm may not know the size $n$
of the network;
nevertheless, as external observers,
we can study the dependence of the algorithm's locality radius and its local running time on $n$.

%Since we only refer here to \emph{sub-linear} number of rounds, let $\LD$ denote $\bigcup_{t\in o(n)}\LD(t)$, and let $\LDP$ denote $\bigcup_{t\in o(n)}\LDP(t)$.


%, we attend to the matter of computationally-bounded local decision. We define complexity classes for polynomial-time sub-linear-rounds decision and show in Section~\ref{sec:uncond} that it is separated from the intersection between the class of sub-linear rounds distributed decision and centralized polynomial-time decision ($\PP$). When nodes are given the size of the graph, this separation becomes much harder to prove, as demonstrated in Appendix~\ref{app:local}\TODO{where}, proving it unconditionally would be proving that $\PP\neq\NP$. In Section~\ref{sec:owf}, we prove the  separation in the case where nodes know the size of the graph conditioned on the existence of injective one-way functions. In Appendix~\ref{app:NLD}\TODO{where}, we show that the distinction disappears when non-determinism is introduced, and non-deterministic polynomial-time local decision in fact equals in power to the intersection of non-deterministic local decision and $\NP$.

\subsection{Unconditional Separation of $\LDP$ from $\LD \cap \PP$}\label{sec:uncond}
By definition we have $\LDP \subseteq \LD$,
as every $\LDP$-algorithm is also an $\LD$-algorithm.
It is also easy to see that $\LDP \subseteq \PP$:
if every node of the network computes its decision in $\poly(n)$ time,
then a poly-time centralized Turing machine can simulate the local algorithm
at every node, and accept iff all nodes accept.
Together we have that $\LDP \subseteq \LD \cap \PP$.
Our first result shows that the containment is strict.

\paragraph{High-level overview.}
To separate $\LDP$ from $\LD \cap \PP$,
we use a variation on the language $\ITER$, which was used in \cite{balliu2018can} to separate $\Pioloc$ from $\LD$.
We call our variation $\ITERIN$.

The idea is to construct a language of paths,
where the center node is given a Turing machine $M$, two inputs $a,b \in \set{0,1}^*$,
and a bound $s$;
the goal is to decide whether $M$ halts on both $a$ and $b$ within at most $s$
computation steps, and accepts either $a$ or $b$ (or both).
The bound $s$ may be much larger than the length of the input (it is encoded in binary),
so an efficient algorithm cannot afford to run $M$ for $s$ steps and check
whether it accepts $a$ or $b$,
but a local algorithm with unbounded computation time can do so,
and therefore $\ITERIN \in \LD$.
To make the task solvable for a polynomial-time centralized Turing machine,
we add additional annotations (in the form of inputs to the nodes):
on the left side of the path, from the center outwards, we write 
the sequence of configurations that $M$ goes through in its computation on $a$,
until it halts;
on the right side of the path we do the same for $b$.
This makes it possible for a poly-time Turing machine to simply examine
the computation sequence of $M$, make sure it is legal (i.e., it matches the transition function of $M$),
and verify that at either the left or the right side of the path (or both)
we have
an accepting configuration of $M$.
Thus, $\ITERIN \in \PP$.

Finally, we prove that an algorithm that is both local and efficient cannot
decide the language $\ITERIN$: intuitively, this is because 
it can neither afford to run $M$ for $s$ steps, nor can it ``see''
the endpoints of the path to verify that at least one of them has an accepting configuration.
The formal proof shows that if there existed a $\PLD$-algorithm for $\ITERIN$
then we could use it to decide in polynomial time a language
that is not in $\PP$.


\paragraph{Detailed construction.}
Let $M$ be a Turing machine,
and let $a,b \in \set{0,1}^*$ be strings such that $M$ halts on input $a$ and on input $b$.
We define a configuration $C^{n_L, n_R}({M,a,b, s}) = (G, x)$, as follows:
\begin{itemize}
	\item $G$ is a path of the form $u_{n_L},\ldots,u_1,v,w_1,\ldots,w_{n_R}$,
		consisting of a \emph{pivot node} $v \in V(G)$,
		a left sub-path $L = u_{n_L},\ldots,u_{1}$,
		and a right sub-path $R = u_1,\ldots,u_{n_R}$.
	\item The input of the pivot node $v$ is $x(v) = (0, \langle M \rangle, a, b, s )$,
		where $\langle M \rangle$ is the encoding of the Turing machine $M$.
	\item For each node $u_i \in L$ on the left sub-path, the input of $u_i$
		is given by $u_i = (i, \langle M \rangle, M_{a,i} )$,
		where $M_{a,i}$ is the configuration of $M$ after $i$ steps
		executing with input $a$
		(recall that the configuration of a Turing machine
		consists of the contents of the tape, the location of the tape head,
		and the current state).
		To avoid the clash in terminology, we refer to configurations of Turing machines
		as \emph{TM-configurations}.
	\item Similarly, for each node $w_i \in R$ on the right sub-path,
		we have $x(w_i) = (i, \langle M \rangle, M_{b,i} )$.
\end{itemize}

We simplify the notation somewhat by writing $C^n(M,a,b,s) = C^{n,n}(M,a,b,s)$,
and $C(M,a,b,s) = C^s(M,a,b,s)$.
Given a configuration $C^{n_L, n_R}({M,a,b,s}) = (G, x)$ as defined above,
we say that a node $u \in V(G)$ is \emph{$r$-central} if the distance of $u$
from the pivot is at most $r$.

The language $\ITERIN$ consists of all configuration $C^{n_L,n_R}(M,a,b,s)$ such that
	the TM-configurations written at the end of both sub-paths are both halting,
	$s \geq \max(n_L, n_R)$,
	and $M$ accepts $a$ or $b$ (or both).

	As we explained above, it is not difficult to see that $\ITERIN$ can be decided by a local algorithm,
and is also in $\PP$:
\begin{claim}\label{claimITINalgos}
	$\ITERIN \in \LD \cap \PP$.
\end{claim}
%\begin{proof}[Proof sketch]
%The key is that $\ITERIN$ contains a "trap door" for each type of algorithm:
%the sequential poly-time algorithm checks that $s \geq \max(n_L, n_R)$,
%but does not run $M$ for $s$ steps (as $s$ may not be polynomial in the input length),
%instead verifying that the TM-configurations evolve according to the transition function of $M$,
%that the TM-configurations at the ends of the sub-paths are halting, and that at least one accepts.
%On the other hand, the $\LD$-algorithm cannot directly check that at least one of the endpoints of the sub-paths
%contains an accepting TM-configuration,
%but the pivot,
%which is not polynomially-bounded,
%can afford to run $M$ for $s$ steps and verify that it accepts one of the inputs.
%\end{proof}

%\begin{proof}[Proof of Claim~\ref{claimITINalgos}]
	%To decide membership in $\ITERIN$ using a $1$-local algorithm:
	%\begin{itemize}
		%\item Each node verifies that its index is consecutive with its neighbors' indices,
			%and that the TM-configuration in its input is indeed the successor
			%to the TM-configuration in the input of its neighbor with a preceding index
			%(unless its index is 0).
		%\item The endpoints of the path verify that they have a halting TM-configuration,
			%and that $s$ is no smaller than their index.
		%\item The pivot runs the TM $M$ on $a$ and on $b$ for $s$ steps each,
			%and verifies that at least one of $a$ and $b$ is accepted.
			%Note that since $s \geq \max(n_L, n_R)$,
			%if both endpoints accept,
			%then $M$ halts in at most $s$ steps on $a$ and on $b$,
			%as both endpoints check that they have a halting TM-configuration,
			%and because the nodes along the way
			%verify the transitions of the TM,
			%the TM-configurations at the endpoints evolve from the initial configuration
			%in at most $n_L$ or $n_R$ steps, respectively.
	%\end{itemize}
	%To decide membership in $\ITERIN$ using a poly-time centralized algorithm:
	%we verify that the input is structured correctly (that the graph is a path,
	%the nodes' indices
	%are correct, the pivot is given correctly-formatted input,
	%and $s$ is at least the length of both sub-paths),
	%and in particular, 
	%that the configurations of the TM on each sub-path evolve according to the transition
	%function of $M$.
	%Examining the endpoints of the sub-paths,
	%we make sure that at least one of them has an accepting configuration of the TM.
%\end{proof}

Next we show that $\ITERIN$ is not decidable by a polynomial-time local algorithm:

\begin{claim}\label{claimITINLDP}
	$\ITERIN \not \in \LDP$.
\end{claim}
\begin{proof}
Suppose for the sake of contradiction that there is a $\LDP$-algorithm $\A$ that decides $\ITERIN$,
and let $t > 0 $ be its locality radius.
Let $\calL \in \DTIME(2^{n}) \setminus \PP$ be some language that is Turing-decidable
in time $O(2^{n})$ but not in polynomial time,
and such that $\eps \not \in \calL$ (here and in the sequel, $\eps$ denotes the empty word).
Such a language exists by the Time Hierarchy Theorem~\cite{hartmanis1965computational}.
We claim that using the $\LDP$-algorithm $\A$ that decides $\ITERIN$, we can construct a polynomial-time Turing machine
that decides $\calL$, a contradiction.

Let $M$ be a $\DTIME(2^{n})$-time Turing machine that decides $\calL$,
and let $f \in O(2^n)$ be a function bounding the running time of $M$ on inputs of length $n$.
Given input $z \in \set{0,1}^*$,
let $C_z = C(M,\eps,z,f(|z|))$
be the configuration that encodes the runs of $M$ on $\eps$ (on the left sub-path) and on $z$
(on the right sub-path)
until $M$ halts, using sub-paths of length $f(|z|)$.
Since we assume that $\eps \not \in \calL$,
we have $C_z \in \ITERIN$ iff $z \in \calL$.

We define a poly-time Turing machine $M'$ for $\calL$ as follows:
on input $z \in \set{0,1}^*$,
$M'$ constructs the configuration $C_z' \coloneq C^{2t}(M, \eps, z, f(|z|))$,
which is essentially the central portion of $C_z$, including only $2t$ 
nodes to the left and to the right of the pivot (a total of $4t+1$ nodes).
Next, $M'$ simulates the local algorithm $\A$
on all the nodes of $C_z'$.
Finally, $M'$ accepts iff $\A$ outputs 1 at all $t$-central nodes of $C_z'$
(ignoring the outputs of the other nodes).

It is not difficult to verify that the running time of $M'$ is polynomial in $|z|$,
in the description length of $M$ (which is constant),
and in $t = o(n)$.
To show that $M'$ indeed decides $\calL$,
suppose first that $z \in \calL$.
Then $C_z \in \ITERIN$ by construction,
and therefore $\A$ must output 1 at all nodes of $C_z$.
But this means that all $t$-central nodes in $C_z'$ must also accept:
for each $t$-central node $u$ in $C_z'$,
the $t$-local view of $u$ is the same in $C_z$ and in $C_z'$,
because $C_z'$ is obtained from $C_z$ by removing only nodes at distance greater than $t$ from $u$.
Since the output of $u$ depends only on its $t$-local view,
and we know that $u$ accepts in $C_z$, it must also accept in $C_z'$
Thus, $M'$ accepts $z$.

Now suppose that $z \not \in \calL$.
In this case, $C_z \not \in \ITERIN$,
because in $C_z$ the two inputs encoded in $x$ are both rejected by $M$ (as $\eps, z \not \in \calL$).
We claim that at least one $t$-central node of $C_z$ must reject;
as above, this means that the same node also rejects in $C_z'$,
causing $M'$ to reject $z$.

Suppose for the sake of contradiction that all $t$-central nodes of $C_z$ accept.
However, since $C_z \not \in \ITERIN$, we know that some node of $C_z$ rejects;
let $u$ be such a node.
The distance of $u$ from the pivot $v$ must be greater than $t$, since we assumed that no $t$-central node rejects.
Now fix some string $a \in \calL$ (which must exist, as $\emptyset \in \PP$ and we assumed $\calL \not \in \PP$),
and let $C_{a,z} = C(M,a,z,f(\max(|a|,|z|)))$
be the configuration encoding the runs of $M$ on $a$ (on the left sub-path)
and on $z$ (on the right sub-path), using paths of length $f(\max(|a|,|z|)$,
so that $M$ halts on both.
Since $a \in \calL$,
we have $C_{a,z} \in \ITERIN$,
and thus all nodes must accept $C_{a,z}$.
This includes node $u$.
However, since $u$ is at distance greater than $t$ from the pivot,
the $t$-local view of $u$ is the same in $C_{a,z}$ and in $C_z$;
thus, $u$ also accepts in $C_z$, a contradiction.
\end{proof}










%\subsection{If $\LDn \cap \PP \not \subseteq \LDnP$, Then $\PP \neq \NP$}


\subsection{Separation of $\LDnP$ From $\LDn \cap \PP$ Assuming Injective One-Way Functions}
\label{sec:owf}
In the previous section we showed that $\LD \cap \PP \not \subseteq \LDP$,
but our proof used the fact that the nodes do not know the size of the graph,
and therefore their output when the graph is a short path is the same as their output on a long path,
provided their local neighborhood stays the same.
We now ask whether the separation continues to hold if nodes do know the size of the network:
let $\LDn, \LDnP$ be variants of $\LD, \LDP$ (resp.),
where nodes receive the size $n$ of the graph as part of their input.
Is it still true that $\LDn \cap \PP \not \subseteq \LDnP$?

Perhaps surprisingly,
even though we are considering deterministic computation models,
the answer turns out to be related to whether or not $\PP = \NP$:
we prove that
$\LDn \cap \PP \not \subseteq \LDnP$ implies $\PP \neq \NP$,
and conversely, under the plausible assumption that injective one-way functions exist,%
\footnote{This is stronger than assuming that $\PP \neq \NP$,
because if $\PP = \NP$ then every function is easy to invert.}
we can still show that 
$\LDn \cap \PP \not \subseteq \LDnP$.

A \emph{one-way function family} is a family $\set{ f_n }_{n \in \nat}$,
where $f_n : \set{0,1}^n \rightarrow \set{0,1}^{m(n)}$ for some $m(n) \geq n$,
such that given an image $y \in \set{0,1}^{m(n)}$,
it is difficult to find a pre-image $x$ such that $f_n(x) = y$ (we refer to~\cite{OdedBook} for the formal definition,
as it is not needed here).
It is known that every one-way function has a \emph{hard-core predicate}~\cite{GL89},
a Boolean predicate that can be computed in poly-time from $x \in \set{0,1}^n$,
but is hard to compute given only $f_n(x)$:
\begin{definition}[Hard-core predicate]\label{def:HCP}
	A family of predicates %\\to handle the margins
 $ \set{ b_n : \set{0,1}^n \rightarrow \set{0,1}}_{n \in \nat}$
computable in poly-time
 is called a \emph{hard-core} of a
	family of functions $\set{ f_n : \set{0,1}^n \rightarrow \set{0,1}^{m(n)}}$ (where $m(n) \geq n$)
	if for every probabilistic, polynomial-time (PPT) algorithm $\Adv$, there is a negligible function $\eps(\cdot)$ such that
	for all sufficiently large $n$ we have
	%\begin{equation*}
	%	\prb{
	%		\Adv( f(z) ) = b(z)
	%	}
	%	{
	%		z \leftarrow U_n
	%	}
	%	<
	%	\frac{1}{2} + \eps(n).
	%\end{equation*}
	$\Pr\left[ \Adv( f(z) ) = b(z) \medspace | \medspace z \leftarrow U_n \right]$,
where $U_n$ denotes the uniform distribution on $\{0,1\}^n$.
\end{definition}

\begin{proof}[Proof of Theorem~\ref{thm:local}, part (3)]
	Fix a family $\calF = \set{ f_n }_{n \in \nat}$ of injective one-way functions,
	where $f_n : \set{0,1}^n \rightarrow \set{0,1}^{m(n)}$ for $m(n) \geq n$,
	and let $\set{ b_n }_{n \in \nat}$ be a family of hard-core predicates for $\calF$.
	Consider a distributed language $\calL$, which includes all configurations $C_z = (G, x_z)$ for $z \in \set{0,1}^{\ast}$,
	where $G$ is a path $v_0,\ldots,v_{n-1}$ of length $n = |z|$,
	and the input assignment $x_z$ is given by
	$x_z(v_0) = (0, b_n(z))$,
	$x_z(v_{n-1}) = (n-1, z)$,
	and $x_z(v_i) = (i, f_n(z))$ for every $0 < i < n-1$.
	We claim that $\calL \in \LDn \cap \PP$, but $\calL \not \in \LDnP$.

	To decide membership in $\calL$ using a local algorithm with unbounded computation time,
	the first node on the path can simply invert $f_n$ to compute $z$ (recall that $f_n$
	is injective),
	and then use $z$ to compute $b_n(z)$ and compare it against
	its input. In addition, the other path nodes
	need to verify that their input is locally consistent with $\calL$
	(e.g., they are indexed properly).

	To decide membership in $\calL$ using a polynomial-time centralized algorithm,
	we can simply read $z$ off of the last node on the path, compute both $f_n(z)$ and $b_n(z)$,
	and verify that the input is consistent with $f_n(z)$ and $b_n(z)$.

	%We first show that $\calL \in \LDnP \cap \PP$.
	%To decide $\calL$ using a local algorithm, it suffices for each node to verify:
	%\begin{itemize}
		%\item If $x(v) = (0, y)$, then $v$ has degree 1,
			%its neighbor $u$ has $x(u) = (1, y')$,
			%and there exists some $z \in \set{0,1}^n$ such that
			%$y = (b_n(z), f_n(z))$ and $y' = (f_n(z), \bot)$
			%(we check by brute-force search over all possible $z$).
		%\item If $x(v) = (i, y)$ where $0 < i < n - 2$,
			%then the degree of $v$ is 2,
			%and its neighbors $u, u'$ have $x(u) = (i - 1, y)$ and $x(u') = (i+1, y)$
			%(or vice-versa).
		%\item If $x(v) = (n - 2, y)$, then $v$'s degree is 2, and one of its neighbors $u$
			%has $x(u) = (n-3, y')$
			%where $y' = (z, \bot), y = (f_n(z), \bot)$ for some $z \in \set{0,1}^n$.
		%\item If $x(v) = (n-1, y)$, then $v$'s degree is 1.
	%\end{itemize}
	%To decide $\calL$ in centralized poly-time,
	%we first verify that the configuration has the correct structure (it is a path and the input has the correct form).
	%If so, let $(z, \bot)$ be the payload of the node with index $n - 1$, 
	%and let $(a, w)$ be the payload of the node with index 0.
	%We use $z$ to compute $b_n(z)$,
	%and verify that $a = b_n(z), w = f_n(z)$, and that all the nodes in-between have payload $(f_n(z), \bot)$.

	Now suppose for the sake of contradiction that $\calL \in \LDnP$,
	and let $A$ be a $t$-local efficient algorithm for $\calL$, for some $t = o(n)$.
	Then for every sufficiently large $n$, 
	we can break the hard-core predicate $b_n$ using the following adversary $\calB$:
	given input $w = f_n(z)$ for some $z \in \set{0,1}^n$,
	the adversary constructs the first $2t$ nodes of the configuration $C' = (G, x')$,
	where $G$ is a path of length $n$,
	and $x'$ is identical to $x_z$,
	except that the inpu of the first node is $(0, 0)$ (since the adversary does not know $b_n(z)$).
	Note that the adversary does not need to know $z$ for this, because $z$ is only given to the last node
	on the path, and $t < n$; the adversary only needs to know $f_n(z)$, which it is given.
	The adversary simulates the first $2t$ nodes in $C'$,
	and
	if the first $t$ nodes in the first version accept, it outputs ``0'';
	otherwise it outputs ``1''.

	We claim that our adversary correctly computes $b_n(z)$ for all $z \in \set{0,1}^n$.
	Given $w \in \set{0,1}^{m(n)}$, there is a unique $z \in \set{0,1}^n$ such that $w = f_n(z)$, because $f$ is injective.
	If $b_n(z) = 0$,
	then the $t$-neighborhood
	of each of the first $t$ nodes in the configuration $C'$ constructed by the adversary is identical
	to their view in the ``true'' configuration $C_z = (G, x_z)$.
	Since $C_z \in \calL$, all nodes must accept, and in particular the first $t$ nodes do;
	therefore the first $t$ nodes also accept in $C'$.
	Now suppose that $b_n(z) = 1$.
	Then the configuration $C'$,
	of which the adversary constructed the first $t$ nodes,
	is not in $\calL$; some node must reject in $C'$.
	Furthermore, one of the first $t$ nodes must reject in $C'$:
	suppose they do not, and let $v_j$ be some node that rejects, with $j > t$.
	In the ``true'' configuration $C_z = (G, x_z)$, the $t$-neighborhood of node $v_j$
	is the same as in $(G_n, x_z')$, because the only difference between the two configurations is the input
	of the first node, which is at distance greater than $t$ from $v_j$.
	But this means that $v_j$ also rejects in $(G_n, x_z) \in \calL$, contradicting the correctness of the local algorithm.
	We conclude that at least one of the first $t$ nodes must reject, and therefore our adversary outputs ``1''.

	We have now shown that $\calB(f_n(z)) = b_n(z)$ for all sufficiently large $n$ and $z \in \set{0,1}^n$.
	This implies that $b_n$ is not a hard-core predicate, as it contradicts Definition~\ref{def:HCP}.
\end{proof}

%\section{Discussion}
\label{sec:discuss}

We conclude by briefly discussing several future research directions.
One natural question is whether we can weaken the assumptions used to construct our
succinct distributed arguments: for example, in Section~\ref{sec:distprover}
we required SNARKs to ensure consistency between the messages sent by different nodes.
However, the computation itself is deterministic, so conceivably SNARGs could suffice
(as in Section~\ref{sec:dargsForP}).
It is interesting to ask whether certifying a deterministic execution of several (or even two)
Turing machines that interact with one another runs up against the same barrier presented in~\cite{gentry2011separating}
on constructing SNARKs from standard cryptographic assumptions.

Another potential direction for obtaining better constructions is to consider weaker adversary models:
for example, we can require soundness only against an adversary that is itself an efficient \emph{distributed}
algorithm (our constructions are sound against centralized provers).
The motivation for distributed certification is typically not that the prover is \emph{malicious},
but rather that it is buggy, or the network is prone to changes.
This can be used to define weaker adversary models that would require less (or no) use of heavyweight
cryptography.

Finally, our characterization of the power of $\LDP$ algorithms
is at present incomplete.
One-way functions are a specific hardness assumption,
and an average-case one (the function is hard to invert on random inputs).
One can define a worst-case version of one-way functions,
but still, it would amount to assuming
that one \emph{specific} problem (inverting a function) is hard to solve,
and it is not clear that separating $\LDnP$ from $\LDn \cap \PP$
requires such an assumption.
It is interesting to investigate whether 
we can separate $\LDnP$ from $\LDn \cap \PP$
under a more general assumption, e.g., $\PP \neq \NP$, or $\PP \neq \NP \cap \coNP$.


\section{}
\subsection{Soundness proof}
\label{app:distprover}


Let $G = (V, E)$ be a graph of size $n$, and let $\ell = \poly(n)$ be the maximum encoding length of
a message sent by $D$ in graphs of size $n$.%
%\footnote{Recall that the encoding of a message consists of the round number, the edge on which it is sent, and the message contents; for an algorithm that runs on polynomial rounds and sends polynomially-long messages, the encoding of a message is polynomial in $n$.}


 

\begin{proof}[Proof of Soundness]
Suppose for the sake of contradiction that there is an efficient adversary $\Pro^*$ such that for some
non-negligible function $\alpha(\cdot)$
and for all sufficiently large $n$,
we have
\begin{gather}\label{soundnesseq}
    \prb
    {
    \calD(G, x)\neq y 
    \\\wedge  ~\forall \var{v}\in V(G) : \\
    \Ver(\crs, \rt, \var{v}, (x(\var{v}), y(\var{v})),\\
    \qquad \qquad N(\var{v}), \pi(\var{v})) = 1
    }
    {
    \crs \leftarrow \Gen(1^\secpar, n) \\
    (G, x, y, \rt, \pi) \leftarrow \Pro^*(\crs, 1^\secpar, 1^n)
    } \geq \alpha(\secpar)
\end{gather}
For every $G,x$, let $\widetilde{\rt}(G,x,y)$ be the Merkle tree root and the distributed proof of the true messages sent in the execution of $\calD$ on $G$. We would like to claim that \cref{soundnesseq} implies that there exists a negligible function $\mu(\cdot)$, such that
\begin{gather}\label{same_rt_eq}
    \prb
    {
    \calD(G, x)\neq y \\ 
    \wedge\ \forall \var{v}\in V(G) : \\
    \Ver(\crs, \rt, \var{v}, (x(\var{v}), y(\var{v})),\\
    \qquad \qquad N(\var{v}), \pi(\var{v})) = 1 \\
    \wedge ~\rt = \widetilde{\rt}
    }
    {
    \crs \leftarrow \Gen(1^\secpar, n) \\
    (G, x, y, \rt, \pi) \leftarrow \Pro^*(\crs, 1^\secpar, 1^n)
    } \geq \alpha(\secpar) - \mu(\secpar)
\end{gather}
We show why this concludes the proof. For every $G,x,y$ such that $y\neq \calD(G,x)$, there exists some $v\in V(G)$ such that $y(v)\neq \calD(G,x)(v)$. Let $\widetilde{\pi}(crs, \rt, G,x,y,v)$ be the SNARG proof for that. Meaning, the checking TM $M$ in node $v$ should reject $y(v)$. Let $X_v = (v, x(v), y(v), N(v))$. We get that there is an efficient adversary $\Adv^*$ such that
\begin{gather*}\label{same_rt}
    \prb
    {
    \SNARGVer(\crs, \rt, X_v, 1, \pi) = 1 \\
    \wedge ~\SNARGVer(\crs, \rt, X_v, 0, \widetilde{\pi}) = 1
    }
    {
    \crs \leftarrow \Gen(1^\secpar, n) \\
    (\rt, X_v, \pi, \widetilde{\pi}) \leftarrow \Adv^*(\crs, 1^\secpar, 1^n)
    } \geq \alpha(\secpar) - \mu(\secpar)
\end{gather*}
which contradicts the soundness of the underlying SNARG.

We now move to prove \cref{same_rt_eq}.% We use the fact that the CRH in use is deterministic given the hash key (which is part of the crs) and so if $\rt\neq\widetilde{\rt}$, then there must be a round $r$ and an edge $(v,u)\in E(G)$ such that there is no valid opening path from $\rt$ to the true message sent from $v$ to $u$ in round $r$. We contradict this by induction on the round $r$ and the edge $(v,u)$. Formally, for every round $r$ and edge $(v,u)\in E(G)$, let $m_{r,v,u}$ be the true message sent by node $v$ to node $u$ in round $r$ of the algorithm. There exists a negligible function $\nu(\cdot)$ such that for every round $r$ and every edge $(v,u)\in E(G)$

Parse $\crs = (\HT.\hk, \SNARG.\crs)$. For every $r\in[R]$, let $\Gen_{r,i}$ be identical to $\Gen$, with round $r$ and inner round $i$ as the first binding index for all nodes. By the index hiding property of the SNARG, \cref{soundnesseq} implies that there exists a negligible function $\mu(\cdot)$ such that for every $(r,i)\in[R]\times[n]$:
\begin{gather*}\label{soundness}
    \prb
    {
    \calD(G, x)\neq y \\ 
    \wedge\ \forall \var{v}\in V(G) : \\
    \Ver(\crs, \rt, \var{v}, (x(\var{v}), y(\var{v})),\\
    \qquad \qquad N(\var{v}), \pi(\var{v})) = 1
    }
    {
    \crs \leftarrow \Gen_{r,i}(1^\secpar, n) \\
    (G, x, y, \rt, \pi) \leftarrow \Pro^*(\crs, 1^\secpar, 1^n)
    } \geq \alpha(\secpar) - \mu(\secpar)
\end{gather*}

By the somewhere argument of knowledge property of the SNARG, the above implies that there exists a negligible function $\nu(\cdot)$ such that for every $r,i\in[R]\times [n]$:
\begin{gather}\label{extracted}
    \prb
    {
    \calD(G, x)\neq y \\
    \wedge\ \forall \var{v}\in V(G) : \\
    \Ver(\crs, \rt, \var{v}, (x(\var{v}), y(\var{v})),\\
    \qquad \qquad N(\var{v}), \pi(\var{v})) = 1 \\
    \wedge ~ C_\var{v}(r, w_\var{v}) = 1
    }
    {
    \crs, \td \leftarrow \Gen_{r,i}(1^\secpar, n) \\
    (G, x, y, \rt, \pi) \leftarrow \Pro^*(\crs, 1^\secpar, 1^n) \\
    \wedge ~\{w_\var{v}\}_{\var{v}\in V(G)}\leftarrow \SNARGExt(td, \pi(\var{v}))
    } \geq \alpha(\secpar) - \nu(\secpar)
\end{gather}
Parse $w_\var{v} = (cf^\var{v}_{r, i-1}, cf^\var{v}_{r, i}, \rho^\var{v}_{r, i-1}, \rho^\var{v}_{r,i}, m^\var{v}_{r,i}, o^\var{v}_{r,i})$\footnote{$m^\var{v}_{r,i}$ could be either sent message or received message}, and for every node $\var{v}$ and round tuple $(r,i)$, let $\widetilde{cf^\var{v}_{r,i}}$, $\widetilde{m^\var{v}_{r,i}}$ be the true configuration and read message for node $\var{v}$ in round $(r,i)$. We argue that \cref{extracted} implies that there exists a negligible function $\xi(\cdot)$ such that for every $(r,i) \in [R]\times[n]$,

\begin{gather}\label{induction}
    \prb
    {
    \calD(G, x)\neq y  \\
    \wedge\ \forall \var{v}\in V(G) : \\
    \quad \Ver(\crs, \rt, \var{v}, (x(\var{v}), y(\var{v})),\\
    \qquad \qquad N(\var{v}), \pi(\var{v})) = 1 \\
    \quad \wedge ~C_\var{v}((r-1, m), w_\var{v}) = 1 \\
    \quad \wedge ~cf^\var{v}_{r-1, m} = \widetilde{cf^\var{v}_{r-1,m}} \\
    \quad \wedge ~m^\var{v}_{r-1,m} = \widetilde{m^\var{v}_{r-1,m}}
    }
    {
    \crs, \td \leftarrow \Gen_{r,i}(1^\secpar, n) \\
    (G, x, y, \rt, \pi) \leftarrow \Pro^*(\crs, 1^\secpar, 1^n)\\
    \wedge ~\{w_v\}_{v\in V(G)}\leftarrow \SNARGExt(\td, \pi(v))
    } \geq \alpha(\secpar) - r\cdot\xi(\secpar)
\end{gather}
If \cref{induction} is true for every round $r$, for every inner round $i$, it implies \cref{same_rt_eq} from the determinism of the $\HT$ scheme and the soundness of the $\HT$ construction scheme \TODO{prove/claim}. We prove \cref{induction} by induction on the round $r$. 

\paragraph{Notation} Throughout the proof, we use the following notation:
\begin{itemize}
    \item For $v\in V(G)$, $i\in [n]$: $u(v,i)\in V(G)$ is the neighbor of $v$ from which $v$ reads the message sent in the last round, in inner round (of $v$) $i$.
    %\item For $u\in V(G)$, $j\in [n]$, $v_{send}(u,j)\in V(G)$ is the neighbor of $u$ to which $u$ sends a message in inner round $j$.
    \item For $u,v\in V(G)$, $j(u,v)$ is the inner round of $u$ in which $u$ sends message to $v$
    %\item For $v,u\in V(G)$, $i_{recv}(v,u)$ is the inner round of $v$ in which $v$ reads the message that $u$ sent it in the last round.
\end{itemize}
%So, $u_{recv}(v, i_{send}(v,u))=u$.

%by $u_{recv}(v,i)$, $v_{send}(u, i)$ the neighbor of $v$ which $v$ reads its message from the last round in inner round $i$, and

%and by $j(u,v)$ the inner node $j$, in which $u$ sends a message to $v$. So, in every round $r$, in the inner round $i$ $v$ reads the message that $u$ sent $v$ in round $r-1$, in inner round $j$.

\paragraph{Base Case} 
For $r=1$, for every node $\var{v}$ and inner round $i$, since there is only one initial configuration $cf^{\var{v}}_0$, which entirely determine the first message that $\var{v}$ sends (and $\var{v}$ does not read a message in the first round), it follows from the definition of $C_\var{v}$ alongside the collision resistance property of the $\MT$ scheme \TODO{prove said CR, define $C_\var{v}$.} that \cref{extracted} implies there exists a negligible function $\xi'(\cdot)$ such that for every $v\in V(G)$:

\begin{gather*}
    \prb
    {
    \calD(G, x)\neq y \\
    \wedge\ \forall \var{v}\in V(G) : \\
    \quad 
        \Ver(\crs, \rt, \var{v}, (x(\var{v}), y(\var{v})),\\
        \qquad \qquad N(\var{v}), \pi(\var{v})) = 1 \\
        \quad \wedge ~ C_\var{v}(1, w_\var{v}) = 1 \\
    %
    \wedge (~cf^v_{1,i} \neq \widetilde{cf^v_{r,1}} 
    \vee ~m^v_{1,i} \neq \widetilde{m^v_{1,i}} )
    }
    {
    \crs, \td \leftarrow \Gen_{r,i}(1^\secpar, n) \\
    (G, x, y, \rt, \pi) \leftarrow \Pro^*(\crs, 1^\secpar, 1^n) \\
    \wedge ~\{w_\var{v}\}_{\var{v}\in V(G)}\leftarrow \SNARGExt(td, \pi(\var{v}))
    } \leq \xi'(\secpar)
\end{gather*}

Let $\xi_1(\cdot) = n\cdot \xi'(\cdot) + \nu(\cdot)$. $\xi_1(\cdot)$ is negligible as well. From union bound, we get the desired:

\begin{gather*}
    \prb
    {
    \calD(G, x)\neq y \\
    \wedge\ \forall \var{v}\in V(G) : \\
    \quad 
        \Ver(\crs, \rt, \var{v}, (x(\var{v}), y(\var{v})),\\
        \qquad \qquad N(\var{v}), \pi(\var{v})) = 1 \\
        \quad \wedge ~ C_\var{v}(1, w_\var{v}) = 1 \\
        \quad \wedge ~cf^\var{v}_{1,i} = \widetilde{cf^\var{v}_{1,i}} \\
        \quad \wedge ~m^\var{v}_{1,i} = \widetilde{m^\var{v}_{1,i}}
    }
    {
    \crs, \td \leftarrow \Gen_{r,i}(1^\secpar, n) \\
    (G, x, y, \rt, \pi) \leftarrow \Pro^*(\crs, 1^\secpar, 1^n) \\
    \wedge ~\{w_\var{v}\}_{\var{v}\in V(G)}\leftarrow \SNARGExt(td, \pi(\var{v}))
    } \\ \geq \alpha(\secpar) - \xi_1(\secpar)
\end{gather*}

\paragraph{Induction Step} Under the assumption that \cref{induction} holds for round $r-1$ for every $i$, we prove it holds for $r$ for every $i$, by induction on $i$. For clearrance, we'll call the induction on $r$ the "$r$-induction", and the induction on $i$ the "$i$-induction".

\paragraph{$i$ Base Case.} For $i=1$, from the $r$-induction hypothesis, we get for every $i$, and specifically, for $i=m$:
\begin{gather*}
    \prb
    {
    \calD(G, x)\neq y  \\
    \wedge\ \forall \var{v}\in V(G) : \\
    \quad \Ver(\crs, \rt, \var{v}, (x(\var{v}), y(\var{v})),\\
    \qquad \qquad N(\var{v}), \pi(\var{v})) = 1 \\
    \quad \wedge ~C_\var{v}((r-1, m), w_\var{v}) = 1 \\
    \quad \wedge ~cf^\var{v}_{r-1, m} = \widetilde{cf^\var{v}_{r-1,m}} \\
    \quad \wedge ~m^\var{v}_{r-1,m} = \widetilde{m^\var{v}_{r-1,m}}
    }
    {
    \crs, \td_{r-1,m} \leftarrow \Gen_{r-1,m}(1^\secpar, n) \\
    (G, x, y, \rt, \pi) \leftarrow \Pro^*(\crs, 1^\secpar, 1^n)\\
    \wedge ~\{w_\var{v}\}_{\var{v}\in V(G)}\leftarrow \SNARGExt(\td_{r-1,m}, \pi(\var{v}))
    } \geq \alpha(\secpar) - (r-1)\cdot\xi(\secpar)
\end{gather*}

For every $v\in V(G)$, let $Gen'(r,1,v)$ be as $Gen$ with the following binding indices:
\begin{itemize}
    \item All nodes have $(r-1,m)$ as the first binding index.
    \item $v$ has $(r,1)$ as the second binding index.
    \item Denote $u = u(u,1)$ and $j = j(u,v)$. $u$ has $(r-1, j)$ as his second binding index. 
\end{itemize}
From index hiding property of the SNARG, we get there exists a negligible function $\upsilon_1(\cdot)$ such that for every $v\in V(G)$:
\begin{gather*}
    \prb
    {
    \calD(G, x)\neq y  \\
    \wedge\ \forall \var{v}\in V(G) : \\
    \quad \Ver(\crs, \rt, \var{v}, (x(\var{v}), y(\var{v})),\\
    \qquad \qquad N(\var{v}), \pi(\var{v})) = 1 \\
    \quad \wedge ~C_\var{v}((r-1, m), w_\var{v}) = 1 \\
    \quad \wedge ~cf^\var{v}_{r-1, m} = \widetilde{cf^\var{v}_{r-1,m}} \\
    \quad \wedge ~m^\var{v}_{r-1,m} = \widetilde{m^\var{v}_{r-1,m}}
    }
    {
    \crs, \td_{r-1,m}, \td^v_{r,1}, \td^{u}_{r-1, j}\leftarrow \Gen'_{r,1,v}(1^\secpar, n) \\
    (G, x, y, \rt, \pi) \leftarrow \Pro^*(\crs, 1^\secpar, 1^n)\\
    \wedge ~\{w_\var{v}\}_{\var{v}\in V(G)}\leftarrow \SNARGExt(\td_{r-1,m}, \pi(\var{v}))
    }\\
    \geq \alpha(\secpar) - (r-1)\cdot\xi(\secpar) -\upsilon_1(\secpar)
\end{gather*}

From the somewhere argument of knowledge of the SNARGs in $v$ and $u$, there exists a negligible function $\upsilon_2(\cdot)$ such that for every $v\in V(G)$:
\begin{gather*}
    \prb
    {
    \calD(G, x)\neq y  \\
    \wedge\ \forall \var{v}\in V(G) : \\
    \quad \Ver(\crs, \rt, \var{v}, (x(\var{v}), y(\var{v})),\\
    \qquad \qquad N(\var{v}), \pi(\var{v})) = 1 \\
    \quad \wedge ~C_\var{v}((r-1, m), w_\var{v}) = 1 \\
    \quad \wedge ~cf^\var{v}_{r-1, m} = \widetilde{cf^\var{v}_{r-1,m}} \\
    \quad \wedge ~m^\var{v}_{r-1,m} = \widetilde{m^\var{v}_{r-1,m}} \\
    %
    \wedge ~C_v((r,1),w'_v) = 1 \\
    \wedge ~C_{u}(r-1,j),w'_{u}) = 1
    }
    {
    \crs, \td_{r-1,m}, \td^v_{r,1}, \td^{u}_{r-1, j}\leftarrow \Gen'_{r,1,v}(1^\secpar, n) \\
    (G, x, y, \rt, \pi) \leftarrow \Pro^*(\crs, 1^\secpar, 1^n)\\
    \wedge ~\{w_\var{v}\}_{\var{v}\in V(G)}\leftarrow \SNARGExt(\td_{r-1,m}, \pi(\var{v})) \\
    \wedge ~w'_v\leftarrow \SNARGExt(\td^v_{r,1}, \pi(v)) \\
    \wedge ~w'_{u}\leftarrow \SNARGExt(\td^{u}_{r-1,j}, \pi(u))
    }\\
    \geq \alpha(\secpar) - (r-1)\cdot\xi(\secpar) - \upsilon_1(\secpar) - \upsilon_2(\secpar)
\end{gather*}

From the $r$-induction hypothesis, when applied with $i=j$, used specifically on $u$, we parse $$w'_{u} = (cf^{u}_{r-1,j-1}, cf^{u}_{r-1,j}, \rho^{u}_{r-1,j-1}, \rho^{u}_{r-1,j}, m^{u}_{r-1,j}, o^{u}_{r-1,j})$$ and get there exists a negligible function $\upsilon_3(\cdot)$ such that:
\begin{gather*}
    \prb
    {
    \calD(G, x)\neq y  \\
    \wedge\ \forall \var{v}\in V(G) : \\
    \quad \Ver(\crs, \rt, \var{v}, (x(\var{v}), y(\var{v})),\\
    \qquad \qquad N(\var{v}), \pi(\var{v})) = 1 \\
    \quad \wedge ~C_\var{v}((r-1, m), w_\var{v}) = 1 \\
    \quad \wedge ~cf^\var{v}_{r-1, m} = \widetilde{cf^\var{v}_{r-1,m}} \\
    \quad \wedge ~m^\var{v}_{r-1,m} = \widetilde{m^\var{v}_{r-1,m}} \\
    %
    \wedge ~C_{u}(r-1,j),w'_{u}) = 1 \\
    \wedge ~cf^u_{r-1,j} = \widetilde{cf^u_{r-1,j}} \\
    \wedge ~m^u_{r-1,j} = \widetilde{m^u_{r-1,j}} \\
    %
    \wedge ~C_v((r,1),w'_v) = 1 
    }
    {
    \crs, \td_{r-1,m}, \td^v_{r,1}, \td^{u}_{r-1, j}\leftarrow \Gen'_{r,1,v}(1^\secpar, n) \\
    (G, x, y, \rt, \pi) \leftarrow \Pro^*(\crs, 1^\secpar, 1^n)\\
    \wedge ~\{w_\var{v}\}_{\var{v}\in V(G)}\leftarrow \SNARGExt(\td_{r-1,m}, \pi(\var{v})) \\
    \wedge ~w'_v\leftarrow \SNARGExt(\td^v_{r,1}, \pi(v)) \\
    \wedge ~w'_{u}\leftarrow \SNARGExt(\td^{u}_{r-1,j}, \pi(u))
    }\\
    \geq \alpha(\secpar) - (r-1)\cdot\xi(\secpar) - \upsilon_1(\secpar) - \upsilon_2(\secpar) - \upsilon_3(\secpar)
\end{gather*}

Parse $w'_v = (cf'^v_{r,0}, cf'^v_{r,1}, \rho'^v_{r,0}, \rho'^v_{r,1}, m'^v_{r,1}, o'^v_{r,1})$. Note that since $(r,0) = ((r-1),m)$, we get that in the last equation $cf^v_{r-1,m}$ is extracted twice: once as part of $w_v$ and once as part of $w'_v$. From the CR property of the SNARG we get that there's a negligible function $\upsilon_4(\cdot)$, such that the following holds:

\begin{gather*}
    \prb
    {
    \calD(G, x)\neq y  \\
    \wedge\ \forall \var{v}\in V(G) : \\
    \quad \Ver(\crs, \rt, \var{v}, (x(\var{v}), y(\var{v})),\\
    \qquad \qquad N(\var{v}), \pi(\var{v})) = 1 \\
    \quad \wedge ~C_\var{v}((r-1, m), w_\var{v}) = 1 \\
    \quad \wedge ~cf^\var{v}_{r-1, m} = \widetilde{cf^\var{v}_{r-1,m}} \\
    \quad \wedge ~m^\var{v}_{r-1,m} = \widetilde{m^\var{v}_{r-1,m}} \\
    %
    \wedge ~C_{u}(r-1,j),w'_{u}) = 1 \\
    \wedge ~cf^u_{r-1,j} = \widetilde{cf^u_{r-1,j}} \\
    \wedge ~m^u_{r-1,j} = \widetilde{m^u_{r-1,j}} \\
    %
    \wedge ~C_v((r,1),w'_v) = 1 \\
    \wedge ~cf'^v_{r,0} = cf^v_{r-1,m}
    }
    {
    \crs, \td_{r-1,m}, \td^v_{r,1}, \td^{u}_{r-1, j}\leftarrow \Gen'_{r,1,v}(1^\secpar, n) \\
    (G, x, y, \rt, \pi) \leftarrow \Pro^*(\crs, 1^\secpar, 1^n)\\
    \wedge ~\{w_\var{v}\}_{\var{v}\in V(G)}\leftarrow \SNARGExt(\td_{r-1,m}, \pi(\var{v})) \\
    \wedge ~w'_v\leftarrow \SNARGExt(\td^v_{r,1}, \pi(v)) \\
    \wedge ~w'_{u}\leftarrow \SNARGExt(\td^{u}_{r-1,j}, \pi(u))
    }\\
    \geq \alpha(\secpar) - (r-1)\cdot\xi(\secpar) - \upsilon_1(\secpar) - \upsilon_2(\secpar) - \upsilon_3(\secpar) - \upsilon_4(\secpar)
\end{gather*}

Moreover, $m^v_{r,1}$ is extracted twice: once from $w'_v$, as $v$ reads it in round $(r,1)$, and once from $w'_{u}$, as $u$ sends it in round $(r-1,j)$. From the CR property of the Merkle tree, we get there exists a negligible function $\upsilon_5(\cdot)$ such that:


\begin{gather*}
    \prb
    {
    \calD(G, x)\neq y  \\
    \wedge\ \forall \var{v}\in V(G) : \\
    \quad \Ver(\crs, \rt, \var{v}, (x(\var{v}), y(\var{v})),\\
    \qquad \qquad N(\var{v}), \pi(\var{v})) = 1 \\
    \quad \wedge ~C_\var{v}((r-1, m), w_\var{v}) = 1 \\
    \quad \wedge ~cf^\var{v}_{r-1, m} = \widetilde{cf^\var{v}_{r-1,m}} \\
    \quad \wedge ~m^\var{v}_{r-1,m} = \widetilde{m^\var{v}_{r-1,m}} \\
    %
    \wedge ~C_{u}(r-1,j),w'_{u}) = 1 \\
    \wedge ~cf^u_{r-1,j} = \widetilde{cf^u_{r-1,j}} \\
    \wedge ~m^u_{r-1,j} = \widetilde{m^u_{r-1,j}} \\
    %
    \wedge ~C_v((r,1),w'_v) = 1 \\
    \wedge ~cf'^v_{r,0} = cf^v_{r-1,m} \\
    \wedge ~m^v_{r,1} = m^u_{r-1,j}
    }
    {
    \crs, \td_{r-1,m}, \td^v_{r,1}, \td^{u}_{r-1, j}\leftarrow \Gen'_{r,1,v}(1^\secpar, n) \\
    (G, x, y, \rt, \pi) \leftarrow \Pro^*(\crs, 1^\secpar, 1^n)\\
    \wedge ~\{w_\var{v}\}_{\var{v}\in V(G)}\leftarrow \SNARGExt(\td_{r-1,m}, \pi(\var{v})) \\
    \wedge ~w'_v\leftarrow \SNARGExt(\td^v_{r,1}, \pi(v)) \\
    \wedge ~w'_{u}\leftarrow \SNARGExt(\td^{u}_{r-1,j}, \pi(u))
    }\\
    \geq \alpha(\secpar) - (r-1)\cdot\xi(\secpar) - \upsilon_1(\secpar) - \upsilon_2(\secpar) - \upsilon_3(\secpar) - \upsilon_4(\secpar) - \upsilon_5(\secpar)
\end{gather*}

Now, since from the definition of the round indexing $\widetilde{cf^v_{r-1,m}} = \widetilde{cf^v_{r,0}}$ and from the definitions of $u,j$, $\widetilde{m^v_{r,1}} = \widetilde{m^u_{r-1, j}}$, We can re-writing the last equation as follows:
\begin{gather*}
    \prb
    {
    \calD(G, x)\neq y  \\
    \wedge\ \forall \var{v}\in V(G) : \\
    \quad \Ver(\crs, \rt, \var{v}, (x(\var{v}), y(\var{v})),\\
    \qquad \qquad N(\var{v}), \pi(\var{v})) = 1 \\
    \quad \wedge ~C_\var{v}((r-1, m), w_\var{v}) = 1 \\
    \quad \wedge ~cf^\var{v}_{r-1, m} = \widetilde{cf^\var{v}_{r-1,m}} \\
    \quad \wedge ~m^\var{v}_{r-1,m} = \widetilde{m^\var{v}_{r-1,m}} \\
    %
    %\wedge ~C_{u}(r-1,j),w'_{u}) = 1 \\
    %\wedge ~cf^u_{r-1,j} = \widetilde{cf^u_{r-1,j}} \\
    %\wedge ~m^u_{r-1,j} = \widetilde{m^u_{r-1,j}} \\
    %
    \wedge ~C_v((r,1),w'_v) = 1 \\
    \wedge ~cf'^v_{r,0} = \widetilde{cf^v_{r-1,m}} \\
    \wedge ~m^v_{r,1} = \widetilde{m^v_{r,1}}
    }
    {
    \crs, \td_{r-1,m}, \td^v_{r,1}, \td^{u}_{r-1, j}\leftarrow \Gen'_{r,1,v}(1^\secpar, n) \\
    (G, x, y, \rt, \pi) \leftarrow \Pro^*(\crs, 1^\secpar, 1^n)\\
    \wedge ~\{w_\var{v}\}_{\var{v}\in V(G)}\leftarrow \SNARGExt(\td_{r-1,m}, \pi(\var{v})) \\
    \wedge ~w'_v\leftarrow \SNARGExt(\td^v_{r,1}, \pi(v)) \\
    %\wedge ~w'_{u}\leftarrow \SNARGExt(\td^{u}_{r-1,j}, \pi(u))
    }\\
    \geq \alpha(\secpar) - (r-1)\cdot\xi(\secpar) - \upsilon_1(\secpar) - \upsilon_2(\secpar) - \upsilon_3(\secpar) - \upsilon_4(\secpar) - \upsilon_5(\secpar)
\end{gather*}
and from the definition of $C_v$, we get that

\begin{gather*}
    \prb
    {
    \calD(G, x)\neq y  \\
    \wedge\ \forall \var{v}\in V(G) : \\
    \quad \Ver(\crs, \rt, \var{v}, (x(\var{v}), y(\var{v})),\\
    \qquad \qquad N(\var{v}), \pi(\var{v})) = 1 \\
    \quad \wedge ~C_\var{v}((r-1, m), w_\var{v}) = 1 \\
    \quad \wedge ~cf^\var{v}_{r-1, m} = \widetilde{cf^\var{v}_{r-1,m}} \\
    \quad \wedge ~m^\var{v}_{r-1,m} = \widetilde{m^\var{v}_{r-1,m}} \\
    %
    %\wedge ~C_{u}(r-1,j),w'_{u}) = 1 \\
    %\wedge ~cf^u_{r-1,j} = \widetilde{cf^u_{r-1,j}} \\
    %\wedge ~m^u_{r-1,j} = \widetilde{m^u_{r-1,j}} \\
    %
    \wedge ~C_v((r,1),w'_v) = 1 \\
    \wedge ~cf'^v_{r,0} = \widetilde{cf^v_{r-1,m}} \\
    \wedge ~m^v_{r,1} = \widetilde{m^v_{r,1}} \\
    \wedge ~cf^v_{r, 1} = \widetilde{cf^v_{r,1}}
    }
    {
    \crs, \td_{r-1,m}, \td^v_{r,1}, \td^{u}_{r-1, j}\leftarrow \Gen'_{r,1,v}(1^\secpar, n) \\
    (G, x, y, \rt, \pi) \leftarrow \Pro^*(\crs, 1^\secpar, 1^n)\\
    \wedge ~\{w_\var{v}\}_{\var{v}\in V(G)}\leftarrow \SNARGExt(\td_{r-1,m}, \pi(\var{v})) \\
    \wedge ~w'_v\leftarrow \SNARGExt(\td^v_{r,1}, \pi(v)) \\
    %\wedge ~w'_{u}\leftarrow \SNARGExt(\td^{u}_{r-1,j}, \pi(u))
    }\\
    \geq \alpha(\secpar) - (r-1)\cdot\xi(\secpar) - \upsilon_1(\secpar) - \upsilon_2(\secpar) - \upsilon_3(\secpar) - \upsilon_4(\secpar) - \upsilon_5(\secpar)
\end{gather*}

From the index hiding property of the SNARG, there exists a negligible function $\upsilon_6(\cdot)$ such that the following holds:

\begin{gather*}
    \prb
    {
    \calD(G, x)\neq y  \\
    \wedge\ \forall \var{v}\in V(G) : \\
    \quad \Ver(\crs, \rt, \var{v}, (x(\var{v}), y(\var{v})),\\
    \qquad \qquad N(\var{v}), \pi(\var{v})) = 1 \\
    \quad \wedge ~C_\var{v}((r-1, m), w_\var{v}) = 1 \\
    \quad \wedge ~cf^\var{v}_{r-1, m} = \widetilde{cf^\var{v}_{r-1,m}} \\
    \quad \wedge ~m^\var{v}_{r-1,m} = \widetilde{m^\var{v}_{r-1,m}} \\
    %
    %\wedge ~C_{u}(r-1,j),w'_{u}) = 1 \\
    %\wedge ~cf^u_{r-1,j} = \widetilde{cf^u_{r-1,j}} \\
    %\wedge ~m^u_{r-1,j} = \widetilde{m^u_{r-1,j}} \\
    %
    \wedge ~C_v((r,1),w'_v) = 1 \\
    %\wedge ~cf'^v_{r,0} = \widetilde{cf^v_{r-1,m}} \\
    \wedge ~m^v_{r,1} = \widetilde{m^v_{r,1}} \\
    \wedge ~cf^v_{r, 1} = \widetilde{cf^v_{r,1}}
    }
    {
    \crs, \td \leftarrow \Gen_{r,i}(1^\secpar, n) \\
    (G, x, y, \rt, \pi) \leftarrow \Pro^*(\crs, 1^\secpar, 1^n)\\
    \wedge ~\{w_\var{v}\}_{\var{v}\in V(G)}\leftarrow \SNARGExt(\td_{r-1,m}, \pi(\var{v})) \\
    \wedge ~w'_v\leftarrow \SNARGExt(\td^v_{r,1}, \pi(v)) \\
    %\wedge ~w'_{u}\leftarrow \SNARGExt(\td^{u}_{r-1,j}, \pi(u))
    }\\
    \geq \alpha(\secpar) - (r-1)\cdot\xi(\secpar) - \upsilon_1(\secpar) - \upsilon_2(\secpar) - \upsilon_3(\secpar) - \upsilon_4(\secpar) - \upsilon_5(\secpar) -\upsilon_6(\secpar)
\end{gather*}

By applying the union bound (as with the base case of the $r$-induction), and setting $\xi(\cdot) = n\cdot\sum_{i=1}^6 \upsilon(\secpar)$, we get the $i$-induction base.


\paragraph{$i$-Induction Step} The proof is exactly like for the base case, with the only difference being using the $i$-induction hypothesis instead of the $r$-induction hypothesis. That is, instead of using \cref{induction} with $r-1,m$, use it with $r,i-1$.



\end{proof}


\bibliographystyle{ACM-Reference-Format}
%\bibliography{bib,biblio}
\bibliography{bib}

%\appendix
%\appendixpage

%\section{Further Background and Discussion on the Cryptographic Primitives Used}
\label{app:crypto}

\subsection{Brief Introduction to Computationally Sound Proof Systems}
\paragraph{Computationally Sound Proofs}
The idea of a proof system that its soundness only holds for adversaries of bounded computational power (also known as an argument) was introduced by Micali in \cite{micali2000computationally}, and was based on an earlier interactive protocol of Kilian (\cite{kilian1992note}) and a variant of the Fiat-Shamir paradigm~\cite{fiat1986prove}.
Kilian's protocol is based on the following idea. The verifier choose a key to a hash function from a collision-resistant hash family, and send it to the prover. The prover then uses a Merkle Tree \cite{merkle1989certified} induced by the hash function to commit to a polynomial Probabilistically Checkable Proof (PCP) \cite{babai1991checking, arora1998probabilistic, arora1998proof, feige1991approximating}, using the instance and the witness (that the prover has, and the verifier does not have), and send the commitment to the verifier. Next, the verifier chooses the queries for the PCP, and the prover then sends answers to the queries along with an opening of the Merkle tree, proving that these indeed where the values in the queries' indices in the PCP.
\TODO{Rotem: capitalized/uncapitalized ro and crs models?}
\paragraph{The Random Oracle Model}
A Random Oracle is simply a deterministic function that is chosen randomly from all possible functions (from a certain input length to a certain output length), that provide an oracle access: the parties cannot hold a full description of the function, but they can query it and get answers to their queries, in $O(1)$ steps per quary. Though this is not a realistic model, it is very useful in both theory and practice~\cite{ROM93BMR}. In the random oracle model, Micali used the paradigm of Fiat and Shamir~\cite{fiat1986prove}, to lose the interaction in Kilian's protocol, and by that, to obtain a computationally sound proof (which later has been called a SNARG).

\paragraph{SNARGs in the CRS model}
Since Micali's seminal work, there have been several attempts~\cite{aiello2000fast, dwork2004succinct, di2008succinct, groth2010short, bitansky2012extractable} to obtain succinct non-interactive arguments (SNARGs) in a more realistic model, but still different from the plain model: the common reference string (CRS) model.\footnote{
Some works, such as \cite{bitansky2013recursive} did construct SNARGs (in fact, SNARKs) in the plain model.
} In the CRS model, it is assumed that all parties have access to a string generated in a trusted manner. The Common Reference String model is a generalization of the Common \emph{Random} String model, where the reference string is taken from the uniform distribution (in the common \emph{reference} string model, it may be taken from any arbitrary distribution). These two versions are in fact equal in power~\cite{canetti2001universally}. The main difference between the ROM and CRS model is that proofs in the ROM are \emph{heuristic}, since the actual protocol instantiation uses a hash function that is blatantly NOT a random oracle. In contrast, proofs in the CRS model have a standard reduction-based proof of security, and so are not heuristic. The main advantage of the CRS model over the Random Oracle model is that security is standard, and doesn't rely on a heuristic belief system that the real protocol that uses a standard hash function is secure. The main disadvantage is that the CRS needs to be generated somehow, and in the absence of a trusted setup, this is not trivial. An example where CRS is used in reality is the Zcash protocol: there, the CRS is generated by a Multi-Party Computation (MPC) protocol \cite{crsZkSnark}. Whenever at least  one participant is honest, then the setup is trust-worthy.

\paragraph{SNARGs from knowledge assumptions and SNARKs}
To this day (to our knowledge), in every construction of a SNARG in the CRS (or in the plain) model, in order to prove its soundness, some knowledge assumption (or: knowledge \emph{extractability}) was assumed. Knowledge assumptions capture the intuition that any algorithm whose output is related to a certain value that is hard to compute (for instance, a convincing proof, that is related to an $\NP$-witness), must obtain that value along the computation. This assumption is non-falsifiable, meaning, one cannot define a game where at the end of the game, we could efficiently decide whether the assumption was broken or not. This is unlike hardness assumptions, where the design of such a game is very easy to design \TODO{word}. Under such assumptions, the SNARG candidate becomes stronger: it becomes a SNARG of knowledge --- a SNARK; instead of only being sound, we can promise (under the knowledge assumption), that any prover that manages to convince the verifier, knows a witness. This is useful for composing such arguments with other primitives, and in particular, this is useful for our constructions.

In \cite{gentry2011separating}, a substantial barrier to proving the soundness of a SNARG for $\NP$ under falsifiable assumptions was presented; it was shown that SNARGs\footnote{
The result of \cite{gentry2011separating} is specific for \emph{adaptive} soundness.
} cannot be proven secure by a reduction to \emph{any} falsifiable assumption if the reduction is black-box in the adversary's code. Since, many works were either focused on what can be done without knowledge assumptions, which include,
for example, SNARGs for deterministic computation~\cite{kalai2019delegate, jawale2021snargs, choudhuri2021snargs},
batch arguments for $\NP$~\cite{choudhuri2021non, hulett2022snargs, devadas2022rate, cryptoeprint:2022/1320},
and incrementally verifiable computation~\cite{paneth2022incrementally},
or focused on refining the knowledge assumptions used and generalizing them~\cite{bitansky2012extractable}.


\subsection{Vector Commitments, Merkle Trees, and Collision Resistant Hash Families}
\paragraph{Vector Commitments}
Vector Commitments are defined formally in~\cite{catalano2013vector}, where they are constructed in a way that makes them more succinct than what we required in this work: they show VCs \TODO{initials for VCs, SNARGs, etc} with commitment and opening length that depends only on the security parameter, and is completely \emph{independent} of the input size.

For our use, a more classic form of VCs suffices; a Merkle Tree~\cite{merkle1989certified} induced by a CRH satisfies our completeness, position-binding \TODO{capitalize?}, and succinctness requirements.
\TODO{Change hash definition to family}
\begin{definition} [Collision-Resistant Hash (CRH)]
    A hash family $(\Gen, \Hash)$ is considered \emph{collision-resistant}, if for any efficient adversary $\mathcal{A}$, there exists a negligible function $\epsilon(\cdot)$, such that for every $\secpar\in \mathbb{N}$,
    \begin{gather*}
        \prob{}
        {
        \begin{array}{ll}
        h(x_1)=h(x_2)
        \end{array}
        \middle\vert
        \begin{array}{ll}
             hk \leftarrow Gen(1^n, 1^\secpar) \\
             x_1, x_2 \leftarrow\calA(hk)
        \end{array}
        } \leq \epsilon(\secpar)
    \end{gather*}
\end{definition}

\paragraph{Merkle Trees} We describe Merkle Trees informally. Let $h:\{0,1\}^2k\to\{0,1\}^k$ be a function. The root of the Merkle Tree induced by $h$ on an input $x = \in\{0,1\}^{n\times 2k}$, $rt_h(x)$ is computed in $\log k$ iterations as follows. In every iteration, the current sequence (starting when the current sequence is the input) is divided into $k$-bit blocks, and $h$ is evaluated on every pair of adjacent odd and even indexed blocks, to obtain the new sequence, which its length is half the length of the previous sequence. In the end, we get a value of length $O(k \log n)$. This is the commitment to $x$. To open a commitment in position $i$, which is the $i$th $2k$-bit block of $x$, the committer only has to show one node in every level of the commitment tree (except for the lowest level, where it has to show two nodes). A verifier can verify that in each level, $h$ was evaluated correctly.

If $h$ is taken from a CRH family, then this scheme is a position-binding VC, since in order to open the same position in two different ways, one must find a collision.

Moreover, for our use in Section~\ref{sec:dargsForP} \TODO{verify section label and all labels and references and capitalized Sections}, where we also require the VC be \emph{inverse collision-resistant}, it is essential we use Merkle Trees. This is because Merkle Trees are based on a \emph{deterministic} function, and the process of verifying an opening is based on evaluating the same function that is used for the commitment. In order to obtain two different commitments and full openings of them (that is, opening for every position to each commitment), one needs to find two different \emph{outputs} for the same \emph{input} of a deterministic function, which is impossible. Note that in the definition of the inverse collision-resistance property, we allowed this to happen with a negligible probability, because nothing more is necessary for our proof, but in fact, when instantiating the VC by a Merkle Tree, we can promise that such "inverse collisions" are never found.





\TODO{Put the various assumptions here --- what's currently in the form of the various corollaries}



\


\subsection{SNARKs for $\NP$}
There is an abundance of constructions of SNARKs. In this work, we refer to the construction in \cite{bitansky2013SNARKsLIPs}, but it could be replaced by any \emph{publically verifiable} SNARK construction, with or without preprocessing. We refer to~\cite{bitansky2013SNARKsLIPs, bitansky2013recursive} for more details on public vs. private verifiability and preprocessing in SNARKs.
From the construction in \cite{bitansky2013SNARKsLIPs}, we get SNARKs for $\NP$ from the knowledge-of-exponent assumption in bilinear groups. 
\TODO{Copy it?}

\subsection{RAM SNARGs for Deterministic Computation}\label{app:ramsnargs}
For a polynomial-time Turing machine $M$, we would like to have a way of verifying that the execution of $M$ on input $x$ was executed correctly, and in particular, the output is correct. We would like to do that more efficiently than simulating the computation, and more importantly, without having access to the entire input. RAM Delegation allows us to do so. In general, a RAM Machine is a deterministic Turing machine that has random access to memory that is much longer (mostly, exponentially longer) than its local state, and a RAM SNARG is a SNARG that proves that a RAM machine indeed outputs a certain output, without having the verifier simulate the entire execution (that requires access to a long memory). A RAM SNARG is associated with a \emph{digest} algorithm, that processes the long input into a much shorter string that the verifier can read. In \cite{cryptoeprint:2022/1320}, this definition is extended to \emph{Flexible SNARGs for RAM}, which is a RAM SNARG where the digest can be implemented by any hash family, and the SNARG is sound if that hash family has local openings.

Let $M$ be a RAM machine. A flexible RAM SNARG for $M$ is associated with a hash family with local opening\footnote{
For the use in \cite{cryptoeprint:2022/1320}, and for our use, a succinct vector commitment satisfies all the required properties of the hash with local openings, where $\HTGen = \VCGen, \HTHash = \VCCom, \HTOpen = \VCOpen, \HTVer = \VCVer$
}
$$\HT = \HTGen, \HTHash, \HTOpen, \HTVer$$ and consists of the following  algorithms.\footnote{
In \cite{cryptoeprint:2022/1320}, they include in the SNARG definition also the digestion algorithm, that uses a key generated by $\Gen$ and applies $\HTHash(x)$ to obtain $d$. Since this is fully defined by the rest of the algorithms mentioned here, we omit it from the definition.
}
\paragraph{$\Gen(1^\secpar, T)\to \var{\crs%, hk, vk
}$:} a setup procedure that takes as input a security parameter $1^\secpar$ and a time bound $T$, and outputs a common reference string $\var{\crs}$.

\paragraph{$\Pro(\var{\crs}, x)\to b, \pi$:}\footnote{
In \cite{cryptoeprint:2022/1320}, the input $x$ is divided into a short explicit input $x_{exp}$ that the verifier has, and a long input the verifier doesn't have $x_{imp}$. Since it is not required for the SNARG's properties that $x_{exp}$ be non-empty, and we only use the node's input outside the SNARG itself, we omit $x_{exp}$ from the definition.
}
takes a common reference string $\var{\crs}$ obtained from $\Gen(1^{\secpar}, T)$,
an instance $x\in\{0,1\}^\ell$, and outputs a bit $b = M(x)$ and a proof $\pi$.

\paragraph{$\Ver(\var{\crs}, %vk,
d, b, \pi)\to \set{0,1}$:} takes a common reference string $\var{\crs}$, %a verification key for $\HT$, $vk$,
a digest if memory $d$, an output bit $b$, and proof $\pi$,
and outputs an acceptance bit.


\begin{definition}[Flexible RAM SNARGs]
\Enote{I might have over-copied here}
Let $M$ be a RAM machine. A Flexible RAM SNARG for $M$ associated with a hash family with local opening
$\HT = (\HTGen, \HTHash, \HTOpen, \HTVer)$ satisfies the following properties.
\paragraph{Completeness} 
There exists a negligible function $\epsilon(\cdot)$, such that for every $\lambda, n\in \mathbb{N}$ such that $n\leq T(n) \leq 2^\lambda$, and every $x\in \{0,1\}^n$ such that $M$ on $x$ halts after $T(n)$ steps,
\begin{gather*}
    \prob{}
    {
    \begin{array}{ll}
    \Ver(\crs, d, b, \pi) = 1 \\
    \wedge b = M(x)
    \end{array}
    \middle\vert
    \begin{array}{ll}
    \var{\crs} \leftarrow \Gen(1^\secpar, T) \\
    (b, \pi) \leftarrow \Pro(\crs, x) \\
    d \leftarrow \HTHash(crs, x)
    \end{array}
    } = 1 - \epsilon(\lambda)
\end{gather*}

\paragraph{Soundness}\footnote{
In \cite{cryptoeprint:2022/1320}, it is shown that this soundness notion can be replaced by a different one, called \emph{Partial Input Soundness}. We do not require it.
}
For any efficient adversarial prover $\Pro^*$ and a polynomial $T = T(\lambda)$, there exists a negligible function $\epsilon(\cdot)$, such that
\begin{gather*}
    \prob{}
    {
    \begin{array}{ll}
    \Ver(\var{\crs}, d, 0, \pi_0) = 1 \\
    \wedge \Ver(\crs, d, 1, \pi_1) = 1
    \end{array}
    \middle\vert
    \begin{array}{ll}
    \var{\crs} \leftarrow \Gen(1^\secpar, T) \\
    (d, x, \pi_0, \pi_1) \leftarrow \Pro^*(\var{\crs}) \\
    \end{array}
    } \leq \epsilon(\secpar)
\end{gather*}
\paragraph{Succinctness} The length of the proof output of $\Pro$ is $\poly(\lambda, \log n,\log T)$.
\paragraph{Verifier Efficiency} $\Ver$ runs in time $\poly(\lambda, |\pi|) = \poly(\lambda, \log n, \log T)$    
\end{definition}



%\section{Pseudocode and Correctness Proofs}
\label{app:code}

\subsection{Succinct Distributed Arguments for NP from SNARKS}\label{app:dargsForNP}
\paragraph{Detailed construction.}
Let $\Lan$ be an $\NP$-language on graphs, and let $V_\Lan$ be a polynomial-time Turing machine such that:
\begin{gather*}
	G\in \Lan \Leftrightarrow \exists w \in \set{0,1}^{\poly(|G|)} \text{ such that } V_\Lan(\var{L}(\var{G}),w)=1.
\end{gather*}


Fix a vector commitment $(\VCGen, \VCCom, \VC.\Open, \VC.\Ver)$.
We define the following $\NP$ language.\footnote{We give here $1^n$ as part of the input since for a succinct commitment scheme, $c$ should be much shorter than $1^n$, whereas the witness size is bounded from below by $n^2$.}\footnote{$\aux$ is existentially quantified, but a polynomial-time verifying machine for $\Lancom$ would not need $\aux$ as part of the witness, since $\VCCom$ is polynomial-time.}
\begin{gather*}
    \Lancom =
    \left\{
    (c, 1^n, \crs)
    \middle\vert
    \ \exists L, w\ :
    \begin{array}{ll}
	    & L = (L_1,\ldots,L_m) \text{ is an adjacency list}\\
         % \qquad L_i=(v,N(v)_i) \\
         & \exists \aux: \\
         & \VCCom(\crs, L_1,\ldots, L_m) = (c, \aux) \\
         & \wedge\ V_\Lan(L, w) = 1 \\
	 & \wedge \text{the graph represented by $L$ is symmetric and connected}
    \end{array}
    \right\}
\end{gather*}

Now fix a SNARK  $(\SNARKGen, \SNARKPro, \SNARKVer, \SNARKExt)$ for $\Lancom$.
The succinct distributed argument $(\Gen, \Pro, \Ver)$ for $\Lan$ is defined as follows.




\begin{subfigures}\label{fig:dargsForNP}
    \begin{nicefig}[h]{The Setup Procedure of Section~\ref{sec:dargsForNP}: $\Gen(1^\secpar, n)$ }{fig:dargsForNPGen}
        \begin{algorithmic}[1]
            \State $\crsvc \leftarrow \VCGen(1^\secpar, n)$
            \State $\crssnark \leftarrow \SNARKGen(1^\secpar, 1^n)$
            \State Output: $\crs = (\crsvc, \crssnark)$
        \end{algorithmic}
    \end{nicefig}
    
    \begin{nicefig}[h]{The Prover of Section~\ref{sec:dargsForNP} : $\Pro(\crs, G, w)$ }{fig:dargsForNPPro}
        \begin{algorithmic}[1]
            \State Parse $\crs=(\crsvc, \crssnark)$
	    \State Let $v_1,\ldots,v_n$ be the nodes of $G$, sorted by UID
	    \State Let $L \leftarrow (L_1,\ldots,L_n)$,
	    where $L_i = ( v_i, N(v_i))$ for each $i \in [n]$
            \State $(c,\aux) \leftarrow \VCCom(\crsvc, L_1,\ldots, L_n)$
            \State Compute for every $i$: $\Lambda_i \leftarrow \VCOpen(\crsvc, L_i, i, \aux)$
            \State Compute $\pisnark \leftarrow \SNARKPro(\crssnark, c, (L, w))$
            \State Output $\{\pi(v_i)\}_{v_i\in V(G)}$, where for every $v_i\in V(G)$: $\pi(v_i) = (c, i, \Lambda_{i}, \pisnark)$
        \end{algorithmic}
    \end{nicefig}

    \begin{nicefig}[h]{The Verifier of Section~\ref{sec:dargsForNP} : $\Ver(\crs, \var{N}(v), v, \pi(v))$ }{fig:dargsForNPVer}
        \begin{algorithmic}[1]
            \State Parse $\crs=(\crsvc, \crssnark)$
            \State Parse $\pi = (c, i, \Lambda_i, \pisnark)$
            \State Verify that for every neighbor $u \in N(v)$:
	    $c(u) = c$ and $\pisnark(u) = \pisnark$ (otherwise output 0)\label{step:verifyCom}
            \State Output 1 if the following holds:
            \begin{itemize}
                \item $\VCVer(\crsvc, c, (v,\var{N}(v)), i, \Lambda_i) = 1$
                \item $\SNARKVer(\crssnark, c, \pi) = 1$
            \end{itemize}
        \end{algorithmic}
    \end{nicefig}
\end{subfigures}




\begin{claim}
$\Gen, \Pro, \Ver$ is a succinct distributed argument.
\end{claim}

\begin{proof}
Perfect completeness and succinctness follow immediately from the perfect completeness and the succinctness of the VC and the SNARK. We now prove soundness.

Assume towards contradiction that there exists an efficient prover $\Pro^*$ and a non-negligible function $\alpha(\cdot)$, such that the following holds with probability at least $\alpha(\lambda)$.
\begin{gather}\label{eq:soundnessBreakNP}
    \prob{}
    {
    \begin{array}{ll}
    G\notin \Lan \\
    \wedge\ \forall v\in V(G) : \Ver(\crs, \var{N}_G(v), v, \pi_{v}) = 1
    \end{array}
    \middle\vert
    \begin{array}{ll}
    \crs \leftarrow \Gen(1^\secpar, n) \\
    (G, \{\pi_{v}\}_{v\in V(G)}) \leftarrow \Pro^*(\crs, 1^\secpar, 1^n) \\
    \end{array}
    }
    .
\end{gather}

First, note that since all nodes verify that they agree with their neighbors
on the vector commitment $c$ and SNARK proof; if all nodes accept, then the prover gave the same 
values to all nodes. We assume this from now on.

We use $\Pro^*$ to construct an efficient adversary $\Adv$ that breaks either the SNARK or the VC.
The adversary $\Adv$ proceeds as follows: 
\begin{itemize}
	\item Given $\crs, 1^{\lambda}, n$,
		it first uses $\Pro^*(\crs, 1^{\lambda}, n)$
		to obtain a graph $G$ and certificates $\set{ \pi(v) }_{v \in V(G)}$.
	\item 
		From $\pi(v)$ (for an arbitrary $v$, since they all agree),
		the adversary extracts the vector commitment $c$ and SNARK proof $\pisnark$.
	\item The adversary extracts the $\NP$-witness $(L^*, w) \leftarrow \SNARKExt_{\Pro^*}(\crs, c, \pisnark)$
		from the SNARK proof.
	\item If $(L^*, w)$ is not a valid witness for the membership $(c, 1^n, \crs) \in \Lancom$,
		then the adversary has broken adaptive proof of knowledge property,
		as $\SNARKVer(\crssnark, c, \pi) = 1$.
	\item Otherwise, $L^*$ is an adjacency list, $L = (L_1,\ldots,L_m)$ (for some $m$ which is not necessarily
		equal to $n$), such that $\VCCom(\crs, L_1,\ldots, L_m) = (c, \aux)$,
		$V_\Lan(L, w) = 1$,
		and the graph $G'$ represented by $L$ is symmetric and connected.
		In particular, since $v_{\Lan}(L, w) = 1$,
		we have $G' \in \calL$.
		Thus, whenever $G \not \in \calL$,
		we must have $G' \neq G$, in other words, $L$
		is not the adjacency list of $G$.

		For each $v \in V$,
		let $i(v)$ be the index appearing in $v$'s certificate, $\pi(v) = (c, i(v), \Lambda(v), \pisnark)$.
		There are two cases:
		\begin{itemize}
			\item For some node $v \in V(G)$ we have
				$L_{i(v)} \neq (v, N(v))$.
				But node $v$ verifies that
				entry $i(v)$ of $c$ opens
				to its true neighborhood,
				i.e.,
				$\VCVer(\crsvc, c, (v,\var{N}(v)), i(v), \Lambda) = 1$,
				and so the adversary has broken the binding property of the vector commitment.
			\item For every $v \in V(G)$
				we have $L_{i(v)} = (v, N(v))$.
				This implies that $i(v) \neq i(v')$ for every $v \neq v' \in V(G)$,
				and therefore $\left| \set{ i(v) }_{v \in V(G)} \right| = n$.
				We claim that in this case $G = G'$,
				that is, $L$ is the true adjacency list of $G$, contradicting
				our assumption that it is not.

				If $|L| = |V(G)|$,
				then $\left| \set{ i(v) }_{v \in V(G)} \right| = n$
				and $L_{i(v)} = (v, N(v))$ for every $v \in V$,
				the list does match $G$.
				Thus, assume that $|L| \neq |V(G)|$.

				It must be that $|L| > |V(G)|$,
				as we already said that $\left| \set{ i(v) }_{v \in V(G)} \right| = n$.
				Thus, there is some entry $(u, N)$ in $L$,
				such that $u$ is not a node of $G$.
				Since $G'$ is connected,
				there is an edge $\set{w, w'}$
				in the cut between $V(G') \setminus V(G)$
				and $V(G)$,
				with $w \in V(G)$ and $w' \not \in V(G)$.
				We know that $L_{i(w)} = ( w, N(w) )$,
				but $w' \not \in N(w)$ (since $w' \not \in V(G)$);
				this is a contradiction,
				since we assumed that $L$ represents $G'$.
		\end{itemize}
\end{itemize}
\end{proof}


%Let $M$ be the polynomial Turing machine such that 
%\begin{equation*}
%(c, 1^\secpar, \crs)\in \Lancom \Leftrightarrow \exists G, w:\ M(c, 1^\secpar, \crs, L, w)=1
%.
%\end{equation*}
%
%Let $G, \{\pi_{v}\}_{v\in V(G)}$ be the graph and the proofs outputted from $\Pro^*(\crs, 1^\lambda ,n)$.
%On input $\crs, 1^\lambda, n$, $\Adv$ outputs $c, i, L_i, \Lambda_{i}, L'_i, \Lambda_{i}', \pisnark^*, L^*$, where:
%\begin{itemize}
    %\item $c, \pisnark$ are the commitment and the SNARK proof in $\pi_v$ that is consistent across all $v\in V(G)$.
    %\item $L^*$ is obtained from $\SNARKExt_{\Pro^*}(crs, c, \pisnark)$
    %\item If there exists $j\in [n]$ such that $L(G)_j\neq L^*$:
    %\begin{itemize}
        %\item Let $v$ be the node with the $j$th smallest identifier, and let $(c, i_v, \Lambda_{i_v}, \pisnark) = \pi_v$ obtained from $\Pro^*(crs, 1^\lambda, n)$.
        %\item $i = i_v$.
        %\item $\Lambda_i = \Lambda_{i_v}$.
        %\item Let $\aux$ be the auxiliary information obtained from $\VCCom(\crs, L_1^*,\ldots, L_n^*)$.\\ $\Lambda_i' = \VCOpen(crs, L^*_i, i_v, \aux)$.
    %\end{itemize}
    %Otherwise, $i, \Lambda_i, \Lambda'_i = \bot, \bot, \bot$
%\end{itemize}
%Let $w^*$ be the witness obtained from $\SNARKExt_{\Pro^*}$. Let $\var{SNARKBreak}$ be the event that $M$ rejects $(c, 1^\lambda, L^*, w^*)$, and let $\var{VCBreak}$ be the event that $\Adv$ outputs $i, \Lambda_i, \Lambda'_i$ such that
%\begin{gather*}
    %\VCVer(\crs, c, L_i, i, \Lambda_i) = 1 \wedge \VCVer(\crs, c, L_i', i, \Lambda_i') = 1
%\end{gather*}
%
%Whenever the event of \ref{eq:soundnessBreakNP} occurs, one of the following must hold:
%\begin{itemize}
    %\item $M$ rejects $(c, 1^\lambda, L^*, w^*)$, in which case, $\var{SNARKBreak}$ occurs.
    %\item $M$ accepts $(c, 1^\lambda, L^*, w^*$. In that case, since $G\notin \Lan$, $L^*\neq L(G)$. So, $\Adv$ outputs $i, \Lambda_i$ where the node $v$ that satisfies $i_v = i$, simulated $\VCVer(\crs, c, L_i, i, \Lambda_i)$ which outputted 1, since $v$ accepted.
    %In addition, since $M$ accepts $(c, 1^\lambda, L^*, w^*$, we have that $\VCCom(\crs, L_1^*,\ldots, L_n^*) = c, \aux$, and from the completeness of the VC, since $L_i'$ was obtained by $\VCOpen(crs, L^*_i, i_v, \aux)$, we get that $\VCVer(\crs, c, L_i', i, \Lambda_i') = 1$. So, $\var{VCBreak}$ occur. 
%\end{itemize}
%Since the event of \ref{eq:soundnessBreakNP} happens with probability at least $\alpha(\lambda)$, one of the events $\SNARKBreak, \VCBreak$ happens with probability at least $\alpha(\lambda)/2$, which is also a non-negligible function of $\lambda$, and so $\Adv$ breaks at least one of the following: the argument of knowledge property of the SNARK, the position-binding of the VC.
%\end{proof}
%



\subsection{Succinct Distributed Arguments for P from RAM SNARGs}\label{app:dargsForP}
In this section we use Flexible RAM SNARGs to construct a succinct distributed argument for $\PP$. Such RAM SNARGs are defined w.r.t some hash family with local opening. For our use, that hash family will be a succinct vector commitment, which already satisfies all of the hash family with local openings requirements. For our use, we also require that the vector commitment has the following property.\footnote{
A succinct, inverse collision-resistant VC can be instanciated by a Merkle Tree \cite{merkle1989certified}.
}
\begin{definition}[Inverse Collision-Resistance]
A VC $(\Gen, \Com, \Open, \Ver)$ is \emph{Inverse Collision-Resistant} if for any efficient adversary $\calA$, there exists a negligible function $\epsilon(\cdot)$, such that for every $\lambda\in \mathbb{N}$,
\begin{gather*}
    \prob{}
    {
    \begin{array}{ll}
         \forall i: \Ver(crs, C^*, m, i) = 1 \\
         \wedge C^* \neq C
    \end{array}
    \middle\vert
    \begin{array}{ll}
         crs \leftarrow Gen(1^\lambda, q) \\
         C^*, \{(m_i, \lambda_i)\}_{i\in [q]} = \calA(crs) \\
         C \leftarrow \Com(crs, m_1,\ldots,m_q)
    \end{array}
    } \leq \epsilon(\lambda)
\end{gather*}
\end{definition}

\begin{theorem}\label{theo:P}
Let $\Lan$ be a graph language, such that $\Lan\in \PP$. Assuming Flexible RAM SNARGs for $\PP$ and Inverse Collision-Resistant VC exist, there is a succinct distributed argument for $\Lan$.  
\end{theorem}


Let $\Lan$ be a language on graphs that is decidable in polynomial time, given the entire graph as input, and let $M_\Lan$ be the Turing machine that decides it:
$$G\in \Lan \Leftrightarrow M_\Lan(L(G)) = 1$$
Fix a vector commitment $(\VCGen, \VCCom, \VCOpen, \VCVer)$ that is inverse collision-resistant and a RAM SNARG $(\SNARGGen, \SNARGPro, \SNARGVer)$ for $M_\Lan$, corresponding to the vector commitment as the hash with local opening. The succinct distributed argument for $\Lan$, $(\Gen, \Pro, \Ver)$, is defined as follows.
\begin{subfigures}\label{fig:dargsForP}
    \begin{nicefig}[h]{The Setup Procedure of Section~\ref{sec:dargsForP}: $\Gen(1^\secpar, n)$ }{fig:dargsForP_Setup}
        \begin{algorithmic}[1]
            \State Compute: $\crsvc = \VCGen(1^\secpar, n)$
            \State Compute: $\crssnarg = \SNARGGen(1^\secpar, 1^n)$
            \State Output: $\crs = (\crsvc, \crssnarg)$
        \end{algorithmic}
    \end{nicefig}
    
    \begin{nicefig}[h]{The Prover of Section~\ref{sec:dargsForP} : $\Pro(\crs, G)$ }{fig:dargsForP_Prover}
        \begin{algorithmic}[1]
            \State Parse $\crs=(\crsvc, \crssnarg)$
            \State Represent $G$ as an adjacency list $L = L_1,\ldots, L_n$.
            \ForEach{$i\in [n]$}
                \State Set $v\in V(G)$ to be the node with the $i$ smallest identifier.
                \State Set $i_v = i$, $L_{i_v} = (v, N_G(v_i))$            
            \EndFor
            \State Compute $(d,\aux) = \VCCom(\crsvc, L_1,\ldots, L_n)$
            \State Compute for every $i$: $\Lambda_i = \VCOpen(\crsvc, L_i, i, aux)$
            \State Compute $\pisnarg, b = \SNARGPro(\crssnarg, L)$
            \State Output $\{\pi_v\}_{v\in V(G)}$, where for every $v\in V(G)$: $\pi_v = (d, i_v, \Lambda_{i_v}, \pisnarg)$
        \end{algorithmic}
    \end{nicefig}

    \begin{nicefig}[h]{The Verifier of Section~\ref{sec:dargsForP} : $\Ver(\crs, \var{N}, v, \pi)$ }{fig:dargsForP_Verifier}
        \begin{algorithmic}[1]
            \State Parse $\crs=(\crsvc, \crssnarg)$
            \State Parse $\pi = (c, i, \Lambda_i, \pisnarg)$
            \State Verify that $\forall u\in \var{N}$, $d(u) = d$ and $\pisnark(u) = \pisnark$ (otherwise output 0)\label{step:verifyComP}
            \State Output 1 if the following holds:
            \begin{itemize}
                \item $\VCVer(\crsvc, d, (v,\var{N}), i, \Lambda_i) = 1$
                \item $\SNARGVer(\crssnarg, d, \pi) = 1$
            \end{itemize}
        \end{algorithmic}
    \end{nicefig}
\end{subfigures}


We now prove the following statement, from which Theorem~\ref{theo:P} follows.
\begin{claim}
$\Gen, \Pro, \Ver$ is a succinct distributed argument.
\end{claim}

\begin{proof}%\label{step:soundnessBreakP}
    Completeness and succinctness follow immediately from the completeness and the succinctness of the VC and the SNARG. We proceed to the proof of soundness.
    Assume towards contradiction that there exists an efficient prover $\Pro^*$ and a non-negligible function $\alpha(\cdot)$ such that:
    \begin{gather}\label{eq:soundnessBreakP}
        \prob{}
        {
        \begin{array}{ll}
        G\notin \Lan \\
        \wedge\ \forall v\in V(G) : \Ver(\var{crs}, \var{N}_G(v), v, \pi_v) = 1
        \end{array}
        \middle\vert
        \begin{array}{ll}
        \var{crs} \leftarrow \Gen(1^\lambda, n) \\
        G, \{\pi_{v}\}_{v\in V(G)} \leftarrow \Pro^*(\var{crs}, 1^\lambda, 1^n) \\
        \end{array}
        } \geq \alpha(\lambda)
    \end{gather}

    First, note that since all nodes verify the consistency of $d$ and $\pisnarg$ with their neighbors,(in Step~\ref{step:verifyComP}), if all nodes accept, then the prover gave the same commitment (digest), and the same SNARG proof $\pi_{\SNARG}$ to all of the nodes.
    
    We use $\Pro^*$ to construct an efficient adversary $\Adv$ that breaks either one of the properties of the SNARG, or one of the properties of the VC. Let $G, \{\pi_{v}\}_{v\in V(G)}$ be the graph and the proofs outputted from $\Pro^*(\crs, 1^\lambda ,n)$, and let $\crsvc, \crssnarg$ be the parsed reference strings from $\crs$.
    On input $\crs, 1^\lambda, n$, $\Adv$ outputs $d, d^*, \pi_0, \pi_1$, where:
    \begin{itemize}
        \item $d^*$ is the commitment in $\pi_v$ that is consistent across all $v\in V(G)$.
        \item $d, \aux = \VCCom(\crsvc, L(G))$. ($\Adv$ doesn't output $\aux$ but we refer it later)
        \item $\pi_0 = \SNARGPro(\crssnarg, L(G))$.
    \end{itemize}
    For every $v\in V(G)$, let $\Lambda_{i_v} = \VCOpen(\crsvc, L_{i_v}, i_v, \aux)$.
    We define the following events.
    \begin{itemize}
        \item Let $\VCCompBreak$ be the event that $d\neq d^*$ and there exists some $v\in V(G)$ such that $\VCVer$ rejects $(\crsvc, d, (v,N_G(v)), i \Lambda_{i_v})$.
        \item Let $\ICRBreak$ be the event that $d\neq d^*$ $\VCVer$ accepts $(\crsvc, d, (v,N_G(v)), i \lambda_{i_v})$.
        \item Let $\SNARGComBreak$ be the event that $\SNARGVer$ rejects $(\crssnarg, d, 0, \pi_0)$.
        \item Let $\SNARGSoundBreak$ be the event that $d = d^*$ and $\SNARGVer$ accepts $(\crssnarg, d, 1, \pi_1)$ and $(\crssnarg, d, 0, \pi_0)$.
    \end{itemize}
    Whenever the event in \ref{eq:soundnessBreakP} occurs, one of the following must hold:
    \begin{itemize}
        \item $d \neq d^*$, so either $\VCCompBreak$ or $\ICRBreak$ occur.
        \item $d = d^*$, so either $\SNARGComBreak$ or $\SNARGSoundBreak$ occur.
    \end{itemize}
    Since the event in \ref{eq:soundnessBreakP} happens with probability at least $\alpha(\lambda)$, one of the events $\VCCompBreak$, $\ICRBreak$, $\SNARGSoundBreak$, $\SNARGCompBreak$ happens with probability at least $\alpha(\lambda)/4$, which is also a non-negligible function of $\lambda$, and so, $\Adv$ breaks at least one of the following: the VC's completeness property, the VC's position-binding property, the SNARG's completeness property, the SNARG's soundness property.
Completeness, succinctness, and verifier efficiency follow naturally from the primitives' properties.
\end{proof}

\subsection{Certifying Executions of Computationally-Efficient Distributed Algorithms}
\label{app:distprover}

In the general case where we have inputs $x : V(G) \rightarrow \calX$
and outputs $y : V(G) \rightarrow \calY$,
the consistency of the local computation at a specific node is captured
by the
language $\calD$,
which consists of all triplets $(\hk, I(v), W(v))$ such that:
\begin{itemize}
	\item $\hk$ is a hash key obtained by calling $\SCRHGen$,
	\item $I(v) = (v, x(v), N(v), y(v), s_{\var{in}}(v), s_{\var{out}}(v))$,
		where
		$v \in \calU$ is the UID of a node,
		$x(v) \in \calX$ is the input of the node,
		$N(v) \in \calU^{\ast}$ is the neighborhood of the node,
		$y(v) \in \calY$ is an output value, and $s_{\var{in}}(v), s_{\var{out}}(v)$ are hash sums;
	\item $W(v) = (\var{msgout}(v), \var{msgin}(v))$ consists of two sets of messages;
	\item $(\hk, I(v), W(v)) \in \calD$ iff when the distributed algorithm $D$ is executed 
at a node with UID $v$, input $x(v)$ and neighbors $N(v)$,
and the incoming messages at node $v$ are $\var{msgin}(v)$,
the node produces output $y(v)$ and sends the messages $\var{msgout}(v)$,
and furthermore,
\begin{equation}
	s_{\var{in}} = \sum_{\var{msg} \in \var{msgin}} \SCRHHash(\hk, \var{msg}),
	\qquad
	s_{\var{out}} = \sum_{\var{msg} \in \var{msgout}} \SCRHHash(\hk, \var{msg}).
	\label{cond:msg_hash}
\end{equation}
\end{itemize}


Let $G = (V, E)$ be a graph of size $n$, and let $\ell = \poly(n)$ be the maximum encoding length of
a message sent by $D$ in graphs of size $n$.%
\footnote{Recall that the encoding of a message consists
of the round number, the edge on which it is sent, and the message contents;
for an algorithm that runs on polynomial rounds and sends polynomially-long messages,
the encoding of a message is polynomial in $n$.}


 \begin{subfigures}\label{fig:distprover}

	 \begin{nicefig}[h]{The Setup Procedure of Section~\ref{sec:distprover}: $\Gen(1^\secpar, n)$}{fig:distprover_Setup}
\begin{algorithmic}[1]
	\State $\hk \leftarrow \SCRHGen(1^\secpar, \ell(n))$
	\State $\crssnark \leftarrow \SNARKGen(1^\secpar, n')$
	\Comment{$n'$ is the encoding length of $(\hk, I(v))$ for a single vertex $v$ in graphs of size $n$}
	\State output $(\hk, \crssnark)$
\end{algorithmic}
\end{nicefig}

	 \begin{nicefig}[h]{The Distributed Prover of Section~\ref{sec:distprover}: $\Pro(\crs, (G, x, y))$}{fig:distprover_Prover}
		 The prover is the following distributed algorithm, executed jointly by the nodes of $G$.
\begin{algorithmic}[1]
	\State Parse $\crs = (\hk, \crssnark)$
	\Comment{All nodes must know the CRS}
	\State Compute the messages $\set{ \var{msgout}(v), \var{msgin}(v) }_{v \in V(G)}$ sent and received during
	the execution of $D$ at each $v \in V(G)$ \Comment{This can be done while $D$ is executing}
	%\State $\hk \leftarrow \SCRHGen( 1^{\secpar}, \ell )$
	\ForEach{$v \in V(G)$} \Comment{In parallel}
		\State $s_{\var{out}}(v) \leftarrow \sum_{\var{msg} \in \var{msgout}(v) } \SCRHHash(\hk, \var{msg})$
		\State $s_{\var{in}}(v) \leftarrow \sum_{\var{msg} \in \var{msgin}(v) } \SCRHHash(\hk, \var{msg})$
		\State $I(v) \leftarrow (v, x(v), N_G(v), y(v), s_{\var{in}}(v), s_{\var{out}}(v))$
		\State $W(v) \leftarrow ( \var{msgout}(v), \var{msgin}(v) )$
		\State $\pisnark(v) \leftarrow \Pro(\var{crs}, (\hk, I(v)), W(v))$
	\EndFor
	\State Compute a spanning tree $T$ of $G$, of height $\leq 2 \mathrm{diam}(G)$
	\State $r \leftarrow $ the root of $T$
	\State $d(r) \leftarrow 0, p(r) \leftarrow \bot$
	\State By broadcast down the tree, for each node $v$ with parent $u$,
		set $p(v) \leftarrow u, d(v) \leftarrow d(u) + 1$
	\State By convergecast up the tree, for each node $v$ set 
	$S_{\var{out}}(v) \leftarrow s_{\var{out}}(v) + \sum_{u \in \mathrm{children}(v)} S_{\var{out}}(u)$,
	$S_{\var{in}}(v) \leftarrow s_{\var{in}}(v) + \sum_{u \in \mathrm{children}(v)} S_{\var{in}}(u)$
	\State Output $\pi(v) = (p(v), d(v), r, s_{\var{out}}(v), s_{\var{in}}(v), S_{\var{out}}(v), S_{\var{in}}(v), \pisnark(v))$
	at each node $v$
\end{algorithmic}
\end{nicefig}

\begin{nicefig}[h]{The Verifier of Section~\ref{sec:distprover} }{fig:distprover_Verifier}
	The verifier at node $v$, with setup $\crs = (\hk, \crssnark)$
	and
	certificate 
	\begin{equation*}
	\pi(v) = (p(v), d(v), r(v), s_{\var{out}}(v), s_{\var{in}}(v), S_{\var{out}}(v), S_{\var{in}}(v), \pisnark(v)),
\end{equation*}
verifies the following conditions:
\begin{algorithmic}[1]
	\State $r(v) = r(u)$ for every $u \in N(v)$
	\If{$r(v) = v$}
		\State $d(v) = 0$
		\State $S_{\var{out}}(v) = S_{\var{in}}(v)$
	\Else
		\State $p(v) \in N(v)$
		\State $d(v) = d(p(v)) + 1$
	\EndIf
	\State $S_{\var{out}}(v) = s_{\var{out}}(v) + \sum_{u \in N(v) : p(u) = v} S_{\var{out}}(u)$
	\State $S_{\var{in}}(v) = s_{\var{in}}(v) + \sum_{u \in N(v) : p(u) = v} S_{\var{in}}(u)$
	\State $\SNARKVer(\crssnark, (\hk, (v, x(v), N(v), y(v), s_{\var{in}}(v), s_{\var{out}}(v))), \pisnark(v)) = 1$.
\end{algorithmic}
\end{nicefig}

\end{subfigures}


\begin{claim}
    $\Gen, \Pro, \Ver$ is a succinct distributed argument for $\Lan_D$
\end{claim}

\begin{proof}
Suppose for the sake of contradiction that there is an efficient adversary $\Adv$ such that for some
non-negligible function $\alpha(\cdot)$
and for all sufficiently large $n$,
we have
\begin{gather*}
    \prb
    {
    G\notin \Lan \\
    \wedge\ \forall v\in V(G) : \\
    \Ver(\var{\crs}, v, (x(v), y(v)),\\
    \qquad\qquad N(v), \pi(v), \pi(N(v))) = 1
    }
    {
    \var{\crs} \leftarrow \Gen(1^\secpar, n) \\
    (G, \set{\pi(v) }_{v\in V(G)}) \leftarrow \Pro^*(\var{\crs}, 1^\secpar, 1^n)
    } \geq \alpha(\secpar).
\end{gather*}
%\TODO{$\Lan$ or $\Lan_D$?}
We use $\Adv^*$ to construct an efficient adversary $\Adv'$ that breaks either the SNARK or the hash.
$\Adv'$ works as follows:
	After obtaining ${\crs} \leftarrow \Gen(1^\secpar, n)$,
	where $\crs = (\hk, \crssnark)$,
	we execute 
	$\Adv^*$ to obtain a graph $G$, along with certificates $\set{ \pi(v) }_{v \in V(G)}$
	for the nodes of $V$.
	From the certificates, $\Adv'$ extracts:
	\begin{itemize}
		\item A collection of hash-sums $\set{ s_{\var{in}}(v), s_{\var{out}}(v) }_{v \in V}$.

			As usual, let us denote $I(v) = (v, x(v), N(v), y(v), s_{\var{in}}(v), s_{\var{out}}(v))$
			for each $v \in V(G)$.
		\item A collection of SNARK proofs $\set{ \pisnark(v) }_{v \in V}$.
		\item 
			Recall that the SNARK proof $\pisnark(v)$ is for the statement 
			``$\exists W(v) : (\hk, I(v), W(v)) \in \calD$''.
			Using the extraction algorithm
			$\SNARKExt{\Adv^*}(\crssnark, (\hk, I(v))$ of the SNARK,
			we extract a witness for each node $v$
			in the form of message sets,
			$W(v) = (\var{msgin}(v), \var{msgout}(v))$.
	\end{itemize}


	Let $\var{ST}$ be the event that the verificiation of the spanning tree
			distances, root and partial sums succeeds at all nodes.
			An easy induction on the height of the tree
			shows that when this event occurs
			we have
			\begin{equation}
				S_{\var{in}}(r) =
				\sum_{v \in V} s_{\var{in}}(v),
				\qquad
				S_{\var{out}}(r) = \sum_{v \in V} s_{\var{out}}(v),
				\label{cond:partial_sums}
			\end{equation}
		where $r \in V$ is the the unique node specified in all the certificates
		as the root of the spanning tree.
		From now on, we condition on this event,
		and refer to $r$ as ``the root''.
		Recall also that as part of the partial sum verification,
		the root verifies that
		\begin{equation}
			S_{\var{in}}(r) = S_{\var{out}}(r).
			\label{cond:sums_equal}
		\end{equation}


	We claim that if $G \not \in \calL$,
	whenever all nodes accept, one of the following events must occur:
	\begin{itemize}
		\item $\var{HashCheat}$:
			the two collections of messages
			given by $\var{Out} = \set{ \var{msg} \in \var{msgout}(v) : v \in V }$
			and by $\var{In} = \set{ \var{msg} \in \var{msgin}(v) : v \in V}$
			are not equal, but they
			break the SCRH property by having
			$\sum_{ m \in \var{Out} }\SCRHHash(\hk, m) = \sum_{ m' \in \var{In} }\SCRHHash(\hk, m')$;
			or
		\item $\var{SnarkCheat}$:
			at some node $v$ we have
			%the prover has found a proof $\pi_{SNARK}(v)$ and a witness $W_v$
			%such that
			$\SNARKVer(\crssnark, (\hk, I(v)), \pisnark(v)) = 1$
			but $M$ rejects $(\hk, I(v), W(v))$,
			violating the argument of knowledge property of the SNARK.
	\end{itemize}
	To see this, suppose that all nodes accept,
	but neither event occurs. We divide into cases:
	\begin{itemize}
		\item Suppose that $M$ accepts $(\hk, I(v), W(v))$ at all $v \in V(G)$
			(i.e., $\var{SnarkCheat}$ has not occurred),
			and the message collections $\var{Out}$ and $\var{In}$ are not equal.
			Recall that $M$ verifies~\eqref{cond:msg_hash},
			and that the event $\var{ST}$ on which we are conditioning 
			implies~\eqref{cond:partial_sums},~\eqref{cond:sums_equal}.
			But~\eqref{cond:msg_hash} and~\eqref{cond:partial_sums}
			together imply that
			\begin{equation*}
				S_{\var{in}}(r) = \sum_{m \in \var{In}} \SCRHHash(\hk, m),
				\qquad
				S_{\var{out}}(r) = \sum_{m \in \var{Out}} \SCRHHash(\hk, m),
			\end{equation*}
			and~\eqref{cond:sums_equal} implies that
			the two hash-sums are equal, $S_{\var{in}}(r) = S_{\var{out}}(r)$;
			thus, $\var{HashCheat}$ has occurred.
		\item Now suppose that the message collections $\var{Out}$ and $\var{In}$ \emph{are} equal,
			but $G \notin \calL_D$.
			Then there is some node $v \in V$ such that $y(v)$ is not the output of the
			distributed algorithm $D$ at node $v$ when executed in $G$.


			Let us say that a message $\var{msg} = (r, \set{u,w}, m)$
			is \emph{correct} if it would be sent in the execution of $D$ in $G$.
			There are two cases:
			\begin{itemize}
				\item All messages in $W(v) = (\var{msgin}(v), \var{msgout}(v))$
					are correct.
					In this case, $M( \hk, I(v), W(v))$
					rejects, as it is not true that $v$
					produces the output $y(v)$ (which appears in $I(v)$)
					when $D$ is executed in $G$.
				\item Some message in $W(v) = (\var{msgin}(v), \var{msgout}(v))$
					is not correct.
					In this case,
				let $t$ be the first round such that $\var{In} = \var{Out}$ 
			includes some incorrect round-$t$ message $\var{msg} = (t, \set{u,v}, m)$,
			and let $u$ be the node such that $\var{msg} \in \var{msgout}(u)$.
			Then $M$ rejects $(\hk, I(u), W(u))$:
			at round $t$, if fed the messages in $\var{msgin}(u)$ up to round $t - 1$
			(which are all correct),
			it is not true that $u$ sends $\var{msg}$ (as this message is incorrect).
			\end{itemize}
			In both cases, $M$ rejects $(\hk, I(u), W(u))$,
			but we assumed that all nodes accept,
			and therefore $\var{SnarkCheat}(v)$ has occurred.
	\end{itemize}
 %\Enote{I think the second one should be first, then it's cleaner to suppose what we suppose}
	We conclude that one of the events $\var{HashCheat},\var{SnarkCheat}$
	occurs with probability at least $\alpha(\lambda)/2$,
	as $\Adv$ generates $G \not \in \calL$ and certificates that all nodes accept
	with probability at least $\alpha(\secpar)$,
	and whenever this occurs, at least one of the events 
 	$\var{HashCheat}, \var{SnarkCheat}$ occurs.

	Since $\alpha(\cdot)$ is non-negligible, $\alpha(\cdot) / 2$
	is also non-negligible.
	If $\var{HashCheat}$ occurs with probability at least $\alpha(\secpar) / 2$,
	this violates the SCRH property of the hash function;
	and if $\var{SnarkCheat}$ occurs with probability at least $\alpha(\secpar) / 2$,
	this violates the argument of knowledge property of the SNARK.
\end{proof}



\begin{theorem}
	\TODO{fill in --- this is a placeholder for the full theorem about the distributed prover}
	\label{thm:dist}
\end{theorem}


%\section{Concrete Instantiations for the Constructions }\label{app:concrete}
We discuss several possibilities for instantiating the primitives used in our constructions. Throughout the first part of this work (Section~\ref{sec:dargs} and Section~\ref{sec:distprover}), we use several cryptographic primitives, some of them quite strong. Here we refine them and show from which concrete assumptions each of them can be constructed. \TODO{CRH is not so concrete}

\begin{theorem}\label{thm:concDargs}
    Let $\Lan$ be a graph language, such that $\Lan\in \NP$. Assuming Collision Resistant Hash Families,
    \begin{enumerate}
        \item There is a succinct \emph{interactive} distributed argument with 4 rounds of communication for $\Lan$. 
        \item In the Random Oracle model, there is a succinct non-interactive distributed argument for $\Lan$.
        \item In the Common Reference String model, assuming \emph{Knowledge-of-Exponent in Bilinear Groups}, there is a succinct non-interactive distributed argument for $\Lan$. 
    \end{enumerate}
\end{theorem}
From Theorem~\ref{thm:centralized} and ~\cite{merkle1989certified}, we get that part (1) follows from \cite{kilian1992note}, part (2) follows from \cite{micali2000computationally} and part (3) follows from~\cite{bitansky2013SNARKsLIPs}.

\begin{theorem}\label{thm:RAMSNARGs}
    Let $\Lan$ be a graph language, such that $\Lan\in \PP$. Assuming Collision Resistant Hash Families, there is a succinct distributed argument for $\Lan$, assuming either
    \begin{enumerate}
        \item The $O(1)-\LIN$ assumption on a pair of cryptographic groups with efficient bilinear map, or
        \item A combination of the sub-exponential $\DDH$ assumption and the $\QR$ assumption.
    \end{enumerate}
\end{theorem}
As shown in  \cite{cryptoeprint:2022/1320}, Flexible RAM SNARGs exist under any of these assumptions. Together with theorem~\ref{theo:P}, this implies Theorem \ref{thm:RAMSNARGs}.

\begin{theorem}
    Let $\calD$ be a distributed algorithm that runs in $T = \poly(n)$ rounds and sends messages of length $\poly(n)$. Assuming Sum-Collision-Resistant Hash Families,
    \begin{enumerate}
        \item There is a succinct \emph{interactive} distributed argument with 4 rounds of communication for $\Lan$. 
        \item In the Random Oracle model, there is a succinct non-interactive distributed argument for $\Lan$.
        \item In the Common Reference String model, assuming \emph{Knowledge-of-Exponent in Bilinear Groups}, there is a succinct non-interactive distributed argument for $\Lan$. 
    \end{enumerate}
    Where in all of the cases above, the prover of the distributed argument has a distributed implementation.
\end{theorem}

%\section{Missing Proofs from Section~\ref{sec:local}}
\label{app:local}
\subsection{If $\LDn \cap \PP \not \subseteq \LDnP$, Then $\PP \neq \NP$}
In Section~\ref{sec:local}, we showed the separation of $\LDP$ from $\LD\cap \PP$ by extensive use of the fact that the nodes in an $\LDP$ algorithm do not know the size of a graph, and thus do not know how much time they are allowed to run. We would have wanted to prove the same without using this fact, that is, in the case where the nodes do know how much time they are allowed to run. Let $\LDn$ and $\LDnP$, be the variants of $\LD$ and $\LDP$ (resp.) where nodes have the size of the graph as part of their input.

In what follows, we demonstrate that proving this statement unconditionally would be \emph{very} unexpected. However, in Section~\ref{sec:owf}, we prove this statement conditioned on the existence of injective one-way functions.

\begin{theorem}
	If $\LDn \cap \PP \not \subseteq \LDnP$, then $\PP \neq \NP$.
	\label{thm:sep_obstacle}
\end{theorem}
\begin{proof}
	Assume that $\PP = \NP$, and let us show that every language $\calL \in \LDn \cap \PP$ is also in $\LDnP$.

	Let $\calL$ be a distributed language that is decided by some $t$-local algorithm $A \in \LDn$, 
	and also by a poly-time Turing machine $M$.
	We first modify $A$ by restricting it so that it accepts only $t$-neighborhoods that can be embedded in some instance in $\calL$, by defining a $t$-local algorithm $A'$ where
	\begin{equation*}
		A'(B) = 1 \qquad \Leftrightarrow \qquad  A(B) = 1\ \text{ and }\ \exists \tilde{G} \in \calL\ \exists u \in \tilde{G} : B = N_{\tilde{G}}^t(u)
	\end{equation*}
	Algorithms $A$ and $A'$ decide the same distributed language, $\calL$:
	every instance $\tilde{G}$ accepted by $A'$ is also accepted by $A$, since 
	whenever $A'(B) = 1$ we also have $A(B) = 1$.
	For the other direction, if an instance $\tilde{G}$ is accepted by $A$, then
	all nodes accept under $A$; also,
	$\tilde{G} \in \calL$,
	and therefore every $t$-neighborhood of $\tilde{G}$ can be embedded in some instance in $\calL$.
	Therefore all nodes accept $\tilde{G}$ under $A'$.
	
	Now let $\calH$ be the following node-language:
	\begin{equation*}
		\calH = \set{ B \in \calB^{t,\n} : \text{there is some $\tilde{G} \in \calL$ and a vertex $u \in \tilde{G}$
		such that $B = N^t_{\tilde{G}}(u)$} }.
	\end{equation*}
 
	We first show that the node-language decided by $A'$ is exactly $\calH$:
	first, suppose that $B \in \calH$. Then there is some $\tilde{G} \in \calL$ and a vertex $u \in \tilde{G}$
	such that $B = N^t_{\tilde{G}}(u)$.
	Since $A'$ decides the distributed language $\calL$, when we run $A'$ in $\tilde{G}$ all nodes must accept,
	and therefore $A'(B) = A'(B_{\tilde{G}}^t(u)) = 1$.
	For the other direction, suppose that $A'$ accepts some $t$-neighborhood $B$.
	Then in particular, by definition of $A'$,
	there exists an instance $\tilde{G} \in \calL$ and a node $u \in \tilde{G}$ such that 
	$B = N_{\tilde{G}}^t(u)$.
	This implies that $B \in \calH$.

	To conclude the proof, we observe that $\calH \in \NP$:
	it is decided by a poly-time Turing machine that takes the input $B$ and witness $\tilde{G}$,
	and accepts iff $M$ accepts $\tilde{G}$ and there is some node $u \in \tilde{G}$ that has 
	$N_{\tilde{G}}^t(u) = B$.
	(Recall that every node of $\tilde{G}$ is annotated with $1^n$, where $n$ is the size of $\tilde{G}$;
	thus, the encoding length of $\tilde{G}$ is polynomial in the encoding length of the annotated $t$-neighborhood $B$.)
	Since we assumed that $\PP = \NP$, this implies that $\calH \in \PP$ as well.
	But $\calH$ is the node-language of $A'$, and $A'$ decides $\calL$, so this implies that 
	$\calL \in \LDnP$, as desired.


\end{proof}



\subsection{$\NPLD = \NLD \cap \NP$}
In this section, we show that the distinction between the local-polynomial classes and the intersection of the local and polynomial classes vanishes when introducing non-determinism. Let $\NLD$ and $\NPLD$ be the non-deterministic variants of $\LD$ and $\LDP$ (resp.). In the case of non-deterministic decision, it no longer matters whether the nodes know the network size, because one can always provide them with the size through their certificates, along with a proof that the size is correct, using a spanning tree~\cite{korman2005proof}.
In fact, the certificate can include the \emph{entire graph}.
For this reason, when we have unique identifiers, nodes can verify that the certificate
describes the network graph accurately, allowing them to locally decide whatever can be decided by a centralized algorithm;
therefore in this case
$\NP \subseteq \NLD$.
When we do not have unique identifiers, it is not possible to verify that the certificates describe the network graph,
but only that the network graph is a \emph{lift} of the graph given in the certificates;
this suffices for our purposes, because every $\NLD$ language is closed under lifts~\cite{fraigniaud2013can}.

\begin{theorem}
	Either with or without unique identifiers,
	we have $\NPLD = \NLD \cap \NP$.
	\label{thm:NPLD}
\end{theorem}
\begin{proof}
	We prove the case without identifiers, since it is the more general case.

	The inclusion $\NPLD \subseteq \NLD \cap \NP$
	is easy to see, as an $\NPLD$-algorithm is in particular an $\NLD$-algorithm,
	and it can also be efficiently simulated by a polynomial-time Turing machine that is given all the nodes' certificates.

	To see the other direction of the inclusion, let $\calL \in \NLD \cap \NP$,
	let $A$ be a $t$-local algorithm for $\calL$, and let $M$ be an $\NP$-verifier for $\calL$.
	We construct the following $\NPLD$-algorithm, $B$:
	in a configuration $(G, x)$ on $n$ nodes,
	we give to each node a certificate $c(v) = ( i, (G', x'), w)$,
	where
	\begin{itemize}
		\item $i \in \set{1,\ldots,n}$ is an index,
		\item $G'$ and $x'$ represent the configuration $(G,x)$, using $\set{1,\ldots,n}$ as the vertices,
		\item $w$ is an $\NP$-witness, such that $M$ accepts $( (G', x'), w )$.%
			\footnote{Technically, $(G', x')$ is a lift of $(G, x)$, and therefore  
			$(G', x') \in \calL$ iff $(G, x) \in \calL$.}
	\end{itemize}
	The nodes locally verify that
	\begin{itemize}
		\item Their $t$-neighborhood in $G'$ is isomorphic to their true neighborhood in $G$,
			using the indices provided in the certificates as the isomorphism;
			and $x'$ correctly describes their input, again using the index.
		\item $M$ accepts $( (G', x'), w)$.
	\end{itemize}
	As shown in~\cite{fraigniaud2013can},
	the first part of the verification passes iff $(G', x')$ is a $t$-lift of $(G, x)$,
	and $\NLD$-languages are closed under lift.
	The completeness of $B$ follows from this fact.
	Soundness follows as well:
	if all nodes accept, then $(G', x')$ is a $t$-lift of $(G, x)$;
	since $M$ accepts $( (G', x'), w)$, we have $(G', x') \in \calL$,
	which implies that $(G, x) \in \calL$ as well.
\end{proof}






\end{document}
\endinput
