
%%
%% The abstract is a short summary of the work to be presented in the
%% article.
\begin{abstract}
In local distributed algorithms, nodes are traditionally allowed to have unbounded computational power.
This makes the model incomparable with centralized notions of efficient 
%computing 
computations such as $\PP$ and $\NP$.
In this paper, we study computationally-bounded distributed local decision and ask what can be achieved by computationally-efficient local algorithms and provers.

The contributions of this work are twofold.
First, we study distributed certification, where we
%are interested in
%certifying
wish to certify
that a distributed network satisfies
some
%desired
property, or that a distributed algorithm has produced correct output.
To that end, a \emph{prover} assigns to each node of the network a certificate, and the nodes then interact amongst themselves to verify the proof.
%In this work
%First,
We introduce the notion of \emph{computationally-sound distributed certification}, where instead of requiring perfect soundness against any prover, we require only that a \emph{computationally-efficient} prover must not be able to convince the network of a false statement, except with negligible probability.
%Using tools from cryptography, we show that under computational assumptions,
We show that under certain cryptographic assumptions, any property in $\NP$ can be certified using a polylogarithmic number of bits by a global prover that knows the entire network,
and any computationally-efficient distributed algorithm can be certified by an efficient distributed prover that produces certificates of polylogarithmic length in the algorithm's local computation time, round complexity, and message size.
%Furthermore, we show that the execution of any computationally efficient distributed algorithm can be certified by an efficient \emph{distributed prover} that produces certificates of polylogarithmic length in the algorithm's local computation time, round complexity, and message size.

%Next, we study the effect of restricting local distributed algorithms to be computationally efficient.
Next, we study the effect of restricting the nodes themselves to be computationally efficient.
We introduce the classes $\PLD$ and $\NPLD$ of polynomial-time local decision and nondeterministic polynomial-time local decision, respectively, and compare them to the centralized complexity classes $\PP$ and $\NP$, and to the distributed classes $\LD$ and $\NLD$, which correspond to local deterministic and nondeterministic decision, respectively.
We show that when the size of the network is not known to the nodes, $\PLD \subsetneq \LD \cap \PP$; that is, there exists a problem that can be decided by an inefficient local algorithm and also by a poly-time centralized algorithm, but not by a poly-time local algorithm.
When the size of the network is known, an unconditional separation of $\PLD$ from $\LD \cap \PP$ would imply that $\PP \neq \NP$;
%but under a computational assumption called Decisional Diffie-Hellman (DDH), we are still able to show that $\PLD \subsetneq \LD \cap \PP$.
however, we are still able to show that $\PLD \subsetneq \LD \cap \PP$ assuming that injective one-way functions exist.
When nondeterminism is introduced, the distinction vanishes, and $\NPLD = \NLD \cap \NP$.
\end{abstract}

